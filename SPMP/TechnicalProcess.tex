\chapter{Technical Process}
	
			\section{Methods, tools and techniques}
			
			\begin{itemize}
				
				\item Google Code \\
				There has been chosen for Google Code because it has some 
				handy features:
				
				\begin{itemize}
			 		\item Revision control system: Subversion in our case, see below.
					\item Issue tracker
					\item Wiki (is sometimes handy e.g. for meeting agenda).
					\item File download server (for our software and document releases).
				\end{itemize}

				\item Subversion \\
				This is a centralized version control system chosen for the 
				following reasons:
				
				\begin{itemize}
					\item It is easy to refactor the source code structure, 
					while preserving files' history.
					\item The whole repository has a single revision that is 
					incremented after each commit.
				\end{itemize}

				\item Eclipse \\
				Eclipse is a software development environment comprising an IDE and 
				a plug-in system to extend it. It is used to develop applications in 
				Java. It was decided to work with Eclipse because 
				it has a very userfriendly interface and because of its popularity. Also it's 
				known by most of the team members.
				
				\item \LaTeX \\
				 \LaTeX{} has been chosen to document this project because it's 
				internationally known and commonly used. Also, a few members 
				were interested in learning this language.
				
			\end{itemize}	
			
			More specific tools will be discussed later.
			
			\section{Software documentation}
			
			Several documents will be publised during the execution of this project. \\
			
			\label{deadlines}
				\begin{tabular}{l l}

					\FL Document & First version available on
					\ML SPMP & October 26, 2009
					\NN SCMP & November 9, 2009
					\NN SRS & November 13, 2009
					\NN SQAP & November 23, 2009
					\NN SDD & November 27, 2009
					\NN STD & December 07, 2009
				\end{tabular}
			
			\section{Project support functions}
			
			Throughout the entire project, Joeri De Koster and Dirk Vermeir will 
			be available for any help. For Wilma related problems, Dirk Van Deun 
			can be contacted.