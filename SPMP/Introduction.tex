\chapter{Introduzione}

\section{Scopo del documento}
		
Il presente documento, oltre a fornire una panoramica del progetto, descrive il processo di management che sarà portato avanti durante le fasi progettazione e sviluppo. Il documento sarà aggiornato in maniera iterativa per offrire informazioni più precise riguardo le diverse fasi del progetto. 


\section{Evoluzione del presente documento}

Allo stato attuale del progetto, molte informazioni nel SPMP risultano incomplete o poco precise. Il documento sarà aggiornato e rivisto con l'avanzare dei lavori in modo da garantire una completa descrizione del processo manageriale condotto.


\section{Definizioni ed acronimi}

Definizioni:			
\begin{itemize}
\item \emph{Deliverables}: con il termine deliverables si ci riferisce generalmente alla documentazione tecnico/commerciale da consegnare al cliente quale risultato dell'esecuzione di una o più fasi di un progetto.
\item \emph{Scheduling}: pianificazione dei tempi e delle precedenze nell'impiego di risorse materiali ed umane per un buon svolgimento del processo di progettazione e sviluppo di un sistema software.
\item \emph{Work breakdown structure (WBS)}: rappresentazione si un progetto che consiste in una strutturazione tipicamente gerarchica delle attività che lo compongono in sotto-attività fino ad un opportuno livello di approfondimento.
\item \emph{Diagramma di Gantt}: strumento di supporto alla gestione dei progetti utilizzato principalmente nell'attività di project management quale rappresentazione dello scheduling delle attività o mansioni che costituiscono un progetto.
\end{itemize}	

Acronimi:
\begin{itemize}
\item \emph{RAD}: Requirement Analysis Document
\item \emph{SDD}: System Design Document
\item \emph{ODD}: Object Design Document
\item \emph{WBS}: Work Breakdown Structure
\end{itemize}