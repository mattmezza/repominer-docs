\chapter{Managerial process}
		
		\section{Objectives and priorities}
			
			During the development of the system, the following objectives should be kept
			in mind:
			
			\begin{enumerate}
				
				\item The project should meet the requirements of the customer
				\item Deadlines have to be respected
				\item Reusability and extensibility of the software is very important
				\item The system should be stable
				
			\end{enumerate}	
			
		\section{Assumptions, dependencies and constraints}
			
			\begin{itemize}
				
				\item No assumptions will be made.
				
				\item Dependencies \\
 					  For the project to succeed, it will depend on
					  the knowledge and the motivation of the team members	
				
				\item Constraints:
				\begin{itemize}
					\item Only open-source software is to be used
					\item The product should run in a Linux environment
					\item The product should run on the Wilma server
				\end{itemize}
				
			\end{itemize}	
			
		\section{Risk Management}
			
			Developing software introduces a lot of risks. At regular times these risks will
			be discussed in order to reduce or eliminate them. Each member is encouraged to report
			possible risk to the project leader.
			
			\subsection{Google goes down for a certain period}
			\textbf{Criticality}: + \\
			We need to make backups of our files and documentation. This will be 
			performed by the configuration team.
			
			
			\subsection{Lack of knowledge of certain tools or mechanisms}
			\textbf{Criticality}: ++++ \\
			Some tools and mechanisms will be completely new to some team members which can
			results in a lack of knowledge and inefficiency. Initially, tutorials will be made
		    as Wiki-pages. Ultimately presentations can be given if the subject is too complex or
		 	if it demands a complete explanation. 
			
			\subsection{Somebody becomes ill}
			\textbf{Criticality}: +++ \\
			This risk has already been considered in the beginning of this 
			project. This is why there is an Assistant(formerly called Backup) for 
			every Management position. If a leader or manager gets ill, the assistant 
			of that function should be able to fully understand his 
			function and replace that leader or manager, for a certain period of time.
			
			\subsection{Risk Table}
			The above risks can be ranked in a table based on the likelihood of the risks, the
			impact on the project, the cost of disposal and the priority of it. 
			This last one will be calculated as follows :
			\begin{center}
			$ priority = (11 - change) . (11 - effect) . disposal $
			\end{center}
			The lower the priority, the most impact it has on the project. 
			
			\begin{table}
				\begin{center}
			\begin{tabular}{l c c c c}
				\\
				\FL Risk & Change & Effect & Disposal & \textbf{Priority}
				\ML Google down  & 2 & 8 & 8 & \textbf{216}
				\NN Lack of knowledge & 6 & 8 & 6 & \textbf{90}
				\NN Illness & 6 & 4 & 3 & \textbf{105}
			\end{tabular}
			\end{center}
			\caption{Risk table}
			\end{table}
			  
			

		\section{Monitoring and controlling mechanisms}	
			
			\subsection{Meetings}
			Meetings will be held every week. The topics to handle at the meeting will be defined
			in an agenda and will be sent to every team member before the start of the
			meeting. Minutes will summarize the meetings and will be available on the project 
			website within 3 days after the meeting.
			
			Should one not be able to attend the meeting, he/she is requested to inform
			the project manager within 3 hours before the start of the meeting.
			
			\subsection{Timesheets}
			A global timesheet will be available every Monday before 12H00. Team 
			members will need to submit their timesheet on Sunday before 23h59
			for approval by the team manager.