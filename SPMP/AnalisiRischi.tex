\chapter{Analisi dei rischi}
\section{Identificazione dei rischi}
In questa sezione sono identificati i possibili problemi cui il progetto potrebbe andare incontro. Per ogni rischio è indicata la probabilità attesa che si verifichi e l'impatto sulla riuscita del progetto. A seguire, nella sezione 5.2, le strategie adottate per ridurre l'impatto dei rischi sul progetto e le azioni da intraprendere nell'eventualità che un rischio si verifichi.
\\ \\
\textbf{Legenda} \\ \\
\textit{Probabilità}:
\begin{itemize}
\item \textit{Bassa}:la probabilità che il rischio si verifichi è compresa nell'intervallo 0\%-30\%.
\item \textit{Media}:la probabilità che il rischio si verifichi è compresa nell'intervallo 30\%-60\%.
\item \textit{Alta}:la probabilità che il rischio si verifichi è compresa nell'intervallo 60\%-100\%.
\end{itemize}
\textit{Impatto}:
\begin{itemize}
\item \textit{Insignificante}: il verificarsi del rischio non compromette la buona riuscita del progetto.
\item \textit{Tollerabile}: classifica rischi di semplice gestione in quanto, le problematiche ad essi collegate, sono facilmente risolvibili.
\item \textit{Serio}: rischi di questo tipo possono rallentare notevolmente il progetto mettendone a rischio la buona riuscita; è necessario risolverli nel più breve tempo possibile.
\item \textit{Catastrofico}: rischi di difficile soluzione; se non affrontati per tempo portano di sicuro al fallimento.
\end{itemize}

\newcolumntype{C}[1]{>{\centering}p{#1}}
\begin{center}
\begin{tabular}{|C{3cm}|C{3cm}|C{3cm}|}
\hline 
\textbf{Rischio} & \textbf{Probabilità} & \textbf{Impatto} \tabularnewline
\hline 
Skill del team insufficienti & Bassi & Tollerabile \tabularnewline
\hline
Poca conoscenza del dominio applicativo & Media & Serio \tabularnewline
\hline
Perdita di un elemento del team & Bassa & Tollerabile \tabularnewline
\hline
Ritardo consegna task & Media & Tollerabile \tabularnewline
\hline
Implementazione non completa & Bassa & Serio \tabularnewline
\hline
Attività prolungata oltre la scadenza prevista in fase di schedule & Media & Serio \tabularnewline
\hline
Incontro settimanale annullato & Bassa & Tollerabile \tabularnewline
\hline
\end{tabular}
\end{center}

\section{Strategie adottate nel risk management}
\textsc{Skill del team insufficienti}
\begin{itemize}
\item \textit{Prevenzione}: assegnare un task con maggior coefficiente di difficoltà ai membri più preparati. Assegnare task particolari a membri del team che hanno acquisito già in precedenza una particolare skill. Durante lo svolgimento dei meeting, cercare di fornire informazioni dettagliate sullo svolgimento dei task assegnati: definire input, lavoro da svolgere e output desiderato, fornendo eventualmente riferimenti su materiale utile per lo svolgimento del task.
\item \textit{Piano di contingenza}: privilegiare task di gruppo affiancando membri del team più preparati a coloro che mostrano maggiori lacune.
\end{itemize}
\textsc{Poca conoscenza del dominio applicativo}
\begin{itemize}
\item \textit{Prevenzione}: prestare particolare attenzione nei meeting iniziali a discutere con il gruppo ogni aspetto del progetto, chiarendo eventuali dubbi, sottolineandone l'utilità e il dominio applicativo in cui va a posizionarsi. 
\item \textit{Piano di contingenza}: dedicare incontri straodinari alle discussioni inerenti il dominio applicativo. Domandare chiarimenti agli sviluppatori del progetto originale.
\end{itemize}
\textsc{Perdita di un elemento del team}
\begin{itemize}
\item \textit{Prevenzione}: assegnare task complicati, quali l'implementazione, a coppie di membri del team, riducendo così il danno nel caso di verificarsi del rischio. Fornare un gruppo omogeneo per quanto riguarda obbiettivi e aspettative sul lavoro da svolgere.
\item \textit{Piano di contingenza}: ridistribuire i task lasciati in sospeso dal membro uscente del team tra le risorse ancora a disposizione. Considerare eventuali modifiche allo scheduling.  
\end{itemize}
\textsc{Ritardo nella consegna del task}
\begin{itemize}
\item \textit{Prevenzione}: fornire incentivi alle risorve nel caso di consegna puntuale del task, penalizzazioni altrimenti. In sede di meeting iniziali far notare che un ritardo su un task potrebbe causare un ritardo nella consegna del progetto collettivo, e quindi un danno per tutti i singoli elementi del team.
\item \textit{Piano di contingenza}: assegnare task più impegnativi a membri del team che dimostrano puntualità nella consegna; considerare eventuali modifiche allo scheduling.
\end{itemize}
\textsc{Implementazione non completa}
\begin{itemize}
\item \textit{Prevenzione}: valutare attentamente il monte ore a disposizione nello stabilire i moduli del sistema da implementare, tenendo conto anche delle esigenze dei membri del team.
\item \textit{Piano di contingenza}: documentare le difficoltà incontrare e motivare il verificarsi del rischio.
\end{itemize}
\textsc{Attività prolungata oltre la scadenza prevista in fase di schedule}
\begin{itemize}
\item \textit{Prevenzione}: monitorare attentamente i progressi del progetto lungo la sua durata.
\item \textit{Piano di contingenza}: riformulare lo schedule ed inviduare possibili attività sulle quali recuperare il ritardo accumulato. Se il ritardo è così grave da pregiudicare la consegna finale del progetto, valutare la possibilità di procrastinarne la stessa.
\end{itemize}
\textsc{Incontro settimanale annullato}
\begin{itemize}
\item \textit{Prevenzione}: stabilire gli incontri settimanalmente in accordo alle necessità e alla disponibilità di tutti i membri del team.
\item \textit{Piano di contingenza}: recuperare l'incontro o affidarsi a mezzi di comunicazione sincrona disponibili sul web.
\end{itemize}

