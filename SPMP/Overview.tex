\chapter{Overview del progetto}
\section{Obiettivi del progetto}
L'obiettivo del progetto è quello di aggiungere al framework RepoMiner un'estensione che permetta di calcolare l'entropia di un sistema software, implementando le metriche descritte nell'assignment.

\section{Assunzioni e vincoli}
RepoMinerEvo sarà realizzato come estensione del progetto RepoMiner, scritto in Java. Si assume che il progetto originario funzioni correttamente; questo sarà gestito ed evolverà indipendentemente da RepoMinerEvo.

Un vincolo, dettato dalla necessità di permettere a tutti i componenti del team di svolgere l'esame liberamente, è di completare il progetto entro il 23 giugno 2014. 
\begin{table}[b]
	\begin{tabular}{|c|c|c|}
	\hline
	\textbf{Nome risorsa} & \textbf{Tipologia} & \textbf{Attività}\\
	\hline
	Computer	& Hardware	& Redazione di documenti e scrittura di codice\\
	\hline
	Subversion	& Software	& Gestione delle versioni\\
	\hline
	Eclipse		& Software	& Implementazione del codice\\
	\hline
	Editor \LaTeX & Software& Redazione documenti\\
	\hline
	\end{tabular}
	\caption{Risorse hardware/software}
	\label{overview:risorsehwsw}
\end{table}
\section{Analisi risorse hardware e software}
In questa sezione sono indicate le risorse hardware e software necessarie allo sviluppo del progetto. La tabella \ref{overview:risorsehwsw} mostra l'elenco delle risorse che si prevede di utilizzare, con le relative attività a cui sono legate.

\section{Stima dei costi del progetto}
Si stima che il budget necessario per completare il progetto sia di 134 ore/uomo. Si prevede che lo sforzo sarà così ripartito:\\
\begin{itemize}
\item Analisi dei requisiti: 12h
\item Impact analysis: 10h
\item Object design: 18h
\item Implementazione: 30h
\item Testing: 40h
\item Meeting: 12h
\item Studio degli articoli scientifici forniti: 12h
\end{itemize}

Si provvederà ad una correzione di tali stime nei report di management.

\section{Deliverable del progetto}
I deliverable che saranno prodotti durante il progetto sono riportati nella tabella \ref{overview:deliverable}. Per RAD e SDD si farà riferimento alla documentazione esistente.

\begin{table}[b]
	\begin{tabular}{|c|c|c|c|}
	\hline
	\textbf{Codice} & \textbf{Nome} & \textbf{Responsabilità} & \textbf{Data di consegna}\\
	\hline
	DL1	& SPMP				& Tutti			 & 15/05/2014\\
	\hline
	DL2	& Analisi Requisiti	& Tutti			 & 19/05/2014\\
	\hline
	DL3	& ODD				& Tutti			 & 24/05/2014\\
	\hline
	DL4 & Test plan			& ???			 & 1/06/2014\\
	\hline
	DL5 & Codice sorgente (1)& ???			 & 1/06/2014\\
	\hline
	DL6 & Test reports (1)	& ???			 & 7/06/2014\\
	\hline
	DL7 & Codice sorgente (2)& ???			 & 7/06/2014\\
	\hline
	DL8 & Test reports (2)	& ???			 & 14/06/2014\\
	\hline
	DL9 & Codice sorgente (3)& ???			 & 14/06/2014\\
	\hline
	DL10& Test reports (3)	& ???			 & 21/06/2014\\
	\hline
	\end{tabular}
	\caption{Deliverable del progetto}
	\label{overview:deliverable}
\end{table}