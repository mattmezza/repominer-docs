\chapter{Overview del progetto}
\section{Obiettivi del progetto}
L'obiettivo del progetto è quello di aggiungere al framework RepoMiner un'estensione che permetta di calcolare l'entropia di un sistema software, implementando le metriche descritte nell'assignment.

\section{Assunzioni e vincoli}
RepoMinerEvo sarà realizzato come estensione del progetto RepoMiner, scritto in Java. Si assume che il progetto originario funzioni correttamente; questo sarà gestito ed evolverà indipendentemente da RepoMinerEvo.

Un vincolo, dettato dalla necessità di permettere a tutti i componenti del team di svolgere l'esame, è di completare il progetto entro il 13 luglio 2014. 
\begin{table}[b]
	\begin{tabular}{|c|c|c|}
	\hline
	\textbf{Nome risorsa} & \textbf{Tipologia} & \textbf{Attività}\\
	\hline
	Computer		& Hardware	& Redazione di documenti e scrittura di codice\\
	\hline
	Subversion		& Software	& Gestione delle versioni del codice\\
	\hline
	Draw.io			& Software	& Diagrammi UML\\
	\hline
	GanttProject	& Software	& Diagrammi Gantt\\
	\hline
	Eclipse			& Software	& Implementazione del codice\\
	\hline
	Editor \LaTeX 	& Software	& Redazione documenti\\
	\hline
	Git		 		& Software 	& Gestione delle versioni della documentazione\\
	\hline
	\end{tabular}
	\caption{Risorse hardware/software}
	\label{overview:risorsehwsw}
\end{table}
\section{Analisi risorse hardware e software}
In questa sezione sono indicate le risorse hardware e software necessarie allo sviluppo del progetto. La tabella \ref{overview:risorsehwsw} mostra l'elenco delle risorse che si prevede di utilizzare, con le relative attività a cui sono legate.

\section{Stima dei costi del progetto}
Si stima che il budget necessario per completare il progetto sia di 570 ore/uomo. Si prevede che lo sforzo sarà così ripartito:\\
\begin{itemize}
\item Analisi dei requisiti: 50h
\item Impact analysis: 30h
\item Object design: 90h
\item Implementazione: 150h
\item Testing: 220h
\item Meeting: 20h
\item Studio degli articoli scientifici forniti: 10h
\end{itemize}

Si provvederà ad una correzione di tali stime nei report di management.

\section{Deliverable del progetto}
I deliverable che saranno prodotti durante il progetto sono riportati nella tabella \ref{overview:deliverable}. Per RAD e SDD si farà riferimento alla documentazione esistente.\\
La responsabilità della revisione dei deliverable è sempre di tutto il team; sono però qui specificati i componenti del team che provvederanno maggiormente alla creazione e al mantenimento di determinati artefatti. Le sigle indicano le iniziali (nome e cognome).

\begin{table}[b]
	\begin{tabular}{|c|c|c|c|}
	\hline
	\textbf{Codice} & \textbf{Nome} & \textbf{Responsabilità} & \textbf{Scadenza}\\
	\hline
	DL1	& SPMP						& Tutti			& 15/05/2014\\
	\hline
	DL2	& Analisi Requisiti			& Tutti			& 19/05/2014\\
	\hline
	DL3	& ODD						& MM e GG		& 9/07/2014\\
	\hline
	DL4 & Test Plan					& CB e SS		& 27/05/2014\\
	\hline
	DL5 & Test Case Specification	& CB e SS	 	& 1/06/2014\\
	\hline
	DL6 & Codice sorgente (1)		& MM e GG		& 31/05/2014\\
	\hline
	DL7 & Test reports (1)			& CB e SS		& 5/06/2014\\
	\hline
	DL8 & Codice sorgente (2)		& MM e GG		& 9/06/2014\\
	\hline
	DL9 & Test reports (2)			& MM e GG		& 28/06/2014\\
	\hline
	DL10& Codice sorgente (3)		& CB e SS		& 2/07/2014\\
	\hline
	DL11& Test reports (3)			& CB e SS		& 7/07/2014\\
	\hline
	\end{tabular}
	\caption{Deliverable del progetto. I numeri \emph{1, 2 e 3} tra parentesi si riferiscono 
			all'incremento in cui saranno creati quei deliverable.}
	\label{overview:deliverable}
\end{table}