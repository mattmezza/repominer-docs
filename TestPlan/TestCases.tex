\chapter{Casi di test}
Per sviluppare i test cases sarà utilizzato il metodo del Category Partition. Il metodo scelto ha numerosi limiti, ma, tuttavia, è il più semplice da applicare. Analizzare le condizioni limite consentirà di effettuare un test più affidabile.

\paragraph{Specifica test formale\\}

\begin{tabular}{lll}
x		&								&									\\
1.		&	0							&	[error]							\\
2.		&	1							&	[property stringok, length1]	\\
3.		&	2-19						&	[property stringok, midlength]	\\
4.		&	20							&	[property stringok, length20]	\\
5.		&	21							&	[error]							\\
		&								&									\\
a		&								&									\\
1.		&	Length 1					&	[if stringok and length1]		\\
2.		&	Length 2-19 				&	[if stringok and midlength]		\\
3.		&	Length 20					&	[if stringok and length20]		\\
		&								&									\\
c		&								&									\\
1.		&	At first position in string	&	[if stringok]					\\
2.		&	At last position in string	&	[if stringok and not length1]	\\
3.		&	In middle of string			&	[if stringok and not length1]	\\
4.		&	Not in string				&	[if stringok]					\\
\end{tabular}

\vspace{1cm}

		\newcolumntype{C}[1]{>{\centering}p{#1}}
		\begin{table}[ht]
		\centering
			\begin{tabular}{|p{4cm}|p{4cm}|p{4cm}|}
				\hline
				\rowcolor{Gray}
				\textbf{Codice} & \textbf{Combinazione} & \textbf{Esito}\tabularnewline
				\hline
				TC 0.01 		& vpiv1 				& errore \tabularnewline
				\hline
				TC 0.02 		& vpiv1vcp1				& errore \tabularnewline
				\hline
				TC 0.03 		& vpiv2vcp2vcb1 		& errore \tabularnewline
				\hline
				TC 0.04 		& vpiv2vcp2vcb2 		& parsing \tabularnewline
				\hline
			\end{tabular}
		\label{Test Case Parsing Field}
		\end {table}