\chapter{Casi di test}
Per sviluppare i test cases sarà utilizzato il metodo del Category Partition. Il metodo scelto ha numerosi limiti, ma, tuttavia, è il più semplice da applicare. Analizzare le condizioni limite consentirà di effettuare un test più affidabile.

%Simone:
%#specifica test per calcolo del numero medio di volte in cui i file di un package hanno subito operazioni di bug fixing
%#specifica test per calcolo del numero di autori di commit effettuati all'interno di un package
%#specifica test per calcolo del numero di linee aggiunte o rimosse (somma, media e massimo)
%#specifica test per calcolo della dimensione media dei file modificati
%#specifica test per calcolo del'extended change model
%#calcolo adapting size entropy

\paragraph{Metrica per il calcolo del numero medio di file cambiati in un package\\}

\begin{tabular}{|l|l|l|l|}
\hline
File del package			&			&			&				\\
\hline
		&	Numero (fnum)				&			&				\\
		&	1.					&	0		&	[property Empty]	\\
		&	2.					&	1		&	[property NonEmpty]	\\
		&	3.					&	Più di 1	&	[property NonEmpty]	\\
		&	Tipo (ftype)				&			&				\\
		&	1.					&	Testo		&	[if NonEmpty][property Text]\\
		&	2.					&	Binario		&	[if NonEmpty][property Bin]\\
		&	Versioning (fver)			&			&				\\
		&	1.					&	Sì		&	[if NonEmpty][property Vers]\\
		&	2.					&	No		&	[if NonEmpty][property NoVers]\\
\hline
Insieme di cambiamenti	&					&			&				\\
\hline
		&	Numero (cnum)				&			&				\\
		&	1.					&	0		&	[property NoChanged]	\\
		&	2.					&	1		&	[property Changed]	\\
		&	3.					&	Più di 1	&	[property Changed]	\\
		&	Cancellazioni (ccan)			&			&				\\
		&	1.					&	0		&	[if Changed][property NoDel]\\
		&	2.					&	Almeno 1	&	[if Changed][property Del]\\
		&	Inserimenti (cins)			&			&				\\
		&	1.					&	0		&	[if Changed][property NoIns]\\
		&	2.					&	Almeno 1	&	[if Changed][property Ins]\\
		&	Messaggio (cmes)			&			&				\\
		&	1.					&	Vuoto		&	[if Changed][property NoMsg]\\
		&	2.					&	Non vuoto	&	[if Changed][property Msg]\\
\hline
Periodo di tempo	&					&			&				\\
		&	Tipologia (pclass)			&			&				\\
		&	1.					&	Tempo		&	[property Time]		\\
		&	2.					&	Cambiamenti	&	[property Changes]	\\
		&	3.					&	Burst		&	[property Burst]	\\
		&	4.					&	Altro		&	[error]			\\
		&	Tipo (ptype)				&			&				\\
		&	1.					&	Non numerico	&	[if not Burst][error]	\\
		&	2.					&	Numerico	&	[if not Burst]		\\
		&	Lunghezza (plun)			&			&				\\
		&	1.					&	$\leq 0$	&	[error]			\\
		&	2.					&	$> 0$		&				\\
		&	3.					&	$> esistenza del sistema$ & 	[if Time][error]\\
		&	4.					&	$> cambiamenti totali$	&	[if Changes][error]\\
\hline
\end{tabular}

\vspace{1cm}

		\newcolumntype{C}[1]{>{\centering}p{#1}}
		\begin{table}[ht]
		\centering
			\begin{tabular}{|p{4cm}|p{4cm}|p{4cm}|}
				\hline
				\rowcolor{Gray}
				\textbf{Codice} & \textbf{Combinazione} & \textbf{Esito}\tabularnewline
				\hline
				TC 0.01 		& vpiv1 			& errore \tabularnewline
				\hline
				TC 0.02 		& vpiv1vcp1			& errore \tabularnewline
				\hline
				TC 0.03 		& vpiv2vcp2vcb1 		& errore \tabularnewline
				\hline
				TC 0.04 		& vpiv2vcp2vcb2 		& parsing \tabularnewline
				\hline
			\end{tabular}
		\label{Test Case Parsing Field}
		\end {table}