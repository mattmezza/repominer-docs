\chapter{Approccio}
Nella sessione di testing di RepoMinerEvo verrà utilizzato un approccio di tipo ``Category Partition'', che permette di avere un sottoinsieme intelligente dei casi di test ottenibili attraverso SECT (Strong Equivalence Class Testing). Si cercherà di minimizzare le scelte delle categorie (selezionando solo quelle ritenute necessarie), in modo da evitare l'esplosione combinatorica dei casi di test.\\
Il testig si dividerà in due fasi distinte:\\
\begin{itemize}
\item Testing di unità, che controlla i singoli componenti (classi, metodi)\\
\item Testing funzionale, che andrà a verificare la funzionalità dell’intero sistema assemblato\\
\end{itemize}
Non ci sarà una fase di testing di integrazione, dato che il sistema da realizzare è piccolo. La strategia di integrazione che si seguirà sarà di tipo big-bang.
\section{Testing di unità}
Con il testing di unità verrà effettuato un controllo delle varie classi (e dei relativi metodi) del sistema, quindi saranno ricercate le condizioni di fallimento andando ad evidenziare gli errori. Il testing di unità sarà eseguito dal team di sviluppo; potranno essere realizzate classi JUnit (per favorire il testing di regressione) o si potrà procedere ad un test manuale delle classi, qualora il tempo non sia sufficiente. Nel secondo caso, gli sviluppatori saranno liberi di scegliere alcuni dei casi di test definiti nel TCS.\\
Il team di testing, inoltre, provvederà alla creazione di file SQL in grado di popolare il database simulando il sistema RepoMiner. Nello specifico, sarà scritto un file SQL per ogni progetto di test individuato nel Test Case Specification.

\section{Testing funzionale}
Con il testing di sistema verrà effettuato un controllo della correttezza dell’intero sistema. Questo test sarà effettuato manualmente dai componenti del team di testing, seguendo le indicazioni presenti nel TCS, per ogni caso di test specificato.