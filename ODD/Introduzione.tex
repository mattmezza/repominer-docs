\chapter{Introduzione}

Questo documento, usato come supporto dell’implementazione, ha lo scopo di produrre un modello capace di integrare in modo coerente e preciso tutti i servizi individuati nelle fasi precedenti. In particolare definisce le interfacce delle classi, le operazioni, i tipi, gli argomenti e la signature dei sottosistemi. Inoltre nei paragrafi successivi sono specificati i trade-off, le linee guida e i design pattern per l’implementazione.

\section{Compromessi dell'Object Design}
\begin{itemize}
  \item \textbf{Comprensibilità vs Costi}\\
  Considerando la comprensibilità del codice un aspetto di fondamentale importanza non solo per la manutenzione ma anche per la fase di testing del prodotto, si è scelto di utilizzare i commenti Javadoc oltre ai commenti standard. Ovviamente questa caratteristica aggiungerà dei costi allo sviluppo del progetto ma renderà ogni classe e metodo facilmente interpretabile anche da chi non ha collaborato al progetto.
  \item \textbf{Costi vs Manutenzione}\\
  T.B.D.
\end{itemize}

%------DEFINIZIONI, ACRONOMI, ABBREVIAZIONI
\section{Definizioni, acronimi, abbreviazioni}

\textsc{SQL}: Structured Query Language. \\ \\
\textsc{GUI}: Graphical User Interface. \\ \\
\textsc{ODD}: Object Design Document. \\ \\