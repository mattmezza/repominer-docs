\chapter{Glossario}
% prova longtable
\begin{center}
\begin{longtable}{lp{.6\columnwidth}}
\hline 
\rowcolor[gray]{.80}
\textbf{Termine} & \textbf{Descrizione} \tabularnewline
\hline 
Attributo & 
Un attributo rappresenta una proprietà di un oggetto; ha un nome, un tipo e può avere un valore di default. Gli attributi rappresentano lo stato dell'oggetto e non sono condivisi con altri oggetti.
\tabularnewline
\hline
Accoppiamento & 
Il grado di dipendeza tra due elementi
\tabularnewline
\hline
Classe & 
Astrazione che specifica lo stato e il comportamento di un insieme di oggetti.
\tabularnewline
\hline
Classe Astratta & 
Una classe che non ha oggetti istanziati da essa.
\tabularnewline
\hline
Costruttore & 
Una operazione che crea un oggetto e inizializza il suo stato.
\tabularnewline
\hline
Coesione & 
Il grado di parentela di un'unità incapsulata.
\tabularnewline
\hline
Database relazionali & 
Raccolta di informazioni di vario tipo, strutturate in modo da essere facilmente reperibili in base a una chiave di ricerca primaria determinata. Le tabelle contengono dati logicamente correlati e sono messe in relazione tra loro.
\tabularnewline
\hline
Interfaccia & 
L'insieme di tutte le signature definite per le operazioni di un oggetto. L'interfaccia definisce l'insieme delle richieste alle quali l'oggetto può rispondere.
\tabularnewline
\hline
JavaDoc & 
Strumento che estrae dai commenti di un programma una documentazione dettagliata del codice.
\tabularnewline
\hline
Metodi & 
Funzioni e procedure che operano sui dati di un oggetto. I programmi dovrebbero interagire con i dati di una classe classe solo attraverso i suoi metodi.
\tabularnewline
\hline
Package & 
Costrutto Java che fornisceun raggruppamento di un insieme di classe e interfacce in relazione tra loro.
\tabularnewline
\hline
Parametro & 
Una variabile passata ad un parametro quando viene chiamato. 
\tabularnewline
\hline
Polimorfismo & 
Oggetti diversi in grado di rispondere allo stesso messaggio in modi diversi; permette agli oggetti di interagire tra loro senza conoscere il tipo esatto.
\tabularnewline
\hline
Override & 
A volte è necessario eseguire l'override (ridefinizione) di attributi e/o metodi in sottoclassi.
\tabularnewline
\hline
Signature & 
Firma di un metodo; è infatti costruita dal nome dell'operazione, dalla lista completa dei parametri e dal tipo del valore di ritorno.
\tabularnewline
\hline
Superclasse & 
Se la classe B eredita dalla classe A; si dice che A è una superclasse di B.
\tabularnewline
\hline
Sottoclasse & 
Se la classe B eredita dalla classe A, diciamo che B è una sottoclasse di A.
\tabularnewline
\hline
\end{longtable}
\end{center}




