\chapter{Glossario}
% prova longtable
\begin{center}
\begin{longtable}{lp{.6\columnwidth}}
\hline 
\rowcolor[gray]{.80}
\textbf{Termine} & \textbf{Descrizione} \tabularnewline
\hline 
SIS & 
Starting Impact Set: artefatti che, dopo una analisi delle specifiche della change request, si presume di dover intaccare andando a implementare il cambiamento richiesto
\tabularnewline
\hline
CIS & 
Candidate Impact Set: ulteriore livello di analisi sugli artefatti individuati dal SIS
\tabularnewline
\hline
Recall & 
Metrica per la valutazione del processo di impact analysis che misura la percentuale degli impatti reali inclusi in CIS
\tabularnewline
\hline
Precision & 
Metrica per la valutazione del processo di impact analysis che misura la percentuale degli impatti candidati che sono impatti reali
\tabularnewline
\hline
DIS & 
Discovered Impact Set: sottostima dell'impatto; si tratta degli artefatti che fanno parte dell'AID ma che non sono stati inclusi nel CIS al momento dell'analisi
\tabularnewline
\hline
AID & 
Actual Impact Set: insieme degli artefatti realmente modificati
\tabularnewline
\hline
FPIS & 
False Positive Impact Set: sovrastima dell'impatto; si tratta degli artefatti che sono stati inclusi nel CIS ma che non sono stati impattati
\tabularnewline
\hline
\end{longtable}
\end{center}




