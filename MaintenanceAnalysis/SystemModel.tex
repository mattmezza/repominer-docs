\chapter{System Model}
\section{Use Case Models}

%template caso d'uso
\begin{usecase}
\addtitle{Nome Use Case}{Calcolo metriche}
\addfield{ID:}{UC\_QualityMetricsExtractor\_0.1}
\addfield{Partecipanti:}{Sviluppatore}
\addfield{Condizioni di ingresso:}{Lo sviluppatore avvia la IDE "Eclipse"}
\addscenario{Flusso di eventi:}{
	\item Lo sviluppatore avvia il tool tramite il tasto "run".
	\item "SIE" apre una finestra della IDE mostrando sulla sinistra la lista dei progetti che è possibile analizzare.
	\item Lo sviluppatore seleziona il progetto, e da' avvio al programma.
	\item "SIE" notifica l'avvenuto calcolo delle metriche tramite un messaggio che compare in console.
}
\addfield{Condizioni di uscita:}{
	\begin{enumerate}
	\item[1.] Il sistema memorizza i dati.
	\item[2.] Lo sviluppatore termina precocemente l'operazione di calcolo.
	\end{enumerate}
}
\addfield{Eccezioni:}{Errore di sistema}
\end{usecase}

\section{Object Models}
