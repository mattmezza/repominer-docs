\chapter{Impact Analysis}
In questa sezione si presenterà una previsione dell'impatto che la modifica richiesta avrà sugli artefatti del sistema esistente. In particolar modo si cercherà di individuare le classi (codice sorgente) che dovranno essere modificate per integrare le funzionalità richieste.

Per effettuare l'analisi è stato utilizzato il tool JRipples, grazie al quale è stato possibile derivare automaticamente il Candidate Impact Set dallo Starting Impact Set definito nella prima fase. 

\paragraph{}
Dato che si propone di aggiungere nuove funzionalità alla struttura esistente e che queste non necessitano della modifica di funzionalità esistenti, si presume che l'impatto sul sistema esistente sarà scarso. 

In particolare, si presume che lo Starting Impact Set conterrà le seguenti classi:
\begin{itemize}
\item AAAA 
\end{itemize}

A seguito dell'analisi condotta automaticamente da JRipples, è stato individuato il seguente Candidate Impact Set:
\begin{itemize}
\item AAAA
\end{itemize}

Una valutazione dettagliata di questa analisi sarà presentata in un report separato dopo l'implementazione completa delle modifiche.