\chapter{Introduction}
Il presente documento si propone di fornire una rapida panoramica del progetto, focalizzandosi sui requisiti funzionali da implementare a partire da una base già esistente. Si provvederà a specificare un design di alto livello delle funzionalità che si andranno a raggiungere. In questo documento andrà a confluire anche tutta l'attività di impact analysis.

%-------SCOPO DEL SISTEMA
\section{Scopo del sistema}
Scopo del sistema è quello di aggiungere funzionalità nuove ad un sistema già esistente, RepoMiner. Tale sistema è in grado di analizzare codice sorgente ed estrarre da esso una serie di metriche. 

Si vuol essere in grado con il sistema proposto, di sfruttare come base di conoscenza le metriche estratte da RepoMiner per calcorare il grado di entropia presente nei package di un sistema software.

%------DEFINIZIONI, ACRONOMI, ABBREVIAZIONI
\section{Definizioni, acronimi, abbreviazioni}
\textsc{Entropia}: in meccanica statistica, l'entropia rappresenta una grandezza che viene interpretata come una misura del disordine presente in un sistema fisico, incluso, nel caso limite, l'universo. L'entropia è una funzione crescente della probabilità dello stato macroscopico di un sistema, e precisamente risulta proporzionale al logaritmo del numero delle configurazioni microscopiche possibili per quello stato macroscopico: la tendenza all'aumento dell'entropia di un sistema isolato corrisponde dunque al fatto che il sistema evolve verso gli stati macroscopici più probabili. \\

\section{Panoramica}

Nell'ambito dell' Ingegneria del Software, il concetto di entropia è stato applicato con successo in diversi contesti, a partire dal concetto di defect predictio

\section{Sistema corrente}
