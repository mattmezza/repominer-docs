\chapter{Sistema proposto}
\section{Requisiti funzionali}
Il sistema deve essere in grado di calcolare metriche relative all'entropia di un progetto software.

Più in particolare, dovrà essere possibile calcolare le seguenti metriche basilari:
\begin{itemize}
\item Numero di revisioni del sistema;
\item Numero medio di volte in cui i file di un package hanno subito
cambiamenti;
\item Numero medio di volte in cui i file di un package hanno subito operazioni
di refactoring;
\item Numero medio di volte in cui i file di un package hanno subito operazioni di
bug fixing;
\item Numero di autori di commit effettuati all’interno di un package;
\item Numero di linee aggiunte o rimosse (somma, media e massimo);
\item Dimensione media dei file modificati.
\end{itemize}

Il sistema dovrà implementare il Basic Code Change Model descritto nell'articolo di Hassan \cite{hassan2009predicting}. Dovranno essere implementate, quindi, le seguenti metriche:
\begin{itemize}
\item Numero di cambiamenti totali di ogni file (righe aggiunte + righe cancellate) di un dato progetto software a seguito di aggiunta di feature (FI) in un periodo di riferimento (es: un mese); per distinguerle dovranno essere analizzati i messaggi di commit: quelli che non contengono parole come "bug-fix", "re-indent", "copyright update", etc. sono da classificare come FI.
\item Entropia dei cambiamenti dei file del progetto in un periodo di riferimento fissato (es: un mese)
\end{itemize}

Il sistema dovrà implementare l'Extended Code Change Model descritto nell'articolo di Hassan \cite{hassan2009predicting}; dovranno essere implementate, dunque, le seguenti metriche:
