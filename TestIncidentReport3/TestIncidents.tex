\chapter{Test Incidents}

In questa sezione verranno indicati tutti i test incident. Si provvederà a fornire le seguenti informazioni per ogni caso di test:
\begin{itemize}
 \item Identificativo del caso di test (tracciabile nel Test Case Specification e nel Test Plan)
 \item Risultato prodotto e risultato atteso
 \item Input che ha portato alla failure
 \item Note aggiuntive
\end{itemize}

Per visualizzare i risultati è sufficiente eseguire la query SQL descritta nella figura \ref{figmetpack}.
\begin{figure}
 \begin{lstlisting}
  SELECT M.name, M.description, M1.value
  FROM repominer.metrics AS M, repominer.project_metrics AS M1 
  WHERE M.id = M1.metric
  \end{lstlisting}
  \caption{\footnotesize Query utilizzata per estrarre i valori relativi alle metriche di progetto}
  \label{figmetpack}
\end{figure}

\newpage
\section{Metrica per il calcolo dell'Extended Code Change model}
\begin{table}[ht]
	\centering
	\begin{tabular}{| l | l | l |}
		\hline
		\rowcolor{lightgray}\textbf{Caso di test}	&	\textbf{Risultato atteso}	&	\textbf{Risultato ottenuto}	\\
		\hline
		TC 10.1						&	Messaggio di errore		&	Errore di sistema	   	\\
		\hline
		TC 10.4						&	0				&	Errore di sistema	   	\\
		\hline
		TC 10.10					&	0				&	Metrica non calcolata	   	\\
		\hline
		TC 10.14					&	0				&	Errore di sistema	   	\\
		\hline
		TC 10.20					&	0				&	Metrica non calcolata	   	\\
		\hline
		TC 10.22					&	Messaggio di errore		&	Errore di sistema	   	\\
		\hline
		TC 10.25					&	0				&	Errore di sistema	   	\\
		\hline
		TC 10.31					&	0				&	Metrica non calcolata	   	\\
		\hline
		TC 10.36					&	0				&	Errore di sistema	   	\\
		\hline
		TC 10.42					&	1				&	Metrica non calcolata	   	\\
		\hline
		TC 10.44					&	Messaggio di errore		&	Errore di sistema	   	\\
		\hline
		TC 10.47					&	0				&	Errore di sistema	   	\\
		\hline
		TC 10.53					&	0				&	Metrica non calcolata	   	\\
		\hline
	\end{tabular}
\end{table}

\subsection*{TC 10.1}
Passi da eseguire per arrivare alla failure:
\begin{itemize}
 \item Avviare Eclipse
 \item Pulire il database
 \item Aprire le impostazioni di Eclipse (Window/Preferences)
 \item Selezionare ``History Metrics Preferences''
 \item Cliccare sul tasto ``Ok''
 \item Caricare il file ``Vuoto.sql'' nella cartella ``sql'' del progetto
 \item Cliccare sul tasto ``Calculate history metrics''
\end{itemize}
Il processo si interrompe e nel terminale di Eclipse (launcher) viene mostrato il messaggio di errore ``net.sf.jeasyorm.RuntimeSQLException: Error inserting entity''
\newline
Il progetto di riferimento ha le seguenti caratteristiche:
\begin{itemize}
 \item Non ci sono file locali
 \item Non sono presenti cambiamenti
\end{itemize}

\paragraph{Note} Le informazioni presenti nel log dell'errore mostrano che questo potrebbe essere dovuto ad un problema con il salvataggio di una data.





\newpage





\subsection*{TC 10.4}
Passi da eseguire per arrivare alla failure:
\begin{itemize}
 \item Avviare Eclipse
 \item Pulire il database
 \item Aprire le impostazioni di Eclipse (Window/Preferences)
 \item Selezionare ``History Metrics Preferences''
 \item Selezionare l'opzione ``Time based periods'' e scrivere all'interno della casella ``Period length'' il valore ``1000000000''
 \item Cliccare sul tasto ``Ok''
 \item Caricare il file ``Vuoto Versioning3.sql'' nella cartella ``sql'' del progetto
 \item Cliccare sul tasto ``Calculate history metrics''
\end{itemize}
Il processo si interrompe e nel terminale di Eclipse (launcher) viene mostrato il messaggio di errore ``net.sf.jeasyorm.RuntimeSQLException: Error inserting entity''
\newline
Il progetto di riferimento ha le seguenti caratteristiche:
\begin{itemize}
 \item Non ci sono file locali
 \item È presente un cambiamento (cancellazione di un file)
\end{itemize}

\paragraph{Note} Le informazioni presenti nel log dell'errore mostrano che questo potrebbe essere dovuto ad un problema con il salvataggio di una data.





\newpage






\subsection*{TC 10.10}
Passi da eseguire per arrivare alla failure:
\begin{itemize}
 \item Avviare Eclipse
 \item Pulire il database
 \item Aprire le impostazioni di Eclipse (Window/Preferences)
 \item Selezionare ``History Metrics Preferences''
 \item Selezionare l'opzione ``Burst based period''
 \item Cliccare sul tasto ``Ok''
 \item Caricare il file ``Vuoto Versioning3.sql'' nella cartella ``sql'' del progetto
 \item Cliccare sul tasto ``Calculate history metrics''
 \item Eseguire la query che mostra i risultati per le metriche di package
\end{itemize}
Non verrà mostrata una tupla in cui il campo ``nome'' ha valore ``ECC\_Model''
\vspace{0.5cm}
\newline
Il progetto di riferimento ha le seguenti caratteristiche:
\begin{itemize}
 \item Non ci sono file locali
 \item È presente un solo cambiamento (cancellazione di un file)
\end{itemize}





\newpage





\subsection*{TC 10.14}
Passi da eseguire per arrivare alla failure:
\begin{itemize}
 \item Avviare Eclipse
 \item Pulire il database
 \item Aprire le impostazioni di Eclipse (Window/Preferences)
 \item Selezionare ``History Metrics Preferences''
 \item Selezionare l'opzione ``Time based periods'' e scrivere all'interno della casella ``Period length'' il valore ``1000000000''
 \item Cliccare sul tasto ``Ok''
 \item Caricare il file ``Progetto9.sql'' nella cartella ``sql'' del progetto
 \item Cliccare sul tasto ``Calculate history metrics''
\end{itemize}
Il processo si interrompe e nel terminale di Eclipse (launcher) viene mostrato il messaggio di errore ``net.sf.jeasyorm.RuntimeSQLException: Error inserting entity''
\newline
Il progetto di riferimento ha le seguenti caratteristiche:
\begin{itemize}
 \item Sono presenti due file locali sotto versioning
 \item Sono presenti due cambiamenti
\end{itemize}

\paragraph{Note} Le informazioni presenti nel log dell'errore mostrano che questo potrebbe essere dovuto ad un problema con il salvataggio di una data.





\newpage






\subsection*{TC 10.20}
Passi da eseguire per arrivare alla failure:
\begin{itemize}
 \item Avviare Eclipse
 \item Pulire il database
 \item Aprire le impostazioni di Eclipse (Window/Preferences)
 \item Selezionare ``History Metrics Preferences''
 \item Selezionare l'opzione ``Burst based period''
 \item Cliccare sul tasto ``Ok''
 \item Caricare il file ``Progetto25.sql'' nella cartella ``sql'' del progetto
 \item Cliccare sul tasto ``Calculate history metrics''
 \item Eseguire la query che mostra i risultati per le metriche di package
\end{itemize}
Non verrà mostrata una tupla in cui il campo ``nome'' ha valore ``ECC\_Model''
\vspace{0.5cm}
\newline
Il progetto di riferimento ha le seguenti caratteristiche:
\begin{itemize}
 \item Ci sono 3 file locali sotto versioning
 \item Sono presenti due cambiamenti
\end{itemize}





\newpage





\subsection*{TC 10.22}
Passi da eseguire per arrivare alla failure:
\begin{itemize}
 \item Avviare Eclipse
 \item Pulire il database
 \item Aprire le impostazioni di Eclipse (Window/Preferences)
 \item Selezionare ``History Metrics Preferences''
 \item Selezionare l'opzione ``Time based period'' e scrivere all'interno della casella ``Period length'' il valore ``17''
 \item Cliccare sul tasto ``Ok''
 \item Caricare il file ``Progetto1.sql'' nella cartella ``sql'' del progetto
 \item Cliccare sul tasto ``Calculate history metrics''
\end{itemize}
Il processo si interrompe e nel terminale di Eclipse (launcher) viene mostrato il messaggio di errore ``net.sf.jeasyorm.RuntimeSQLException: Error inserting entity''
\newline
Il progetto di riferimento ha le seguenti caratteristiche:
\begin{itemize}
 \item È presente un file locale
 \item Non sono presenti cambiamenti
\end{itemize}

\paragraph{Note} Le informazioni presenti nel log dell'errore mostrano che questo potrebbe essere dovuto ad un problema con il salvataggio di una data.





\newpage





\subsection*{TC 10.25}
Passi da eseguire per arrivare alla failure:
\begin{itemize}
 \item Avviare Eclipse
 \item Pulire il database
 \item Aprire le impostazioni di Eclipse (Window/Preferences)
 \item Selezionare ``History Metrics Preferences''
 \item Selezionare l'opzione ``Time based periods'' e scrivere all'interno della casella ``Period length'' il valore ``1000000000''
 \item Cliccare sul tasto ``Ok''
 \item Caricare il file ``Progetto4.sql'' nella cartella ``sql'' del progetto
 \item Cliccare sul tasto ``Calculate history metrics''
\end{itemize}
Il processo si interrompe e nel terminale di Eclipse (launcher) viene mostrato il messaggio di errore ``net.sf.jeasyorm.RuntimeSQLException: Error inserting entity''
\newline
Il progetto di riferimento ha le seguenti caratteristiche:
\begin{itemize}
 \item È presente un file locale non sotto versioning
 \item Sono presenti due cambiamenti
\end{itemize}

\paragraph{Note} Le informazioni presenti nel log dell'errore mostrano che questo potrebbe essere dovuto ad un problema con il salvataggio di una data.





\newpage






\subsection*{TC 10.31}
Passi da eseguire per arrivare alla failure:
\begin{itemize}
 \item Avviare Eclipse
 \item Pulire il database
 \item Aprire le impostazioni di Eclipse (Window/Preferences)
 \item Selezionare ``History Metrics Preferences''
 \item Selezionare l'opzione ``Burst based period''
 \item Cliccare sul tasto ``Ok''
 \item Caricare il file ``Progetto4.sql'' nella cartella ``sql'' del progetto
 \item Cliccare sul tasto ``Calculate history metrics''
 \item Eseguire la query che mostra i risultati per le metriche di package
\end{itemize}
Non verrà mostrata una tupla in cui il campo ``nome'' ha valore ``ECC\_Model''
\vspace{0.5cm}
\newline
Il progetto di riferimento ha le seguenti caratteristiche:
\begin{itemize}
 \item È presente un file locale non sotto versioning
 \item Sono presenti due cambiamenti
\end{itemize}





\newpage





\subsection*{TC 10.36}
Passi da eseguire per arrivare alla failure:
\begin{itemize}
 \item Avviare Eclipse
 \item Pulire il database
 \item Aprire le impostazioni di Eclipse (Window/Preferences)
 \item Selezionare ``History Metrics Preferences''
 \item Selezionare l'opzione ``Time based periods'' e scrivere all'interno della casella ``Period length'' il valore ``1000000000''
 \item Cliccare sul tasto ``Ok''
 \item Caricare il file ``Progetto19.sql'' nella cartella ``sql'' del progetto
 \item Cliccare sul tasto ``Calculate history metrics''
\end{itemize}
Il processo si interrompe e nel terminale di Eclipse (launcher) viene mostrato il messaggio di errore ``net.sf.jeasyorm.RuntimeSQLException: Error inserting entity''
\newline
Il progetto di riferimento ha le seguenti caratteristiche:
\begin{itemize}
 \item Sono presenti due file sotto versioning
 \item Sono presenti due cambiamenti
\end{itemize}

\paragraph{Note} Le informazioni presenti nel log dell'errore mostrano che questo potrebbe essere dovuto ad un problema con il salvataggio di una data.





\newpage






\subsection*{TC 10.42}
Passi da eseguire per arrivare alla failure:
\begin{itemize}
 \item Avviare Eclipse
 \item Pulire il database
 \item Aprire le impostazioni di Eclipse (Window/Preferences)
 \item Selezionare ``History Metrics Preferences''
 \item Selezionare l'opzione ``Burst based period''
 \item Cliccare sul tasto ``Ok''
 \item Caricare il file ``Progetto19.sql'' nella cartella ``sql'' del progetto
 \item Cliccare sul tasto ``Calculate history metrics''
 \item Eseguire la query che mostra i risultati per le metriche di package
\end{itemize}
Non verrà mostrata una tupla in cui il campo ``nome'' ha valore ``ECC\_Model''
\vspace{0.5cm}
\newline
Il progetto di riferimento ha le seguenti caratteristiche:
\begin{itemize}
 \item Sono presenti due file locali sotto versioning
 \item Sono presenti due cambiamenti
\end{itemize}





\newpage





\subsection*{TC 10.44}
Passi da eseguire per arrivare alla failure:
\begin{itemize}
 \item Avviare Eclipse
 \item Pulire il database
 \item Aprire le impostazioni di Eclipse (Window/Preferences)
 \item Selezionare ``History Metrics Preferences''
 \item Selezionare l'opzione ``Time based period'' e scrivere all'interno della casella ``Period length'' il valore ``17''
 \item Cliccare sul tasto ``Ok''
 \item Caricare il file ``Progetto11.sql'' nella cartella ``sql'' del progetto
 \item Cliccare sul tasto ``Calculate history metrics''
\end{itemize}
Il processo si interrompe e nel terminale di Eclipse (launcher) viene mostrato il messaggio di errore ``net.sf.jeasyorm.RuntimeSQLException: Error inserting entity''
\newline
Il progetto di riferimento ha le seguenti caratteristiche:
\begin{itemize}
 \item È presente un file locale non sotto versioning
 \item Non sono presenti cambiamenti
\end{itemize}

\paragraph{Note} Le informazioni presenti nel log dell'errore mostrano che questo potrebbe essere dovuto ad un problema con il salvataggio di una data.





\newpage





\subsection*{TC 10.47}
Passi da eseguire per arrivare alla failure:
\begin{itemize}
 \item Avviare Eclipse
 \item Pulire il database
 \item Aprire le impostazioni di Eclipse (Window/Preferences)
 \item Selezionare ``History Metrics Preferences''
 \item Selezionare l'opzione ``Time based periods'' e scrivere all'interno della casella ``Period length'' il valore ``1000000000''
 \item Cliccare sul tasto ``Ok''
 \item Caricare il file ``Progetto14.sql'' nella cartella ``sql'' del progetto
 \item Cliccare sul tasto ``Calculate history metrics''
\end{itemize}
Il processo si interrompe e nel terminale di Eclipse (launcher) viene mostrato il messaggio di errore ``net.sf.jeasyorm.RuntimeSQLException: Error inserting entity''
\newline
Il progetto di riferimento ha le seguenti caratteristiche:
\begin{itemize}
 \item Sono presenti due file non sotto versioning
 \item Sono presenti due cambiamenti
\end{itemize}

\paragraph{Note} Le informazioni presenti nel log dell'errore mostrano che questo potrebbe essere dovuto ad un problema con il salvataggio di una data.





\newpage






\subsection*{TC 10.53}
Passi da eseguire per arrivare alla failure:
\begin{itemize}
 \item Avviare Eclipse
 \item Pulire il database
 \item Aprire le impostazioni di Eclipse (Window/Preferences)
 \item Selezionare ``History Metrics Preferences''
 \item Selezionare l'opzione ``Burst based period''
 \item Cliccare sul tasto ``Ok''
 \item Caricare il file ``Progetto14.sql'' nella cartella ``sql'' del progetto
 \item Cliccare sul tasto ``Calculate history metrics''
 \item Eseguire la query che mostra i risultati per le metriche di package
\end{itemize}
Non verrà mostrata una tupla in cui il campo ``nome'' ha valore ``ECC\_Model''
\vspace{0.5cm}
\newline
Il progetto di riferimento ha le seguenti caratteristiche:
\begin{itemize}
 \item Sono presenti due file non sotto versioning
 \item Sono presenti due cambiamenti
\end{itemize}





\newpage





