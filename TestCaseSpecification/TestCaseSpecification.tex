\chapter{Test Case}

\section{Metrica per il calcolo del numero di revisioni del sistema}

\begin{table}[ht]
\begin{tabular}{|p{3cm}|p{9cm}|}
\hline
\cellcolor{lightgray}Codice				& TC 1.1								\\
\hline
\cellcolor{lightgray}Combinazione		& cnum1									\\
\hline
\cellcolor{lightgray}Precondizione		& L'utente ha aperto Eclipse e ha installato il plugin.		\\
\hline
\cellcolor{lightgray}Comando			& L'utente seleziona il progetto ``Vuoto''  e clicca sull'opzione ``Calcola metrica per il numero medio di revisioni del sistema''.	\\
\hline
\cellcolor{lightgray}Risultato atteso	& Viene salvato nel database il valore 0 relativo a questa metrica.\\
\hline
\end{tabular}
\end{table}

\begin{table}[ht]
\begin{tabular}{|p{3cm}|p{9cm}|}
\hline
\cellcolor{lightgray}Codice				& TC 1.2								\\
\hline
\cellcolor{lightgray}Combinazione		& cnum2									\\
\hline
\cellcolor{lightgray}Precondizione		& L'utente ha aperto Eclipse e ha installato il plugin.		\\
\hline
\cellcolor{lightgray}Comando			& L'utente seleziona il progetto ``Vuoto Versioning 1''  e clicca sull'opzione ``Calcola metrica per il numero medio di revisioni del sistema''.	\\
\hline
\cellcolor{lightgray}Risultato atteso	& Viene salvato nel database il valore 1 relativo a questa metrica.\\
\hline
\end{tabular}
\end{table}
\clearpage

\section{Metrica per il calcolo del numero medio di volte in cui i file di un package hanno subito cambiamenti}

\begin{table}[ht]
\begin{tabular}{|p{3cm}|p{9cm}|}
\hline
\cellcolor{lightgray}Codice				& TC 2.1								\\
\hline
\cellcolor{lightgray}Combinazione		& fnum1cnum1								\\
\hline
\cellcolor{lightgray}Precondizione		& L'utente ha aperto Eclipse e ha installato il plugin.		\\
\hline
\cellcolor{lightgray}Comando			& L'utente seleziona il progetto ``Vuoto''  e clicca sull'opzione ``Calcola metrica per il numero medio di volte in cui i file di un package hanno subito cambiamenti''.	\\
\hline
\cellcolor{lightgray}Risultato atteso	& Viene salvato nel database il valore 0 relativo a questa metrica.\\
\hline
\end{tabular}
\end{table}

\begin{table}[ht]
\begin{tabular}{|p{3cm}|p{9cm}|}
\hline
\cellcolor{lightgray}Codice				& TC 2.2								\\
\hline
\cellcolor{lightgray}Combinazione		& fnum1cnum2								\\
\hline
\cellcolor{lightgray}Precondizione		& L'utente ha aperto Eclipse e ha installato il plugin.		\\
\hline
\cellcolor{lightgray}Comando			& L'utente seleziona il progetto ``Vuoto Versioning 1''  e clicca sull'opzione ``Calcola metrica per il numero medio di volte in cui i file di un package hanno subito cambiamenti''.	\\
\hline
\cellcolor{lightgray}Risultato atteso	& Viene salvato nel database il valore 1 relativo a questa metrica.\\
\hline
\end{tabular}
\end{table}

\begin{table}[ht]
\begin{tabular}{|p{3cm}|p{9cm}|}
\hline
\cellcolor{lightgray}Codice				& TC 2.3								\\
\hline
\cellcolor{lightgray}Combinazione		& fnum2ftype1fver1cnum1									\\
\hline
\cellcolor{lightgray}Precondizione		& L'utente ha aperto Eclipse e ha installato il plugin.		\\
\hline
\cellcolor{lightgray}Comando			& L'utente seleziona il progetto ``Progetto 6''  e clicca sull'opzione ``Calcola metrica per il numero medio di volte in cui i file di un package hanno subito cambiamenti''.	\\
\hline
\cellcolor{lightgray}Risultato atteso	& Viene salvato nel database il valore 0 relativo a questa metrica.\\
\hline
\end{tabular}
\end{table}

\begin{table}[ht]
\begin{tabular}{|p{3cm}|p{9cm}|}
\hline
\cellcolor{lightgray}Codice				& TC 2.4								\\
\hline
\cellcolor{lightgray}Combinazione		& fnum2ftype1fver2cnum1									\\
\hline
\cellcolor{lightgray}Precondizione		& L'utente ha aperto Eclipse e ha installato il plugin.		\\
\hline
\cellcolor{lightgray}Comando			& L'utente seleziona il progetto ``Progetto 1''  e clicca sull'opzione ``Calcola metrica per il numero medio di volte in cui i file di un package hanno subito cambiamenti''.	\\
\hline
\cellcolor{lightgray}Risultato atteso	& Viene salvato nel database il valore 0 relativo a questa metrica.\\
\hline
\end{tabular}
\end{table}

\begin{table}[ht]
\begin{tabular}{|p{3cm}|p{9cm}|}
\hline
\cellcolor{lightgray}Codice				& TC 2.5								\\
\hline
\cellcolor{lightgray}Combinazione		& fnum2ftype2fver1cnum1								\\
\hline
\cellcolor{lightgray}Precondizione		& L'utente ha aperto Eclipse e ha installato il plugin.		\\
\hline
\cellcolor{lightgray}Comando			& L'utente seleziona il progetto ``Progetto 16''  e clicca sull'opzione ``Calcola metrica per il numero medio di volte in cui i file di un package hanno subito cambiamenti''.	\\
\hline
\cellcolor{lightgray}Risultato atteso	& Viene salvato nel database il valore 0 relativo a questa metrica.\\
\hline
\end{tabular}
\end{table}

\begin{table}[ht]
\begin{tabular}{|p{3cm}|p{9cm}|}
\hline
\cellcolor{lightgray}Codice				& TC 2.6								\\
\hline
\cellcolor{lightgray}Combinazione		& fnum2ftype2fver2cnum1								\\
\hline
\cellcolor{lightgray}Precondizione		& L'utente ha aperto Eclipse e ha installato il plugin.		\\
\hline
\cellcolor{lightgray}Comando			& L'utente seleziona il progetto ``Progetto 11''  e clicca sull'opzione ``Calcola metrica per il numero medio di volte in cui i file di un package hanno subito cambiamenti''.	\\
\hline
\cellcolor{lightgray}Risultato atteso	& Viene salvato nel database il valore 0 relativo a questa metrica.\\
\hline
\end{tabular}
\end{table}

\begin{table}[ht]
\begin{tabular}{|p{3cm}|p{9cm}|}
\hline
\cellcolor{lightgray}Codice				& TC 2.7								\\
\hline
\cellcolor{lightgray}Combinazione		& fnum2ftype1fver1cnum2									\\
\hline
\cellcolor{lightgray}Precondizione		& L'utente ha aperto Eclipse e ha installato il plugin.		\\
\hline
\cellcolor{lightgray}Comando			& L'utente seleziona il progetto ``Progetto 7''  e clicca sull'opzione ``Calcola metrica per il numero medio di volte in cui i file di un package hanno subito cambiamenti''.	\\
\hline
\cellcolor{lightgray}Risultato atteso	& Viene salvato nel database il valore 1 relativo a questa metrica.\\
\hline
\end{tabular}
\end{table}

\begin{table}[ht]
\begin{tabular}{|p{3cm}|p{9cm}|}
\hline
\cellcolor{lightgray}Codice				& TC 2.8								\\
\hline
\cellcolor{lightgray}Combinazione		& fnum2ftype1fver2cnum2										\\
\hline
\cellcolor{lightgray}Precondizione		& L'utente ha aperto Eclipse e ha installato il plugin.		\\
\hline
\cellcolor{lightgray}Comando			& L'utente seleziona il progetto ``Progetto 2''  e clicca sull'opzione ``Calcola metrica per il numero medio di volte in cui i file di un package hanno subito cambiamenti''.	\\
\hline
\cellcolor{lightgray}Risultato atteso	& Viene salvato nel database il valore 0 relativo a questa metrica.\\
\hline
\end{tabular}
\end{table}

\begin{table}[ht]
\begin{tabular}{|p{3cm}|p{9cm}|}
\hline
\cellcolor{lightgray}Codice				& TC 2.9								\\
\hline
\cellcolor{lightgray}Combinazione		& fnum2ftype2fver1cnum2									\\
\hline
\cellcolor{lightgray}Precondizione		& L'utente ha aperto Eclipse e ha installato il plugin.		\\
\hline
\cellcolor{lightgray}Comando			& L'utente seleziona il progetto ``Progetto 17''  e clicca sull'opzione ``Calcola metrica per il numero medio di volte in cui i file di un package hanno subito cambiamenti''.	\\
\hline
\cellcolor{lightgray}Risultato atteso	& Viene salvato nel database il valore 1 relativo a questa metrica.\\
\hline
\end{tabular}
\end{table}

\begin{table}[ht]
\begin{tabular}{|p{3cm}|p{9cm}|}
\hline
\cellcolor{lightgray}Codice				& TC 2.10								\\
\hline
\cellcolor{lightgray}Combinazione		& fnum2ftype2fver2cnum2									\\
\hline
\cellcolor{lightgray}Precondizione		& L'utente ha aperto Eclipse e ha installato il plugin.		\\
\hline
\cellcolor{lightgray}Comando			& L'utente seleziona il progetto ``Progetto 12''  e clicca sull'opzione ``Calcola metrica per il numero medio di volte in cui i file di un package hanno subito cambiamenti''.	\\
\hline
\cellcolor{lightgray}Risultato atteso	& Viene salvato nel database il valore 0 relativo a questa metrica.\\
\hline
\end{tabular}
\end{table}
\clearpage

\section{Metrica per il calcolo del numero medio di refactoring di un package}
\begin{table}[ht]
\begin{tabular}{|p{3cm}|p{9cm}|}
\hline
\cellcolor{lightgray}Codice				& TC 3.1								\\
\hline
\cellcolor{lightgray}Combinazione		& fnum1cnum1									\\
\hline
\cellcolor{lightgray}Precondizione		& L'utente ha aperto Eclipse e ha installato il plugin.		\\
\hline
\cellcolor{lightgray}Comando			& L'utente seleziona il progetto ``Vuoto''  e clicca sull'opzione ``Calcola metrica per il numero medio di refactoring di un package''.	\\
\hline
\cellcolor{lightgray}Risultato atteso	& Viene salvato nel database il valore 0 relativo a questa metrica.\\
\hline
\end{tabular}
\end{table}

\begin{table}[ht]
\begin{tabular}{|p{3cm}|p{9cm}|}
\hline
\cellcolor{lightgray}Codice				& TC 3.2								\\
\hline
\cellcolor{lightgray}Combinazione		& fnum1cnum2cmes1 									\\
\hline
\cellcolor{lightgray}Precondizione		& L'utente ha aperto Eclipse e ha installato il plugin.				\\
\hline
\cellcolor{lightgray}Comando			& L'utente seleziona il progetto ``Vuoto Versioning 1''  e clicca sull'opzione ``Calcola metrica per il numero medio di refactoring di un package''.	\\
\hline
\cellcolor{lightgray}Risultato atteso	& Viene salvato nel database il valore 0 relativo a questa metrica	\\
\hline
\end{tabular}
\end{table}

\begin{table}[ht]
\begin{tabular}{|p{3cm}|p{9cm}|}
\hline
\cellcolor{lightgray}Codice				& TC 3.3								\\
\hline
\cellcolor{lightgray}Combinazione		& fnum1cnum2cmes2									\\
\hline
\cellcolor{lightgray}Precondizione		& L'utente ha aperto Eclipse e ha installato il plugin.					\\
\hline
\cellcolor{lightgray}Comando			& L'utente seleziona il progetto ``Refactoring 1''  e clicca sull'opzione ``Calcola metrica per il numero medio di refactoring di un package''.	\\
\hline
\cellcolor{lightgray}Risultato atteso	& Viene salvato nel database il valore 0 relativo a questa metrica	\\
\hline
\end{tabular}
\end{table}

\begin{table}[ht]
\begin{tabular}{|p{3cm}|p{9cm}|}
\hline
\cellcolor{lightgray}Codice				& TC 3.4								\\
\hline
\cellcolor{lightgray}Combinazione		& fnum1cnum2cmes3									\\
\hline
\cellcolor{lightgray}Precondizione		& L'utente ha aperto Eclipse e ha installato il plugin.			\\
\hline
\cellcolor{lightgray}Comando			& L'utente seleziona il progetto ``Bugfix 1''  e clicca sull'opzione ``Calcola metrica per il numero medio di refactoring di un package''.	\\
\hline
\cellcolor{lightgray}Risultato atteso	& Viene salvato nel database il valore 0 relativo a questa metrica	\\
\hline
\end{tabular}
\end{table}

\begin{table}[ht]
\begin{tabular}{|p{3cm}|p{9cm}|}
\hline
\cellcolor{lightgray}Codice				& TC 3.5								\\
\hline
\cellcolor{lightgray}Combinazione		& fnum2ftype1fver1cnum1 									\\
\hline
\cellcolor{lightgray}Precondizione		& L'utente ha aperto Eclipse e ha installato il plugin.			\\
\hline
\cellcolor{lightgray}Comando			& L'utente seleziona il progetto ``Progetto 6''  e clicca sull'opzione ``Calcola metrica per il numero medio di refactoring di un package''.	\\
\hline
\cellcolor{lightgray}Risultato atteso	& Viene salvato nel database il valore 0 relativo a questa metrica	\\
\hline
\end{tabular}
\end{table}

\begin{table}[ht]
\begin{tabular}{|p{3cm}|p{9cm}|}
\hline
\cellcolor{lightgray}Codice				& TC 3.6								\\
\hline
\cellcolor{lightgray}Combinazione		& fnum2ftype1fver2cnum1									\\
\hline
\cellcolor{lightgray}Precondizione		& L'utente ha aperto Eclipse e ha installato il plugin.				\\
\hline
\cellcolor{lightgray}Comando			& L'utente seleziona il progetto ``Progetto 1''  e clicca sull'opzione ``Calcola metrica per il numero medio di refactoring di un package''.	\\
\hline
\cellcolor{lightgray}Risultato atteso	& Viene salvato nel database il valore 0 relativo a questa metrica	\\
\hline
\end{tabular}
\end{table}

\begin{table}[ht]
\begin{tabular}{|p{3cm}|p{9cm}|}
\hline
\cellcolor{lightgray}Codice				& TC 3.7								\\
\hline
\cellcolor{lightgray}Combinazione		& fnum2ftype2fver1cnum1								\\
\hline
\cellcolor{lightgray}Precondizione		& L'utente ha aperto Eclipse e ha installato il plugin.									\\
\hline
\cellcolor{lightgray}Comando			& L'utente seleziona il progetto ``Progetto 16''  e clicca sull'opzione ``Calcola metrica per il numero medio di refactoring di un package''.	\\
\hline
\cellcolor{lightgray}Risultato atteso	& Viene salvato nel database il valore 0 relativo a questa metrica	\\
\hline
\end{tabular}
\end{table}

\begin{table}[ht]
\begin{tabular}{|p{3cm}|p{9cm}|}
\hline
\cellcolor{lightgray}Codice				& TC 3.8								\\
\hline
\cellcolor{lightgray}Combinazione		& fnum2ftype2fver2cnum1 									\\
\hline
\cellcolor{lightgray}Precondizione		& L'utente ha aperto Eclipse e ha installato il plugin.				\\
\hline
\cellcolor{lightgray}Comando			& L'utente seleziona il progetto ``Progetto 11''  e clicca sull'opzione ``Calcola metrica per il numero medio di refactoring di un package''.	\\
\hline
\cellcolor{lightgray}Risultato atteso	& Viene salvato nel database il valore 0 relativo a questa metrica	\\
\hline
\end{tabular}
\end{table}

\begin{table}[ht]
\begin{tabular}{|p{3cm}|p{9cm}|}
\hline
\cellcolor{lightgray}Codice				& TC 3.9								\\
\hline
\cellcolor{lightgray}Combinazione		& fnum2ftype1fver1cnum2cmes1 									\\
\hline
\cellcolor{lightgray}Precondizione		& L'utente ha aperto Eclipse e ha installato il plugin.								\\
\hline
\cellcolor{lightgray}Comando			& L'utente seleziona il progetto ``Progetto 21''  e clicca sull'opzione ``Calcola metrica per il numero medio di refactoring di un package''.	\\
\hline
\cellcolor{lightgray}Risultato atteso	& Viene salvato nel database il valore 0 relativo a questa metrica	\\
\hline
\end{tabular}
\end{table}

\begin{table}[ht]
\begin{tabular}{|p{3cm}|p{9cm}|}
\hline
\cellcolor{lightgray}Codice				& TC 3.10								\\
\hline
\cellcolor{lightgray}Combinazione		& fnum2ftype1fver2cnum2cmes1 									\\
\hline
\cellcolor{lightgray}Precondizione		& L'utente ha aperto Eclipse e ha installato il plugin.									\\
\hline
\cellcolor{lightgray}Comando			& L'utente seleziona il progetto ``Progetto 27''  e clicca sull'opzione ``Calcola metrica per il numero medio di refactoring di un package''.	\\
\hline
\cellcolor{lightgray}Risultato atteso	& Viene salvato nel database il valore 0 relativo a questa metrica	\\
\hline
\end{tabular}
\end{table}

\clearpage

\begin{table}[ht]
\begin{tabular}{|p{3cm}|p{9cm}|}
\hline
\cellcolor{lightgray}Codice				& TC 3.11								\\
\hline
\cellcolor{lightgray}Combinazione		& fnum2ftype2fver1cnum2cmes1 									\\
\hline
\cellcolor{lightgray}Precondizione		& L'utente ha aperto Eclipse e ha installato il plugin.		\\
\hline
\cellcolor{lightgray}Comando			& L'utente seleziona il progetto ``Progetto 29''  e clicca sull'opzione ``Calcola metrica per il numero medio di refactoring di un package''.	\\
\hline
\cellcolor{lightgray}Risultato atteso	& Viene salvato nel database il valore 0 relativo a questa metrica.\\
\hline
\end{tabular}
\end{table}

\begin{table}[ht]
\begin{tabular}{|p{3cm}|p{9cm}|}
\hline
\cellcolor{lightgray}Codice				& TC 3.12								\\
\hline
\cellcolor{lightgray}Combinazione		& fnum2ftype2fver2cnum2cmes1 									\\
\hline
\cellcolor{lightgray}Precondizione		& L'utente ha aperto Eclipse e ha installato il plugin.				\\
\hline 
\cellcolor{lightgray}Comando 			& L'utente seleziona il progetto ``Progetto 31''  e clicca sull'opzione ``Calcola metrica per il numero medio di refactoring di un package''.	\\
\hline
\cellcolor{lightgray}Risultato atteso	& Viene salvato nel database il valore 0 relativo a questa metrica	\\
\hline
\end{tabular}
\end{table}

\begin{table}[ht]
\begin{tabular}{|p{3cm}|p{9cm}|}
\hline
\cellcolor{lightgray}Codice				& TC 3.13								\\
\hline
\cellcolor{lightgray}Combinazione		& fnum2ftype1fver1cnum2cmes2 									\\
\hline
\cellcolor{lightgray}Precondizione		& L'utente ha aperto Eclipse e ha installato il plugin.					\\
\hline
\cellcolor{lightgray}Comando			& L'utente seleziona il progetto ``Refactoring 2''  e clicca sull'opzione ``Calcola metrica per il numero medio di refactoring di un package''.	\\
\hline
\cellcolor{lightgray}Risultato atteso	& Viene salvato nel database il valore 1 relativo a questa metrica	\\
\hline
\end{tabular}
\end{table}

\begin{table}[ht]
\begin{tabular}{|p{3cm}|p{9cm}|}
\hline
\cellcolor{lightgray}Codice				& TC 3.14								\\
\hline
\cellcolor{lightgray}Combinazione		& fnum2ftype1fver2cnum2cmes2									\\
\hline
\cellcolor{lightgray}Precondizione		& L'utente ha aperto Eclipse e ha installato il plugin.			\\
\hline
\cellcolor{lightgray}Comando			& L'utente seleziona il progetto ``Refactoring 3''  e clicca sull'opzione ``Calcola metrica per il numero medio di refactoring di un package''.	\\
\hline
\cellcolor{lightgray}Risultato atteso	& Viene salvato nel database il valore 0 relativo a questa metrica	\\
\hline
\end{tabular}
\end{table}

\begin{table}[ht]
\begin{tabular}{|p{3cm}|p{9cm}|}
\hline
\cellcolor{lightgray}Codice				& TC 3.15								\\
\hline
\cellcolor{lightgray}Combinazione		& fnum2ftype2fver1cnum2cmes2  									\\
\hline
\cellcolor{lightgray}Precondizione		& L'utente ha aperto Eclipse e ha installato il plugin.			\\
\hline
\cellcolor{lightgray}Comando			& L'utente seleziona il progetto ``Refactoring 4''  e clicca sull'opzione 
``Calcola metrica per il numero medio di refactoring di un package''.	\\
\hline
\cellcolor{lightgray}Risultato atteso	& Viene salvato nel database il valore 0 relativo a questa metrica	\\
\hline
\end{tabular}
\end{table}

\begin{table}[ht]
\begin{tabular}{|p{3cm}|p{9cm}|}
\hline
\cellcolor{lightgray}Codice				& TC 3.16								\\
\hline
\cellcolor{lightgray}Combinazione		& fnum2ftype2fver2cnum2cmes2 									\\
\hline
\cellcolor{lightgray}Precondizione		& L'utente ha aperto Eclipse e ha installato il plugin.				\\
\hline
\cellcolor{lightgray}Comando			& L'utente seleziona il progetto ``Refactoring 5''  e clicca sull'opzione ``Calcola metrica per il numero medio di refactoring di un package''.	\\
\hline
\cellcolor{lightgray}Risultato atteso	& Viene salvato nel database il valore 0 relativo a questa metrica	\\
\hline
\end{tabular}
\end{table}

\begin{table}[ht]
\begin{tabular}{|p{3cm}|p{9cm}|}
\hline
\cellcolor{lightgray}Codice				& TC 3.17								\\
\hline
\cellcolor{lightgray}Combinazione		& fnum2ftype1fver2cnum2cmes3 								\\
\hline
\cellcolor{lightgray}Precondizione		& L'utente ha aperto Eclipse e ha installato il plugin.									\\
\hline
\cellcolor{lightgray}Comando			& L'utente seleziona il progetto ``Bugfix 3''  e clicca sull'opzione ``Calcola metrica per il numero medio di refactoring di un package''.	\\
\hline
\cellcolor{lightgray}Risultato atteso	& Viene salvato nel database il valore 0 relativo a questa metrica	\\
\hline
\end{tabular}
\end{table}

\begin{table}[ht]
\begin{tabular}{|p{3cm}|p{9cm}|}
\hline
\cellcolor{lightgray}Codice				& TC 3.18								\\
\hline
\cellcolor{lightgray}Combinazione		& fnum2ftype2fver1cnum2cmes3  									\\
\hline
\cellcolor{lightgray}Precondizione		& L'utente ha aperto Eclipse e ha installato il plugin.				\\
\hline
\cellcolor{lightgray}Comando			& L'utente seleziona il progetto ``Bugfix 4''  e clicca sull'opzione ``Calcola metrica per il numero medio di refactoring di un package''.	\\
\hline
\cellcolor{lightgray}Risultato atteso	& Viene salvato nel database il valore 0 relativo a questa metrica	\\
\hline
\end{tabular}
\end{table}

\begin{table}[ht]
\begin{tabular}{|p{3cm}|p{9cm}|}
\hline
\cellcolor{lightgray}Codice				& TC 3.19								\\
\hline
\cellcolor{lightgray}Combinazione		& fnum2ftype2fver2cnum2cmes3 									\\
\hline
\cellcolor{lightgray}Precondizione		& L'utente ha aperto Eclipse e ha installato il plugin.								\\
\hline
\cellcolor{lightgray}Comando			& L'utente seleziona il progetto ``Bugfix 5''  e clicca sull'opzione ``Calcola metrica per il numero medio di refactoring di un package''.	\\
\hline
\cellcolor{lightgray}Risultato atteso	& Viene salvato nel database il valore 0 relativo a questa metrica	\\
\hline
\end{tabular}
\end{table}

\begin{table}[ht]
\begin{tabular}{|p{3cm}|p{9cm}|}
\hline
\cellcolor{lightgray}Codice				& TC 3.20								\\
\hline
\cellcolor{lightgray}Combinazione		& fnum2ftype1fver1cnum2cmes3 									\\
\hline
\cellcolor{lightgray}Precondizione		& L'utente ha aperto Eclipse e ha installato il plugin.									\\
\hline
\cellcolor{lightgray}Comando			& L'utente seleziona il progetto `Bugfix 2''  e clicca sull'opzione ``Calcola metrica per il numero medio di refactoring di un package''.	\\
\hline
\cellcolor{lightgray}Risultato atteso	& Viene salvato nel database il valore 0 relativo a questa metrica	\\
\hline
\end{tabular}
\end{table}

\clearpage

\section{Metrica per il calcolo del numero medio di bug-fix di un package}

\begin{table}[ht]
\begin{tabular}{|p{3cm}|p{9cm}|}
\hline
\cellcolor{lightgray}Codice				& TC 4.1								\\
\hline
\cellcolor{lightgray}Combinazione		& fnum1cnum1									\\
\hline
\cellcolor{lightgray}Precondizione		& L'utente ha aperto Eclipse e ha installato il plugin.		\\
\hline
\cellcolor{lightgray}Comando			& L'utente seleziona il progetto ``Vuoto''  e clicca sull'opzione ``Calcola metrica per il numero medio di bug-fix di un package''.	\\
\hline
\cellcolor{lightgray}Risultato atteso	& Viene salvato nel database il valore 0 relativo a questa metrica.\\
\hline
\end{tabular}
\end{table}

\begin{table}[ht]
\begin{tabular}{|p{3cm}|p{9cm}|}
\hline
\cellcolor{lightgray}Codice				& TC 4.2								\\
\hline
\cellcolor{lightgray}Combinazione		& fnum1cnum2cmes1 									\\
\hline
\cellcolor{lightgray}Precondizione		& L'utente ha aperto Eclipse e ha installato il plugin.				\\
\hline
\cellcolor{lightgray}Comando			& L'utente seleziona il progetto `Vuoto Versioning 1''  e clicca sull'opzione ``Calcola metrica per il numero medio di bug-fix di un package''.	\\
\hline
\cellcolor{lightgray}Risultato atteso	& Viene salvato nel database il valore 0 relativo a questa metrica	\\
\hline
\end{tabular}
\end{table}

\begin{table}[ht]
\begin{tabular}{|p{3cm}|p{9cm}|}
\hline
\cellcolor{lightgray}Codice				& TC 4.3								\\
\hline
\cellcolor{lightgray}Combinazione		& fnum1cnum2cmes2									\\
\hline
\cellcolor{lightgray}Precondizione		& L'utente ha aperto Eclipse e ha installato il plugin.					\\
\hline
\cellcolor{lightgray}Comando			& L'utente seleziona il progetto `Bugfix 1''  e clicca sull'opzione ``Calcola metrica per il numero medio di bug-fix di un package''.	\\
\hline
\cellcolor{lightgray}Risultato atteso	& Viene salvato nel database il valore 0 relativo a questa metrica	\\
\hline
\end{tabular}
\end{table}

\begin{table}[ht]
\begin{tabular}{|p{3cm}|p{9cm}|}
\hline
\cellcolor{lightgray}Codice				& TC 4.4								\\
\hline
\cellcolor{lightgray}Combinazione		& fnum1cnum2cmes3								\\
\hline
\cellcolor{lightgray}Precondizione		& L'utente ha aperto Eclipse e ha installato il plugin.			\\
\hline
\cellcolor{lightgray}Comando			& L'utente seleziona il progetto ``Refactoring 1''  e clicca sull'opzione ``Calcola metrica per il numero medio di bug-fix di un package''.	\\
\hline
\cellcolor{lightgray}Risultato atteso	& Viene salvato nel database il valore 0 relativo a questa metrica	\\
\hline
\end{tabular}
\end{table}

\begin{table}[ht]
\begin{tabular}{|p{3cm}|p{9cm}|}
\hline
\cellcolor{lightgray}Codice				& TC 4.5								\\
\hline
\cellcolor{lightgray}Combinazione		& fnum2ftype1fver1cnum1 									\\
\hline
\cellcolor{lightgray}Precondizione		& L'utente ha aperto Eclipse e ha installato il plugin.			\\
\hline
\cellcolor{lightgray}Comando			& L'utente seleziona il progetto ``Progetto 6''  e clicca sull'opzione ``Calcola metrica per il numero medio di bug-fix di un package''.	\\
\hline
\cellcolor{lightgray}Risultato atteso	& Viene salvato nel database il valore 0 relativo a questa metrica	\\
\hline
\end{tabular}
\end{table}

\begin{table}[ht]
\begin{tabular}{|p{3cm}|p{9cm}|}
\hline
\cellcolor{lightgray}Codice				& TC 4.6								\\
\hline
\cellcolor{lightgray}Combinazione		& fnum2ftype1fver2cnum1								\\
\hline
\cellcolor{lightgray}Precondizione		& L'utente ha aperto Eclipse e ha installato il plugin.				\\
\hline
\cellcolor{lightgray}Comando			& L'utente seleziona il progetto ``Progetto 1''  e clicca sull'opzione ``Calcola metrica per il numero medio di bug-fix di un package''.	\\
\hline
\cellcolor{lightgray}Risultato atteso	& Viene salvato nel database il valore 0 relativo a questa metrica	\\
\hline
\end{tabular}
\end{table}

\begin{table}[ht]
\begin{tabular}{|p{3cm}|p{9cm}|}
\hline
\cellcolor{lightgray}Codice				& TC 4.7								\\
\hline
\cellcolor{lightgray}Combinazione		& fnum2ftype2fver1cnum1							\\
\hline
\cellcolor{lightgray}Precondizione		& L'utente ha aperto Eclipse e ha installato il plugin.									\\
\hline
\cellcolor{lightgray}Comando			& L'utente seleziona il progetto ``Progetto 16''  e clicca sull'opzione ``Calcola metrica per il numero medio di bug-fix di un package''.	\\
\hline
\cellcolor{lightgray}Risultato atteso	& Viene salvato nel database il valore 0 relativo a questa metrica	\\
\hline
\end{tabular}
\end{table}

\begin{table}[ht]
\begin{tabular}{|p{3cm}|p{9cm}|}
\hline
\cellcolor{lightgray}Codice				& TC 4.8								\\
\hline
\cellcolor{lightgray}Combinazione		& fnum2ftype2fver2cnum1 									\\
\hline
\cellcolor{lightgray}Precondizione		& L'utente ha aperto Eclipse e ha installato il plugin.				\\
\hline
\cellcolor{lightgray}Comando			& L'utente seleziona il progetto ``Progetto 11''  e clicca sull'opzione ``Calcola metrica per il numero medio di bug-fix di un package''.	\\
\hline
\cellcolor{lightgray}Risultato atteso	& Viene salvato nel database il valore 0 relativo a questa metrica	\\
\hline
\end{tabular}
\end{table}

\begin{table}[ht]
\begin{tabular}{|p{3cm}|p{9cm}|}
\hline
\cellcolor{lightgray}Codice				& TC 4.9								\\
\hline
\cellcolor{lightgray}Combinazione		& fnum2ftype1fver1cnum2cmes1 									\\
\hline
\cellcolor{lightgray}Precondizione		& L'utente ha aperto Eclipse e ha installato il plugin.								\\
\hline
\cellcolor{lightgray}Comando			& L'utente seleziona il progetto ``Progetto 7''  e clicca sull'opzione ``Calcola metrica per il numero medio di bug-fix di un package''.	\\
\hline
\cellcolor{lightgray}Risultato atteso	& Viene salvato nel database il valore 0 relativo a questa metrica	\\
\hline
\end{tabular}
\end{table}

\begin{table}[ht]
\begin{tabular}{|p{3cm}|p{9cm}|}
\hline
\cellcolor{lightgray}Codice				& TC 4.10								\\
\hline
\cellcolor{lightgray}Combinazione		& fnum2ftype1fver1cnum2cmes2 									\\
\hline
\cellcolor{lightgray}Precondizione		& L'utente ha aperto Eclipse e ha installato il plugin.									\\
\hline
\cellcolor{lightgray}Comando			& L'utente seleziona il progetto ``Bugfix 2''  e clicca sull'opzione ``Calcola metrica per il numero medio di bug-fix di un package''.	\\
\hline
\cellcolor{lightgray}Risultato atteso	& Viene salvato nel database il valore 1 relativo a questa metrica	\\
\hline
\end{tabular}
\end{table}
\clearpage

\begin{table}[ht]
\begin{tabular}{|p{3cm}|p{9cm}|}
\hline
\cellcolor{lightgray}Codice				& TC 4.11								\\
\hline
\cellcolor{lightgray}Combinazione		& fnum2ftype1fver1cnum2cmes3									\\
\hline
\cellcolor{lightgray}Precondizione		& L'utente ha aperto Eclipse e ha installato il plugin.		\\
\hline
\cellcolor{lightgray}Comando			& L'utente seleziona il progetto ``Refactoring 2''  e clicca sull'opzione ``Calcola metrica per il numero medio di bug-fix di un package''.	\\
\hline
\cellcolor{lightgray}Risultato atteso	& Viene salvato nel database il valore 0 relativo a questa metrica.\\
\hline
\end{tabular}
\end{table}

\begin{table}[ht]
\begin{tabular}{|p{3cm}|p{9cm}|}
\hline
\cellcolor{lightgray}Codice				& TC 4.12								\\
\hline
\cellcolor{lightgray}Combinazione		& fnum2ftype1fver2cnum2cmes1 									\\
\hline
\cellcolor{lightgray}Precondizione		& L'utente ha aperto Eclipse e ha installato il plugin.				\\
\hline
\cellcolor{lightgray}Comando			& L'utente seleziona il progetto `Progetto 27''  e clicca sull'opzione ``Calcola metrica per il numero medio di bug-fix di un package''.	\\
\hline
\cellcolor{lightgray}Risultato atteso	& Viene salvato nel database il valore 0 relativo a questa metrica	\\
\hline
\end{tabular}
\end{table}

\begin{table}[ht]
\begin{tabular}{|p{3cm}|p{9cm}|}
\hline
\cellcolor{lightgray}Codice				& TC 4.13								\\
\hline
\cellcolor{lightgray}Combinazione		& fnum2ftype1fver2cnum2cmes2									\\
\hline
\cellcolor{lightgray}Precondizione		& L'utente ha aperto Eclipse e ha installato il plugin.					\\
\hline
\cellcolor{lightgray}Comando			& L'utente seleziona il progetto ``Bugfix 3''  e clicca sull'opzione ``Calcola metrica per il numero medio di bug-fix di un package''.	\\
\hline
\cellcolor{lightgray}Risultato atteso	& Viene salvato nel database il valore 0 relativo a questa metrica	\\
\hline
\end{tabular}
\end{table}

\begin{table}[ht]
\begin{tabular}{|p{3cm}|p{9cm}|}
\hline
\cellcolor{lightgray}Codice				& TC 4.14								\\
\hline
\cellcolor{lightgray}Combinazione		& fnum2ftype1fver2cnum2cmes3								\\
\hline
\cellcolor{lightgray}Precondizione		& L'utente ha aperto Eclipse e ha installato il plugin.			\\
\hline
\cellcolor{lightgray}Comando			& L'utente seleziona il progetto ``Refactoring 3''  e clicca sull'opzione ``Calcola metrica per il numero medio di bug-fix di un package''.	\\
\hline
\cellcolor{lightgray}Risultato atteso	& Viene salvato nel database il valore 0 relativo a questa metrica	\\
\hline
\end{tabular}
\end{table}

\begin{table}[ht]
\begin{tabular}{|p{3cm}|p{9cm}|}
\hline
\cellcolor{lightgray}Codice				& TC 4.15								\\
\hline
\cellcolor{lightgray}Combinazione		& fnum2ftype2fver1cnum2cmes1 									\\
\hline
\cellcolor{lightgray}Precondizione		& L'utente ha aperto Eclipse e ha installato il plugin.			\\
\hline
\cellcolor{lightgray}Comando			& L'utente seleziona il progetto ``Progetto 29''  e clicca sull'opzione ``Calcola metrica per il numero medio di bug-fix di un package''.	\\
\hline
\cellcolor{lightgray}Risultato atteso	& Viene salvato nel database il valore 0 relativo a questa metrica	\\
\hline
\end{tabular}
\end{table}

\begin{table}[ht]
\begin{tabular}{|p{3cm}|p{9cm}|}
\hline
\cellcolor{lightgray}Codice				& TC 4.16								\\
\hline
\cellcolor{lightgray}Combinazione		& fnum2ftype2fver1cnum2cmes2								\\
\hline
\cellcolor{lightgray}Precondizione		& L'utente ha aperto Eclipse e ha installato il plugin.				\\
\hline
\cellcolor{lightgray}Comando			& L'utente seleziona il progetto ``Bugfix 4''  e clicca sull'opzione ``Calcola metrica per il numero medio di bug-fix di un package''.	\\
\hline
\cellcolor{lightgray}Risultato atteso	& Viene salvato nel database il valore 1 relativo a questa metrica	\\
\hline
\end{tabular}
\end{table}

\begin{table}[ht]
\begin{tabular}{|p{3cm}|p{9cm}|}
\hline
\cellcolor{lightgray}Codice				& TC 4.17								\\
\hline
\cellcolor{lightgray}Combinazione		& fnum2ftype2fver1cnum2cmes3							\\
\hline
\cellcolor{lightgray}Precondizione		& L'utente ha aperto Eclipse e ha installato il plugin.									\\
\hline
\cellcolor{lightgray}Comando			& L'utente seleziona il progetto ``Refactoring 4''  e clicca sull'opzione ``Calcola metrica per il numero medio di bug-fix di un package''.	\\
\hline
\cellcolor{lightgray}Risultato atteso	& Viene salvato nel database il valore 0 relativo a questa metrica	\\
\hline
\end{tabular}
\end{table}

\begin{table}[ht]
\begin{tabular}{|p{3cm}|p{9cm}|}
\hline
\cellcolor{lightgray}Codice				& TC 4.18								\\
\hline
\cellcolor{lightgray}Combinazione		& fnum2ftype2fver2cnum2cmes1 									\\
\hline
\cellcolor{lightgray}Precondizione		& L'utente ha aperto Eclipse e ha installato il plugin.				\\
\hline
\cellcolor{lightgray}Comando			& L'utente seleziona il progetto ``Progetto 31''  e clicca sull'opzione ``Calcola metrica per il numero medio di bug-fix di un package''.	\\
\hline
\cellcolor{lightgray}Risultato atteso	& Viene salvato nel database il valore 0 relativo a questa metrica	\\
\hline
\end{tabular}
\end{table}

\begin{table}[ht]
\begin{tabular}{|p{3cm}|p{9cm}|}
\hline
\cellcolor{lightgray}Codice				& TC 4.19								\\
\hline
\cellcolor{lightgray}Combinazione		& fnum2ftype2fver2cnum2cmes2 									\\
\hline
\cellcolor{lightgray}Precondizione		& L'utente ha aperto Eclipse e ha installato il plugin.								\\
\hline
\cellcolor{lightgray}Comando			& L'utente seleziona il progetto ``Bugfix 5''  e clicca sull'opzione ``Calcola metrica per il numero medio di bug-fix di un package''.	\\
\hline
\cellcolor{lightgray}Risultato atteso	& Viene salvato nel database il valore 0 relativo a questa metrica	\\
\hline
\end{tabular}
\end{table}

\begin{table}[ht]
\begin{tabular}{|p{3cm}|p{9cm}|}
\hline
\cellcolor{lightgray}Codice				& TC 4.20								\\
\hline
\cellcolor{lightgray}Combinazione		& fnum2ftype2fver2cnum2cmes3 									\\
\hline
\cellcolor{lightgray}Precondizione		& L'utente ha aperto Eclipse e ha installato il plugin.									\\
\hline
\cellcolor{lightgray}Comando			& L'utente seleziona il progetto ``Refactoring 5''  e clicca sull'opzione ``Calcola metrica per il numero medio di bug-fix di un package''.	\\
\hline
\cellcolor{lightgray}Risultato atteso	& Viene salvato nel database il valore 0 relativo a questa metrica	\\
\hline
\end{tabular}
\end{table}
\clearpage

\section{Metrica per il calcolo del numero di autori di commit di un package}

\begin{table}[ht]
\begin{tabular}{|p{3cm}|p{9cm}|}
\hline
\cellcolor{lightgray}Codice				& TC 5.1								\\
\hline
\cellcolor{lightgray}Combinazione		& fnum1cnum1									\\
\hline
\cellcolor{lightgray}Precondizione		& L'utente ha aperto Eclipse e ha installato il plugin.		\\
\hline
\cellcolor{lightgray}Comando			& L'utente seleziona il progetto ``Vuoto''  e clicca sull'opzione ``Calcola metrica per il numero di autori di commit di un package''.	\\
\hline
\cellcolor{lightgray}Risultato atteso	& Viene salvato nel database il valore 0 relativo a questa metrica.\\
\hline
\end{tabular}
\end{table}

\begin{table}[ht]
\begin{tabular}{|p{3cm}|p{9cm}|}
\hline
\cellcolor{lightgray}Codice				& TC 5.2								\\
\hline
\cellcolor{lightgray}Combinazione		& fnum1cnum2cnaut1ceaut1 									\\
\hline
\cellcolor{lightgray}Precondizione		& L'utente ha aperto Eclipse e ha installato il plugin.				\\
\hline
\cellcolor{lightgray}Comando			& L'utente seleziona il progetto ``Vuoto Versioning 3''  e clicca sull'opzione ``Calcola metrica per il numero di autori di commit di un package''.	\\
\hline
\cellcolor{lightgray}Risultato atteso	& Viene salvato nel database il valore 0 relativo a questa metrica	\\
\hline
\end{tabular}
\end{table}

\begin{table}[ht]
\begin{tabular}{|p{3cm}|p{9cm}|}
\hline
\cellcolor{lightgray}Codice				& TC 5.3								\\
\hline
\cellcolor{lightgray}Combinazione		& fnum2ftype1cnaut2ceaut1								\\
\hline
\cellcolor{lightgray}Precondizione		& L'utente ha aperto Eclipse e ha installato il plugin.					\\
\hline
\cellcolor{lightgray}Comando			& L'utente seleziona il progetto ``Vuoto Versioning 5''  e clicca sull'opzione ``Calcola metrica per il numero di autori di commit di un package''.	\\
\hline
\cellcolor{lightgray}Risultato atteso	& Viene salvato nel database il valore 0 relativo a questa metrica	\\
\hline
\end{tabular}
\end{table}

\begin{table}[ht]
\begin{tabular}{|p{3cm}|p{9cm}|}
\hline
\cellcolor{lightgray}Codice				& TC 5.4								\\
\hline
\cellcolor{lightgray}Combinazione		& fnum1cnum2cnaut1ceaut2							\\
\hline
\cellcolor{lightgray}Precondizione		& L'utente ha aperto Eclipse e ha installato il plugin.			\\
\hline
\cellcolor{lightgray}Comando			& L'utente seleziona il progetto ``Vuoto Versioning 4''  e clicca sull'opzione ``Calcola metrica per il numero di autori di commit di un package''.	\\
\hline
\cellcolor{lightgray}Risultato atteso	& Viene salvato nel database il valore 0 relativo a questa metrica	\\
\hline
\end{tabular}
\end{table}

\begin{table}[ht]
\begin{tabular}{|p{3cm}|p{9cm}|}
\hline
\cellcolor{lightgray}Codice				& TC 5.5								\\
\hline
\cellcolor{lightgray}Combinazione		& fnum1cnum2cnaut2ceaut2 									\\
\hline
\cellcolor{lightgray}Precondizione		& L'utente ha aperto Eclipse e ha installato il plugin.			\\
\hline
\cellcolor{lightgray}Comando			& L'utente seleziona il progetto ``Vuoto Versioning 6''  e clicca sull'opzione ``Calcola metrica per il numero di autori di commit di un package''.	\\
\hline
\cellcolor{lightgray}Risultato atteso	& Viene salvato nel database il valore 0 relativo a questa metrica	\\
\hline
\end{tabular}
\end{table}

\begin{table}[ht]
\begin{tabular}{|p{3cm}|p{9cm}|}
\hline
\cellcolor{lightgray}Codice				& TC 5.6								\\
\hline
\cellcolor{lightgray}Combinazione		& fnum2ftype1fver1cnum1								\\
\hline
\cellcolor{lightgray}Precondizione		& L'utente ha aperto Eclipse e ha installato il plugin.				\\
\hline
\cellcolor{lightgray}Comando			& L'utente seleziona il progetto ``Progetto 6''  e clicca sull'opzione ``Calcola metrica per il numero di autori di commit di un package''.	\\
\hline
\cellcolor{lightgray}Risultato atteso	& Viene salvato nel database il valore 0 relativo a questa metrica	\\
\hline
\end{tabular}
\end{table}

\begin{table}[ht]
\begin{tabular}{|p{3cm}|p{9cm}|}
\hline
\cellcolor{lightgray}Codice				& TC 5.7								\\
\hline
\cellcolor{lightgray}Combinazione		& fnum2ftype1fver1cnum2cnaut1ceaut1							\\
\hline
\cellcolor{lightgray}Precondizione		& L'utente ha aperto Eclipse e ha installato il plugin.									\\
\hline
\cellcolor{lightgray}Comando			& L'utente seleziona il progetto ``Progetto 7''  e clicca sull'opzione ``Calcola metrica per il numero di autori di commit di un package''.	\\
\hline
\cellcolor{lightgray}Risultato atteso	& Viene salvato nel database il valore 1 relativo a questa metrica	\\
\hline
\end{tabular}
\end{table}

\begin{table}[ht]
\begin{tabular}{|p{3cm}|p{9cm}|}
\hline
\cellcolor{lightgray}Codice				& TC 5.8								\\
\hline
\cellcolor{lightgray}Combinazione		& fnum2ftype1fver1cnum2cnaut1ceaut2 									\\
\hline
\cellcolor{lightgray}Precondizione		& L'utente ha aperto Eclipse e ha installato il plugin.				\\
\hline
\cellcolor{lightgray}Comando			& L'utente seleziona il progetto ``Progetto 8''  e clicca sull'opzione ``Calcola metrica per il numero di autori di commit di un package''.	\\
\hline
\cellcolor{lightgray}Risultato atteso	& Viene salvato nel database il valore 1 relativo a questa metrica	\\
\hline
\end{tabular}
\end{table}

\begin{table}[ht]
\begin{tabular}{|p{3cm}|p{9cm}|}
\hline
\cellcolor{lightgray}Codice				& TC 5.9								\\
\hline
\cellcolor{lightgray}Combinazione		& fnum2ftype1fver1cnum2cnaut2ceaut1 									\\
\hline
\cellcolor{lightgray}Precondizione		& L'utente ha aperto Eclipse e ha installato il plugin.								\\
\hline
\cellcolor{lightgray}Comando			& L'utente seleziona il progetto ``Progetto 9''  e clicca sull'opzione ``Calcola metrica per il numero di autori di commit di un package''.	\\
\hline
\cellcolor{lightgray}Risultato atteso	& Viene salvato nel database il valore 2 relativo a questa metrica	\\
\hline
\end{tabular}
\end{table}

\begin{table}[ht]
\begin{tabular}{|p{3cm}|p{9cm}|}
\hline
\cellcolor{lightgray}Codice				& TC 5.10								\\
\hline
\cellcolor{lightgray}Combinazione		& fnum2ftype1fver1cnum2cnaut2ceaut2 									\\
\hline
\cellcolor{lightgray}Precondizione		& L'utente ha aperto Eclipse e ha installato il plugin.									\\
\hline
\cellcolor{lightgray}Comando			& L'utente seleziona il progetto ``Progetto 10''  e clicca sull'opzione ``Calcola metrica per il numero di autori di commit di un package''.	\\
\hline
\cellcolor{lightgray}Risultato atteso	& Viene salvato nel database il valore 2 relativo a questa metrica	\\
\hline
\end{tabular}
\end{table}

\begin{table}[ht]
\begin{tabular}{|p{3cm}|p{9cm}|}
\hline
\cellcolor{lightgray}Codice				& TC 5.11								\\
\hline
\cellcolor{lightgray}Combinazione		& fnum2ftype1fver2cnum1									\\
\hline
\cellcolor{lightgray}Precondizione		& L'utente ha aperto Eclipse e ha installato il plugin.		\\
\hline
\cellcolor{lightgray}Comando			& L'utente seleziona il progetto ``Progetto 1''  e clicca sull'opzione ``Calcola metrica per il numero di autori di commit di un package''.	\\
\hline
\cellcolor{lightgray}Risultato atteso	& Viene salvato nel database il valore 0 relativo a questa metrica.\\
\hline
\end{tabular}
\end{table}

\begin{table}[ht]
\begin{tabular}{|p{3cm}|p{9cm}|}
\hline
\cellcolor{lightgray}Codice				& TC 5.12								\\
\hline
\cellcolor{lightgray}Combinazione		& fnum2ftype1fver2cnum2cnaut1ceaut1 									\\
\hline
\cellcolor{lightgray}Precondizione		& L'utente ha aperto Eclipse e ha installato il plugin.				\\
\hline
\cellcolor{lightgray}Comando			& L'utente seleziona il progetto ``Progetto 2''  e clicca sull'opzione ``Calcola metrica per il numero di autori di commit di un package''.	\\
\hline
\cellcolor{lightgray}Risultato atteso	& Viene salvato nel database il valore 0 relativo a questa metrica	\\
\hline
\end{tabular}
\end{table}

\begin{table}[ht]
\begin{tabular}{|p{3cm}|p{9cm}|}
\hline
\cellcolor{lightgray}Codice				& TC 5.13								\\
\hline
\cellcolor{lightgray}Combinazione		& fnum2ftype1fver2cnum2cnaut1ceaut2									\\
\hline
\cellcolor{lightgray}Precondizione		& L'utente ha aperto Eclipse e ha installato il plugin.					\\
\hline
\cellcolor{lightgray}Comando			& L'utente seleziona il progetto ``Progetto 3''  e clicca sull'opzione ``Calcola metrica per il numero di autori di commit di un package''.	\\
\hline
\cellcolor{lightgray}Risultato atteso	& Viene salvato nel database il valore 0 relativo a questa metrica	\\
\hline
\end{tabular}
\end{table}

\begin{table}[ht]
\begin{tabular}{|p{3cm}|p{9cm}|}
\hline
\cellcolor{lightgray}Codice				& TC 5.14								\\
\hline
\cellcolor{lightgray}Combinazione		& fnum2ftype1fver2cnum2cnaut2ceaut1								\\
\hline
\cellcolor{lightgray}Precondizione		& L'utente ha aperto Eclipse e ha installato il plugin.			\\
\hline
\cellcolor{lightgray}Comando			& L'utente seleziona il progetto ``Progetto 4''  e clicca sull'opzione ``Calcola metrica per il numero di autori di commit di un package''.	\\
\hline
\cellcolor{lightgray}Risultato atteso	& Viene salvato nel database il valore 0 relativo a questa metrica	\\
\hline
\end{tabular}
\end{table}

\begin{table}[ht]
\begin{tabular}{|p{3cm}|p{9cm}|}
\hline
\cellcolor{lightgray}Codice				& TC 5.15								\\
\hline
\cellcolor{lightgray}Combinazione		& fnum2ftype1fver2cnum2cnaut2ceaut2 									\\
\hline
\cellcolor{lightgray}Precondizione		& L'utente ha aperto Eclipse e ha installato il plugin.			\\
\hline
\cellcolor{lightgray}Comando			& L'utente seleziona il progetto ``Progetto 5''  e clicca sull'opzione ``Calcola metrica per il numero di autori di commit di un package''.	\\
\hline
\cellcolor{lightgray}Risultato atteso	& Viene salvato nel database il valore 0 relativo a questa metrica	\\
\hline
\end{tabular}
\end{table}



\begin{table}[ht]
\begin{tabular}{|p{3cm}|p{9cm}|}
\hline
\cellcolor{lightgray}Codice				& TC 5.16								\\
\hline
\cellcolor{lightgray}Combinazione		& fnum2ftype2fver1cnum1								\\
\hline
\cellcolor{lightgray}Precondizione		& L'utente ha aperto Eclipse e ha installato il plugin.				\\
\hline
\cellcolor{lightgray}Comando			& L'utente seleziona il progetto ``Progetto 16''  e clicca sull'opzione ``Calcola metrica per il numero di autori di commit di un package''.	\\
\hline
\cellcolor{lightgray}Risultato atteso	& Viene salvato nel database il valore 0 relativo a questa metrica	\\
\hline
\end{tabular}
\end{table}

\clearpage

\begin{table}[ht]
\begin{tabular}{|p{3cm}|p{9cm}|}
\hline
\cellcolor{lightgray}Codice				& TC 5.17								\\
\hline
\cellcolor{lightgray}Combinazione		& fnum2ftype2fver1cnum2cnaut1ceaut1							\\
\hline
\cellcolor{lightgray}Precondizione		& L'utente ha aperto Eclipse e ha installato il plugin.									\\
\hline
\cellcolor{lightgray}Comando			& L'utente seleziona il progetto ``Progetto 17''  e clicca sull'opzione ``Calcola metrica per il numero di autori di commit di un package''.	\\
\hline
\cellcolor{lightgray}Risultato atteso	& Viene salvato nel database il valore 1 relativo a questa metrica	\\
\hline
\end{tabular}
\end{table}

\begin{table}[ht]
\begin{tabular}{|p{3cm}|p{9cm}|}
\hline
\cellcolor{lightgray}Codice				& TC 5.18								\\
\hline
\cellcolor{lightgray}Combinazione		& fnum2ftype2fver1cnum2cnaut1ceaut2 									\\
\hline
\cellcolor{lightgray}Precondizione		& L'utente ha aperto Eclipse e ha installato il plugin.				\\
\hline
\cellcolor{lightgray}Comando			& L'utente seleziona il progetto ``Progetto 18''  e clicca sull'opzione ``Calcola metrica per il numero di autori di commit di un package''.	\\
\hline
\cellcolor{lightgray}Risultato atteso	& Viene salvato nel database il valore 1 relativo a questa metrica	\\
\hline
\end{tabular}
\end{table}

\begin{table}[ht]
\begin{tabular}{|p{3cm}|p{9cm}|}
\hline
\cellcolor{lightgray}Codice				& TC 5.19								\\
\hline
\cellcolor{lightgray}Combinazione		& fnum2ftype2fver1cnum2cnaut2ceaut1 									\\
\hline
\cellcolor{lightgray}Precondizione		& L'utente ha aperto Eclipse e ha installato il plugin.								\\
\hline
\cellcolor{lightgray}Comando			& L'utente seleziona il progetto ``Progetto 19''  e clicca sull'opzione L'utente seleziona il progetto ``Progetto ''  e clicca sull'opzione ``Calcola metrica per il numero di autori di commit di un package''.	\\
\hline
\cellcolor{lightgray}Risultato atteso	& Viene salvato nel database il valore 2 relativo a questa metrica	\\
\hline
\end{tabular}
\end{table}

\begin{table}[ht]
\begin{tabular}{|p{3cm}|p{9cm}|}
\hline
\cellcolor{lightgray}Codice				& TC 5.20								\\
\hline
\cellcolor{lightgray}Combinazione		& fnum2ftype2fver1cnum2cnaut2ceaut2 									\\
\hline
\cellcolor{lightgray}Precondizione		& L'utente ha aperto Eclipse e ha installato il plugin.									\\
\hline
\cellcolor{lightgray}Comando			& L'utente seleziona il progetto ``Progetto 20''  e clicca sull'opzione ``Calcola metrica per il numero di autori di commit di un package''.	\\
\hline
\cellcolor{lightgray}Risultato atteso	& Viene salvato nel database il valore 2 relativo a questa metrica	\\
\hline
\end{tabular}
\end{table}

\begin{table}[ht]
\begin{tabular}{|p{3cm}|p{9cm}|}
\hline
\cellcolor{lightgray}Codice				& TC 5.21								\\
\hline
\cellcolor{lightgray}Combinazione		& fnum2ftype2fver2cnum1									\\
\hline
\cellcolor{lightgray}Precondizione		& L'utente ha aperto Eclipse e ha installato il plugin.		\\
\hline
\cellcolor{lightgray}Comando			& L'utente seleziona il progetto ``Progetto 11''  e clicca sull'opzione ``Calcola metrica per il numero di autori di commit di un package''.	\\
\hline
\cellcolor{lightgray}Risultato atteso	& Viene salvato nel database il valore 0 relativo a questa metrica.\\
\hline
\end{tabular}
\end{table}

\begin{table}[ht]
\begin{tabular}{|p{3cm}|p{9cm}|}
\hline
\cellcolor{lightgray}Codice				& TC 5.22								\\
\hline
\cellcolor{lightgray}Combinazione		& fnum2ftype2fver2cnum2cnaut1ceaut1 									\\
\hline
\cellcolor{lightgray}Precondizione		& L'utente ha aperto Eclipse e ha installato il plugin.				\\
\hline
\cellcolor{lightgray}Comando			& L'utente seleziona il progetto ``Progetto 12''  e clicca sull'opzione ``Calcola metrica per il numero di autori di commit di un package''.	\\
\hline
\cellcolor{lightgray}Risultato atteso	& Viene salvato nel database il valore 0 relativo a questa metrica	\\
\hline
\end{tabular}
\end{table}

\begin{table}[ht]
\begin{tabular}{|p{3cm}|p{9cm}|}
\hline
\cellcolor{lightgray}Codice				& TC 5.23								\\
\hline
\cellcolor{lightgray}Combinazione		& fnum2ftype2fver2cnum2cnaut1ceaut2									\\
\hline
\cellcolor{lightgray}Precondizione		& L'utente ha aperto Eclipse e ha installato il plugin.					\\
\hline
\cellcolor{lightgray}Comando			& L'utente seleziona il progetto ``Progetto 13''  e clicca sull'opzione ``Calcola metrica per il numero di autori di commit di un package''.	\\
\hline
\cellcolor{lightgray}Risultato atteso	& Viene salvato nel database il valore 0 relativo a questa metrica	\\
\hline
\end{tabular}
\end{table}

\begin{table}[ht]
\begin{tabular}{|p{3cm}|p{9cm}|}
\hline
\cellcolor{lightgray}Codice				& TC 5.24								\\
\hline
\cellcolor{lightgray}Combinazione		& fnum2ftype2fver2cnum2cnaut2ceaut1								\\
\hline
\cellcolor{lightgray}Precondizione		& L'utente ha aperto Eclipse e ha installato il plugin.			\\
\hline
\cellcolor{lightgray}Comando			& L'utente seleziona il progetto ``Progetto 14''  e clicca sull'opzione ``Calcola metrica per il numero di autori di commit di un package''.	\\
\hline
\cellcolor{lightgray}Risultato atteso	& Viene salvato nel database il valore 0 relativo a questa metrica	\\
\hline
\end{tabular}
\end{table}

\begin{table}[ht]
\begin{tabular}{|p{3cm}|p{9cm}|}
\hline
\cellcolor{lightgray}Codice				& TC 5.25								\\
\hline
\cellcolor{lightgray}Combinazione		& fnum2ftype2fver2cnum2cnaut2ceaut2 									\\
\hline
\cellcolor{lightgray}Precondizione		& L'utente ha aperto Eclipse e ha installato il plugin.			\\
\hline
\cellcolor{lightgray}Comando			& L'utente seleziona il progetto ``Progetto 15''  e clicca sull'opzione ``Calcola metrica per il numero di autori di commit di un package''.	\\
\hline
\cellcolor{lightgray}Risultato atteso	& Viene salvato nel database il valore 0 relativo a questa metrica	\\
\hline
\end{tabular}
\end{table}

\clearpage

\section{Metrica per il calcolo del numero di righe aggiunte e rimosse}
Nei casi di test che seguono i valori indicati nell'oracolo si riferiscono nell'ordine a somma, media, massimo.\\

\begin{table}[ht]
\begin{tabular}{|p{3cm}|p{9cm}|}
\hline
\cellcolor{lightgray}Codice				& TC 6.1								\\
\hline
\cellcolor{lightgray}Combinazione		& fnum1cnum1ccan1cins1									\\
\hline
\cellcolor{lightgray}Precondizione		& L'utente ha aperto Eclipse e ha installato il plugin.		\\
\hline
\cellcolor{lightgray}Comando			& L'utente seleziona il progetto ``Progetto ?''  e clicca sull'opzione ``Calcola metrica per il numero di righe aggiunte e rimosse''.	\\
\hline
\cellcolor{lightgray}Risultato atteso	& Vengono salvati nel database i valori 0, 0, 0 relativi a questa metrica.\\
\hline
\end{tabular}
\end{table}

\begin{table}[ht]
\begin{tabular}{|p{3cm}|p{9cm}|}
\hline
\cellcolor{lightgray}Codice				& TC 6.2								\\
\hline
\cellcolor{lightgray}Combinazione		& fnum1cnum2ccan2cins1								\\
\hline
\cellcolor{lightgray}Precondizione		& L'utente ha aperto Eclipse e ha installato il plugin.					\\
\hline
\cellcolor{lightgray}Comando			& L'utente seleziona il progetto ``Vuoto Versioning 1''  e clicca sull'opzione ``Calcola metrica per il numero di righe aggiunte e rimosse''.	\\
\hline
\cellcolor{lightgray}Risultato atteso	& Vengono salvati nel database i valori 0, 0, 0 relativi a questa metrica.\\
\hline
\end{tabular}
\end{table}

\begin{table}[ht]
\begin{tabular}{|p{3cm}|p{9cm}|}
\hline
\cellcolor{lightgray}Codice				& TC 6.3								\\
\hline
\cellcolor{lightgray}Combinazione		& fnum2ftype1fver1cnum1ccan1cins1 									\\
\hline
\cellcolor{lightgray}Precondizione		& L'utente ha aperto Eclipse e ha installato il plugin.			\\
\hline
\cellcolor{lightgray}Comando			& L'utente seleziona il progetto ``Progetto 6''  e clicca sull'opzione ``Calcola metrica per il numero di righe aggiunte e rimosse''.	\\
\hline
\cellcolor{lightgray}Risultato atteso	& Vengono salvati nel database i valori 0, 0, 0 relativi a questa metrica.\\
\hline
\end{tabular}
\end{table}

\begin{table}[ht]
\begin{tabular}{|p{3cm}|p{9cm}|}
\hline
\cellcolor{lightgray}Codice				& TC 6.4								\\
\hline
\cellcolor{lightgray}Combinazione		& fnum2ftype1fver1cnum2ccan1cins2								\\
\hline
\cellcolor{lightgray}Precondizione		& L'utente ha aperto Eclipse e ha installato il plugin.				\\
\hline
\cellcolor{lightgray}Comando			& L'utente seleziona il progetto ``Progetto 21''  e clicca sull'opzione ``Calcola metrica per il numero di righe aggiunte e rimosse''.	\\
\hline
\cellcolor{lightgray}Risultato atteso	& Vengono salvati nel database i valori 12, 12, 12 relativi a questa metrica.\\
\hline
\end{tabular}
\end{table}

\begin{table}[ht]
\begin{tabular}{|p{3cm}|p{9cm}|}
\hline
\cellcolor{lightgray}Codice				& TC 6.5								\\
\hline
\cellcolor{lightgray}Combinazione		& fnum2ftype1fver1cnum2ccan2cins1							\\
\hline
\cellcolor{lightgray}Precondizione		& L'utente ha aperto Eclipse e ha installato il plugin.									\\
\hline
\cellcolor{lightgray}Comando			& L'utente seleziona il progetto ``Progetto 22''  e clicca sull'opzione ``Calcola metrica per il numero di righe aggiunte e rimosse''.	\\
\hline
\cellcolor{lightgray}Risultato atteso	& Vengono salvati nel database i valori 12, 12, 12 relativi a questa metrica.\\
\hline
\end{tabular}
\end{table}

\begin{table}[ht]
\begin{tabular}{|p{3cm}|p{9cm}|}
\hline
\cellcolor{lightgray}Codice				& TC 6.6								\\
\hline
\cellcolor{lightgray}Combinazione		& fnum2ftype1fver1cnum2ccan2cins2 									\\
\hline
\cellcolor{lightgray}Precondizione		& L'utente ha aperto Eclipse e ha installato il plugin.				\\
\hline
\cellcolor{lightgray}Comando			& L'utente seleziona il progetto ``Progetto 23''  e clicca sull'opzione ``Calcola metrica per il numero di righe aggiunte e rimosse''.	\\
\hline
\cellcolor{lightgray}Risultato atteso	& Vengono salvati nel database i valori 25, 25, 25 relativi a questa metrica.\\
\hline
\end{tabular}
\end{table}

\begin{table}[ht]
\begin{tabular}{|p{3cm}|p{9cm}|}
\hline
\cellcolor{lightgray}Codice				& TC 6.7								\\
\hline
\cellcolor{lightgray}Combinazione		& fnum2ftype1fver2cnum1ccan1cins1 									\\
\hline
\cellcolor{lightgray}Precondizione		& L'utente ha aperto Eclipse e ha installato il plugin.								\\
\hline
\cellcolor{lightgray}Comando			& L'utente seleziona il progetto ``Progetto 1''  e clicca sull'opzione ``Calcola metrica per il numero di righe aggiunte e rimosse''.	\\
\hline
\cellcolor{lightgray}Risultato atteso	& Vengono salvati nel database i valori 0, 0, 0 relativi a questa metrica.\\
\hline
\end{tabular}
\end{table}

\begin{table}[ht]
\begin{tabular}{|p{3cm}|p{9cm}|}
\hline
\cellcolor{lightgray}Codice				& TC 6.8								\\
\hline
\cellcolor{lightgray}Combinazione		& fnum2ftype1fver2cnum2ccan2cins1 									\\
\hline
\cellcolor{lightgray}Precondizione		& L'utente ha aperto Eclipse e ha installato il plugin.									\\
\hline
\cellcolor{lightgray}Comando			& L'utente seleziona il progetto ``Progetto 27''  e clicca sull'opzione ``Calcola metrica per il numero di righe aggiunte e rimosse''.	\\
\hline
\cellcolor{lightgray}Risultato atteso	& Vengono salvati nel database i valori 0, 0, 0 relativi a questa metrica.\\
\hline
\end{tabular}
\end{table}

\begin{table}[ht]
\begin{tabular}{|p{3cm}|p{9cm}|}
\hline
\cellcolor{lightgray}Codice				& TC 6.9								\\
\hline
\cellcolor{lightgray}Combinazione		& fnum2ftype2fver1cnum1									\\
\hline
\cellcolor{lightgray}Precondizione		& L'utente ha aperto Eclipse e ha installato il plugin.		\\
\hline
\cellcolor{lightgray}Comando			& L'utente seleziona il progetto ``Progetto 16''  e clicca sull'opzione ``Calcola metrica per il numero di righe aggiunte e rimosse''.	\\
\hline
\cellcolor{lightgray}Risultato atteso	& Vengono salvati nel database i valori 0, 0, 0 relativi a questa metrica.\\
\hline
\end{tabular}
\end{table}

\begin{table}[ht]
\begin{tabular}{|p{3cm}|p{9cm}|}
\hline
\cellcolor{lightgray}Codice				& TC 6.10								\\
\hline
\cellcolor{lightgray}Combinazione		& fnum2ftype2fver1cnum2 									\\
\hline
\cellcolor{lightgray}Precondizione		& L'utente ha aperto Eclipse e ha installato il plugin.				\\
\hline
\cellcolor{lightgray}Comando			& L'utente seleziona il progetto ``Progetto 17''  e clicca sull'opzione ``Calcola metrica per il numero di righe aggiunte e rimosse''.	\\
\hline
\cellcolor{lightgray}Risultato atteso	& Vengono salvati nel database i valori 0, 0, 0 relativi a questa metrica.\\
\hline
\end{tabular}
\end{table}

\begin{table}[ht]
\begin{tabular}{|p{3cm}|p{9cm}|}
\hline
\cellcolor{lightgray}Codice				& TC 6.11								\\
\hline
\cellcolor{lightgray}Combinazione		& fnum2ftype2fver2cnum1									\\
\hline
\cellcolor{lightgray}Precondizione		& L'utente ha aperto Eclipse e ha installato il plugin.					\\
\hline
\cellcolor{lightgray}Comando			& L'utente seleziona il progetto ``Progetto 11''  e clicca sull'opzione ``Calcola metrica per il numero di righe aggiunte e rimosse''.	\\
\hline
\cellcolor{lightgray}Risultato atteso	& Vengono salvati nel database i valori 0, 0, 0 relativi a questa metrica.\\
\hline
\end{tabular}
\end{table}

\begin{table}[ht]
\begin{tabular}{|p{3cm}|p{9cm}|}
\hline
\cellcolor{lightgray}Codice				& TC 6.12								\\
\hline
\cellcolor{lightgray}Combinazione		& fnum2ftype2fver2cnum2								\\
\hline
\cellcolor{lightgray}Precondizione		& L'utente ha aperto Eclipse e ha installato il plugin.			\\
\hline
\cellcolor{lightgray}Comando			& L'utente seleziona il progetto ``Progetto 12''  e clicca sull'opzione ``Calcola metrica per il numero di righe aggiunte e rimosse''.	\\
\hline
\cellcolor{lightgray}Risultato atteso	& Vengono salvati nel database i valori 0, 0, 0 relativi a questa metrica.\\
\hline
\end{tabular}
\end{table}

\clearpage

\section{Metrica per il calcolo della dimensione media dei file modificati}

\begin{table}[ht]
\begin{tabular}{|p{3cm}|p{9cm}|}
\hline
\cellcolor{lightgray}Codice				& TC 7.1								\\
\hline
\cellcolor{lightgray}Combinazione		& fnum1cnum1									\\
\hline
\cellcolor{lightgray}Precondizione		& L'utente ha aperto Eclipse e ha installato il plugin.		\\
\hline
\cellcolor{lightgray}Comando			& L'utente seleziona il progetto ``Vuoto''  e clicca sull'opzione ``Calcola metrica per la dimensione media dei file modificati''.	\\
\hline
\cellcolor{lightgray}Risultato atteso	& Viene salvato nel database il valore 0 relativo a questa metrica.\\
\hline
\end{tabular}
\end{table}

\begin{table}[ht]
\begin{tabular}{|p{3cm}|p{9cm}|}
\hline
\cellcolor{lightgray}Codice				& TC 7.2								\\
\hline
\cellcolor{lightgray}Combinazione		& fnum1cnum2 									\\
\hline
\cellcolor{lightgray}Precondizione		& L'utente ha aperto Eclipse e ha installato il plugin.				\\
\hline
\cellcolor{lightgray}Comando			& L'utente seleziona il progetto ``Vuoto Versioning 1''  e clicca sull'opzione ``Calcola metrica per la dimensione media dei file modificati''.	\\
\hline
\cellcolor{lightgray}Risultato atteso	& Viene salvato nel database il valore 0 relativo a questa metrica	\\
\hline
\end{tabular}
\end{table}

\begin{table}[ht]
\begin{tabular}{|p{3cm}|p{9cm}|}
\hline
\cellcolor{lightgray}Codice				& TC 7.3								\\
\hline
\cellcolor{lightgray}Combinazione		& fnum2ftype1fver1cnum1								\\
\hline
\cellcolor{lightgray}Precondizione		& L'utente ha aperto Eclipse e ha installato il plugin.					\\
\hline
\cellcolor{lightgray}Comando			& L'utente seleziona il progetto ``Progetto 6''  e clicca sull'opzione ``Calcola metrica per la dimensione media dei file modificati''.	\\
\hline
\cellcolor{lightgray}Risultato atteso	& Viene salvato nel database il valore 0 relativo a questa metrica	\\
\hline
\end{tabular}
\end{table}

\begin{table}[ht]
\begin{tabular}{|p{3cm}|p{9cm}|}
\hline
\cellcolor{lightgray}Codice				& TC 7.4								\\
\hline
\cellcolor{lightgray}Combinazione		& fnum2ftype1fver1cnum2							\\
\hline
\cellcolor{lightgray}Precondizione		& L'utente ha aperto Eclipse e ha installato il plugin.			\\
\hline
\cellcolor{lightgray}Comando			& L'utente seleziona il progetto ``Progetto 7''  e clicca sull'opzione ``Calcola metrica per la dimensione media dei file modificati''.	\\
\hline
\cellcolor{lightgray}Risultato atteso	& Viene salvato nel database il valore 11 relativo a questa metrica	\\
\hline
\end{tabular}
\end{table}

\begin{table}[ht]
\begin{tabular}{|p{3cm}|p{9cm}|}
\hline
\cellcolor{lightgray}Codice				& TC 7.5								\\
\hline
\cellcolor{lightgray}Combinazione		& fnum2ftype1fver2cnum1 									\\
\hline
\cellcolor{lightgray}Precondizione		& L'utente ha aperto Eclipse e ha installato il plugin.			\\
\hline
\cellcolor{lightgray}Comando			& L'utente seleziona il progetto ``Progetto 1''  e clicca sull'opzione ``Calcola metrica per la dimensione media dei file modificati''.	\\
\hline
\cellcolor{lightgray}Risultato atteso	& Viene salvato nel database il valore 0 relativo a questa metrica	\\
\hline
\end{tabular}
\end{table}

\begin{table}[ht]
\begin{tabular}{|p{3cm}|p{9cm}|}
\hline
\cellcolor{lightgray}Codice				& TC 7.6								\\
\hline
\cellcolor{lightgray}Combinazione		& fnum2ftype1fver2cnum2								\\
\hline
\cellcolor{lightgray}Precondizione		& L'utente ha aperto Eclipse e ha installato il plugin.				\\
\hline
\cellcolor{lightgray}Comando			& L'utente seleziona il progetto ``Progetto 2''  e clicca sull'opzione ``Calcola metrica per la dimensione media dei file modificati''.	\\
\hline
\cellcolor{lightgray}Risultato atteso	& Viene salvato nel database il valore 0 relativo a questa metrica	\\
\hline
\end{tabular}
\end{table}

\begin{table}[ht]
\begin{tabular}{|p{3cm}|p{9cm}|}
\hline
\cellcolor{lightgray}Codice				& TC 7.7								\\
\hline
\cellcolor{lightgray}Combinazione		& fnum2ftype1fver1cnum2ccan2cins1							\\
\hline
\cellcolor{lightgray}Precondizione		& L'fnum2ftype2fver1cnum1 ha aperto Eclipse e ha installato il plugin.									\\
\hline
\cellcolor{lightgray}Comando			& L'utente seleziona il progetto ``Progetto 16''  e clicca sull'opzione ``Calcola metrica per la dimensione media dei file modificati''.	\\
\hline
\cellcolor{lightgray}Risultato atteso	& Viene salvato nel database il valore 0 relativo a questa metrica	\\
\hline
\end{tabular}
\end{table}

\begin{table}[ht]
\begin{tabular}{|p{3cm}|p{9cm}|}
\hline
\cellcolor{lightgray}Codice				& TC 7.8								\\
\hline
\cellcolor{lightgray}Combinazione		& fnum2ftype2fver1cnum2 									\\
\hline
\cellcolor{lightgray}Precondizione		& L'utente ha aperto Eclipse e ha installato il plugin.				\\
\hline
\cellcolor{lightgray}Comando			& L'utente seleziona il progetto ``Progetto 17''  e clicca sull'opzione ``Calcola metrica per la dimensione media dei file modificati''.	\\
\hline
\cellcolor{lightgray}Risultato atteso	& Viene salvato nel database il valore 0 relativo a questa metrica	\\
\hline
\end{tabular}
\end{table}

\begin{table}[ht]
\begin{tabular}{|p{3cm}|p{9cm}|}
\hline
\cellcolor{lightgray}Codice				& TC 7.9								\\
\hline
\cellcolor{lightgray}Combinazione		& fnum2ftype2fver2cnum1 									\\
\hline
\cellcolor{lightgray}Precondizione		& L'utente ha aperto Eclipse e ha installato il plugin.								\\
\hline
\cellcolor{lightgray}Comando			& L'utente seleziona il progetto ``Progetto 11''  e clicca sull'opzione ``Calcola metrica per la dimensione media dei file modificati''.	\\
\hline
\cellcolor{lightgray}Risultato atteso	& Viene salvato nel database il valore 0 relativo a questa metrica	\\
\hline
\end{tabular}
\end{table}

\begin{table}[ht]
\begin{tabular}{|p{3cm}|p{9cm}|}
\hline
\cellcolor{lightgray}Codice				& TC 7.10								\\
\hline
\cellcolor{lightgray}Combinazione		& fnum2ftype2fver2cnum2 									\\
\hline
\cellcolor{lightgray}Precondizione		& L'utente ha aperto Eclipse e ha installato il plugin.									\\
\hline
\cellcolor{lightgray}Comando			& L'utente seleziona il progetto ``Progetto 12''  e clicca sull'opzione ``Calcola metrica per la dimensione media dei file modificati''.	\\
\hline
\cellcolor{lightgray}Risultato atteso	& Viene salvato nel database il valore 0 relativo a questa metrica	\\
\hline
\end{tabular}
\end{table}

\clearpage

\section{Metrica per il calcolo del numero di cambiamenti totali di ogni file di un dato progetto (inserimento di feature)}
I casi d'uso seguenti si riferiscono solo al test di unità. Nell'oracolo sono indicati i file e il relativo valore della metrica.\\

\begin{table}[ht]
\begin{tabular}{|p{3cm}|p{9cm}|}
\hline
\cellcolor{lightgray}Codice				& TC 8.1								\\
\hline
\cellcolor{lightgray}Combinazione		& fnum1cnum1ccan1cins1									\\
\hline
\cellcolor{lightgray}Precondizione		& L'utente ha aperto Eclipse e ha installato il plugin.		\\
\hline
\cellcolor{lightgray}Comando			& L'utente seleziona il progetto ``Progetto ?''  e clicca sull'opzione ``Calcola metrica per il numero di cambiamenti totali di ogni file di un dato progetto''.	\\
\hline
\cellcolor{lightgray}Risultato atteso	& È associato a tutti i file il valore 0 relativo a questa metrica.\\
\hline
\end{tabular}
\end{table}

\begin{table}[ht]
\begin{tabular}{|p{3cm}|p{9cm}|}
\hline
\cellcolor{lightgray}Codice				& TC 8.2								\\
\hline
\cellcolor{lightgray}Combinazione		& fnum1cnum2ccan2cins1cmes1									\\
\hline
\cellcolor{lightgray}Precondizione		& L'utente ha aperto Eclipse e ha installato il plugin.		\\
\hline
\cellcolor{lightgray}Comando			& L'utente seleziona il progetto ``Vuoto Versioning 1''  e clicca sull'opzione ``Calcola metrica per il numero di cambiamenti totali di ogni file di un dato progetto''.	\\
\hline
\cellcolor{lightgray}Risultato atteso	& È associato a tutti i file il valore 0 relativo a questa metrica.\\
\hline
\end{tabular}
\end{table}

\begin{table}[ht]
\begin{tabular}{|p{3cm}|p{9cm}|}
\hline
\cellcolor{lightgray}Codice				& TC 8.3								\\
\hline
\cellcolor{lightgray}Combinazione		& fnum1cnum2ccan2cins1cmes2									\\
\hline
\cellcolor{lightgray}Precondizione		& L'utente ha aperto Eclipse e ha installato il plugin.		\\
\hline
\cellcolor{lightgray}Comando			& L'utente seleziona il progetto ``Vuoto Versioning 3''  e clicca sull'opzione ``Calcola metrica per il numero di cambiamenti totali di ogni file di un dato progetto''.	\\
\hline
\cellcolor{lightgray}Risultato atteso	& È associato a tutti i file il valore 0 relativo a questa metrica.\\
\hline
\end{tabular}
\end{table}

\begin{table}[ht]
\begin{tabular}{|p{3cm}|p{9cm}|}
\hline
\cellcolor{lightgray}Codice				& TC 8.4								\\
\hline
\cellcolor{lightgray}Combinazione		& fnum2ftype1fver1cnum1ccan1cins1									\\
\hline
\cellcolor{lightgray}Precondizione		& L'utente ha aperto Eclipse e ha installato il plugin.		\\
\hline
\cellcolor{lightgray}Comando			& L'utente seleziona il progetto ``Progetto 6''  e clicca sull'opzione ``Calcola metrica per il numero di cambiamenti totali di ogni file di un dato progetto''.	\\
\hline
\cellcolor{lightgray}Risultato atteso	& È associato a tutti i file il valore 0 relativo a questa metrica.\\
\hline
\end{tabular}
\end{table}

\begin{table}[ht]
\begin{tabular}{|p{3cm}|p{9cm}|}
\hline
\cellcolor{lightgray}Codice				& TC 8.5								\\
\hline
\cellcolor{lightgray}Combinazione		& fnum2ftype1fver1cnum2ccan1cins2cmes1									\\
\hline
\cellcolor{lightgray}Precondizione		& L'utente ha aperto Eclipse e ha installato il plugin.		\\
\hline
\cellcolor{lightgray}Comando			& L'utente seleziona il progetto ``Progetto 21''  e clicca sull'opzione ``Calcola metrica per il numero di cambiamenti totali di ogni file di un dato progetto''.	\\
\hline
\cellcolor{lightgray}Risultato atteso	& È associato al file Primo.java il valore 1 relativo a questa metrica.\\
\hline
\end{tabular}
\end{table}

\begin{table}[ht]
\begin{tabular}{|p{3cm}|p{9cm}|}
\hline
\cellcolor{lightgray}Codice				& TC 8.6								\\
\hline
\cellcolor{lightgray}Combinazione		& fnum2ftype1fver1cnum2ccan2cins1cmes1									\\
\hline
\cellcolor{lightgray}Precondizione		& L'utente ha aperto Eclipse e ha installato il plugin.		\\
\hline
\cellcolor{lightgray}Comando			& L'utente seleziona il progetto ``Progetto 22''  e clicca sull'opzione ``Calcola metrica per il numero di cambiamenti totali di ogni file di un dato progetto''.	\\
\hline
\cellcolor{lightgray}Risultato atteso	& È associato al file Primo.java il valore 1 relativo a questa metrica.\\
\hline
\end{tabular}
\end{table}

\begin{table}[ht]
\begin{tabular}{|p{3cm}|p{9cm}|}
\hline
\cellcolor{lightgray}Codice				& TC 8.7								\\
\hline
\cellcolor{lightgray}Combinazione		& fnum2ftype1fver1cnum2ccan2cins2cmes1									\\
\hline
\cellcolor{lightgray}Precondizione		& L'utente ha aperto Eclipse e ha installato il plugin.		\\
\hline
\cellcolor{lightgray}Comando			& L'utente seleziona il progetto ``Progetto 23''  e clicca sull'opzione 
``Calcola metrica per il numero di cambiamenti totali di ogni file di un dato progetto''.	\\
\hline
\cellcolor{lightgray}Risultato atteso	& È associato al file Primo.java il valore 1 relativo a questa metrica.\\
\hline
\end{tabular}
\end{table}

\begin{table}[ht]
\begin{tabular}{|p{3cm}|p{9cm}|}
\hline
\cellcolor{lightgray}Codice				& TC 8.8								\\
\hline
\cellcolor{lightgray}Combinazione		& fnum2ftype1fver1cnum2ccan1cins2cmes2									\\
\hline
\cellcolor{lightgray}Precondizione		& L'utente ha aperto Eclipse e ha installato il plugin.		\\
\hline
\cellcolor{lightgray}Comando			& L'utente seleziona il progetto ``Progetto 24''  e clicca sull'opzione ``Calcola metrica per il numero di cambiamenti totali di ogni file di un dato progetto''.	\\
\hline
\cellcolor{lightgray}Risultato atteso	& Sono associati al file Primo.java il valore 0, al file Secondo.java il valore 1 e al file Terzo.java il valore 0 relativi a questa metrica.\\
\hline
\end{tabular}
\end{table}

\begin{table}[ht]
\begin{tabular}{|p{3cm}|p{9cm}|}
\hline
\cellcolor{lightgray}Codice				& TC 8.9								\\
\hline
\cellcolor{lightgray}Combinazione		& fnum2ftype1fver1cnum2ccan2cins1cmes2									\\
\hline
\cellcolor{lightgray}Precondizione		& L'utente ha aperto Eclipse e ha installato il plugin.		\\
\hline
\cellcolor{lightgray}Comando			& L'utente seleziona il progetto ``Progetto 25''  e clicca sull'opzione ``Calcola metrica per il numero di cambiamenti totali di ogni file di un dato progetto''.	\\
\hline
\cellcolor{lightgray}Risultato atteso	& È associato a tutti i file il valore 0 relativo a questa metrica.\\
\hline
\end{tabular}
\end{table}

\clearpage

\begin{table}[ht]
\begin{tabular}{|p{3cm}|p{9cm}|}
\hline
\cellcolor{lightgray}Codice				& TC 8.10								\\
\hline
\cellcolor{lightgray}Combinazione		& fnum2ftype1fver1cnum2ccan2cins2cmes2									\\
\hline
\cellcolor{lightgray}Precondizione		& L'utente ha aperto Eclipse e ha installato il plugin.		\\
\hline
\cellcolor{lightgray}Comando			& L'utente seleziona il progetto ``Progetto 26''  e clicca sull'opzione ``Calcola metrica per il numero di cambiamenti totali di ogni file di un dato progetto''.	\\
\hline
\cellcolor{lightgray}Risultato atteso	& Sono associati al file Primo.java il valore 0, al file Secondo.java il valore 0 e al file Terzo.java il valore 1 relativi a questa metrica.\\
\hline
\end{tabular}
\end{table}

\begin{table}[ht]
\begin{tabular}{|p{3cm}|p{9cm}|}
\hline
\cellcolor{lightgray}Codice				& TC 8.11								\\
\hline
\cellcolor{lightgray}Combinazione		& fnum2ftype1fver2cnum1ccan1cins1									\\
\hline
\cellcolor{lightgray}Precondizione		& L'utente ha aperto Eclipse e ha installato il plugin.		\\
\hline
\cellcolor{lightgray}Comando			& L'utente seleziona il progetto ``Progetto 1''  e clicca sull'opzione ``Calcola metrica per il numero di cambiamenti totali di ogni file di un dato progetto''.	\\
\hline
\cellcolor{lightgray}Risultato atteso	& È associato a tutti i file il valore 0 relativo a questa metrica.\\
\hline
\end{tabular}
\end{table}

\begin{table}[ht]
\begin{tabular}{|p{3cm}|p{9cm}|}
\hline
\cellcolor{lightgray}Codice				& TC 8.12								\\
\hline
\cellcolor{lightgray}Combinazione		& fnum2ftype1fver2cnum2ccan2cins1cmes1									\\
\hline
\cellcolor{lightgray}Precondizione		& L'utente ha aperto Eclipse e ha installato il plugin.		\\
\hline
\cellcolor{lightgray}Comando			& L'utente seleziona il progetto ``Progetto 27''  e clicca sull'opzione ``Calcola metrica per il numero di cambiamenti totali di ogni file di un dato progetto''.	\\
\hline
\cellcolor{lightgray}Risultato atteso	& È associato a tutti i file il valore 0 relativo a questa metrica.\\
\hline
\end{tabular}
\end{table}

\begin{table}[ht]
\begin{tabular}{|p{3cm}|p{9cm}|}
\hline
\cellcolor{lightgray}Codice				& TC 8.13								\\
\hline
\cellcolor{lightgray}Combinazione		& fnum2ftype1fver2cnum2ccan2cins1cmes2									\\
\hline
\cellcolor{lightgray}Precondizione		& L'utente ha aperto Eclipse e ha installato il plugin.		\\
\hline
\cellcolor{lightgray}Comando			& L'utente seleziona il progetto ``Progetto 28''  e clicca sull'opzione ``Calcola metrica per il numero di cambiamenti totali di ogni file di un dato progetto''.	\\
\hline
\cellcolor{lightgray}Risultato atteso	& È associato a tutti i file il valore 0 relativo a questa metrica.\\
\hline
\end{tabular}
\end{table}

\begin{table}[ht]
\begin{tabular}{|p{3cm}|p{9cm}|}
\hline
\cellcolor{lightgray}Codice				& TC 8.14								\\
\hline
\cellcolor{lightgray}Combinazione		& fnum2ftype2fver1cnum1									\\
\hline
\cellcolor{lightgray}Precondizione		& L'utente ha aperto Eclipse e ha installato il plugin.		\\
\hline
\cellcolor{lightgray}Comando			& L'utente seleziona il progetto ``Progetto 16''  e clicca sull'opzione ``Calcola metrica per il numero di cambiamenti totali di ogni file di un dato progetto''.	\\
\hline
\cellcolor{lightgray}Risultato atteso	& È associato a tutti i file il valore 0 relativo a questa metrica.\\
\hline
\end{tabular}
\end{table}

\begin{table}[ht]
\begin{tabular}{|p{3cm}|p{9cm}|}
\hline
\cellcolor{lightgray}Codice				& TC 8.15								\\
\hline
\cellcolor{lightgray}Combinazione		& fnum2ftype2fver1cnum2cmes1									\\
\hline
\cellcolor{lightgray}Precondizione		& L'utente ha aperto Eclipse e ha installato il plugin.		\\
\hline
\cellcolor{lightgray}Comando			& L'utente seleziona il progetto ``Progetto 29''  e clicca sull'opzione ``Calcola metrica per il numero di cambiamenti totali di ogni file di un dato progetto''.	\\
\hline
\cellcolor{lightgray}Risultato atteso	& È associato al file Primo.pdf il valore 1 relativo a questa metrica.\\
\hline
\end{tabular}
\end{table}

\begin{table}[ht]
\begin{tabular}{|p{3cm}|p{9cm}|}
\hline
\cellcolor{lightgray}Codice				& TC 8.16								\\
\hline
\cellcolor{lightgray}Combinazione		& fnum2ftype2fver1cnum2cmes2									\\
\hline
\cellcolor{lightgray}Precondizione		& L'utente ha aperto Eclipse e ha installato il plugin.		\\
\hline
\cellcolor{lightgray}Comando			& L'utente seleziona il progetto ``Progetto 30''  e clicca sull'opzione 
``Calcola metrica per il numero di cambiamenti totali di ogni file di un dato progetto''.	\\
\hline
\cellcolor{lightgray}Risultato atteso	& È associato a tutti i file il valore 0 relativo a questa metrica.\\
\hline
\end{tabular}
\end{table}

\begin{table}[ht]
\begin{tabular}{|p{3cm}|p{9cm}|}
\hline
\cellcolor{lightgray}Codice				& TC 8.17								\\
\hline
\cellcolor{lightgray}Combinazione		& fnum2ftype2fver2cnum1									\\
\hline
\cellcolor{lightgray}Precondizione		& L'utente ha aperto Eclipse e ha installato il plugin.		\\
\hline
\cellcolor{lightgray}Comando			& L'utente seleziona il progetto ``Progetto 11''  e clicca sull'opzione ``Calcola metrica per il numero di cambiamenti totali di ogni file di un dato progetto''.	\\
\hline
\cellcolor{lightgray}Risultato atteso	& È associato a tutti i file il valore 0 relativo a questa metrica.\\
\hline
\end{tabular}
\end{table}

\begin{table}[ht]
\begin{tabular}{|p{3cm}|p{9cm}|}
\hline
\cellcolor{lightgray}Codice				& TC 8.18								\\
\hline
\cellcolor{lightgray}Combinazione		& fnum2ftype2fver2cnum2cmes1									\\
\hline
\cellcolor{lightgray}Precondizione		& L'utente ha aperto Eclipse e ha installato il plugin.		\\
\hline
\cellcolor{lightgray}Comando			& L'utente seleziona il progetto ``Progetto 31''  e clicca sull'opzione ``Calcola metrica per il numero di cambiamenti totali di ogni file di un dato progetto''.	\\
\hline
\cellcolor{lightgray}Risultato atteso	& È associato a tutti i file il valore 0 relativo a questa metrica.\\
\hline
\end{tabular}
\end{table}

\begin{table}[ht]
\begin{tabular}{|p{3cm}|p{9cm}|}
\hline
\cellcolor{lightgray}Codice				& TC 8.19								\\
\hline
\cellcolor{lightgray}Combinazione		& fnum2ftype2fver2cnum2cmes2									\\
\hline
\cellcolor{lightgray}Precondizione		& L'utente ha aperto Eclipse e ha installato il plugin.		\\
\hline
\cellcolor{lightgray}Comando			& L'utente seleziona il progetto ``Progetto 32''  e clicca sull'opzione ``Calcola metrica per il numero di cambiamenti totali di ogni file di un dato progetto''.	\\
\hline
\cellcolor{lightgray}Risultato atteso	& È associato a tutti i file il valore 0 relativo a questa metrica.\\
\hline
\end{tabular}
\end{table}

\clearpage

\section{Metrica per il calcolo del Basic Code Change model}

\begin{table}[ht]
\begin{tabular}{|p{3cm}|p{9cm}|}
\hline
\cellcolor{lightgray}Codice				& TC 9.1								\\
\hline
\cellcolor{lightgray}Combinazione		& fnum1cnum1									\\
\hline
\cellcolor{lightgray}Precondizione		& L'utente ha aperto Eclipse e ha installato il plugin.		\\
\hline
\cellcolor{lightgray}Comando			& L'utente seleziona il progetto ``Vuoto''  e clicca sull'opzione ``Calcola metrica per il Basic Code Change model''.	\\
\hline
\cellcolor{lightgray}Risultato atteso	& Viene salvato nel database il valore 0 relativo a questa metrica.\\
\hline
\end{tabular}
\end{table}


\begin{table}[ht]
\begin{tabular}{|p{3cm}|p{9cm}|}
\hline
\cellcolor{lightgray}Codice				& TC 9.2								\\
\hline
\cellcolor{lightgray}Combinazione		& fnum1cnum2pclass1ptype1									\\
\hline
\cellcolor{lightgray}Precondizione		& L'utente ha aperto Eclipse e ha installato il plugin.		\\
\hline
\cellcolor{lightgray}Comando			& L'utente apre la configurazione del plugin, seleziona sorgente ``Tempo'' e inserisce il valore ``abcdef''. L'utente seleziona il progetto ``Vuoto Versioning 1''  e clicca sull'opzione ``Calcola metrica per il Basic Code Change model''.	\\
\hline
\cellcolor{lightgray}Risultato atteso	& Viene restituito un errore.\\
\hline
\end{tabular}
\end{table}


\begin{table}[ht]
\begin{tabular}{|p{3cm}|p{9cm}|}
\hline
\cellcolor{lightgray}Codice				& TC 9.3								\\
\hline
\cellcolor{lightgray}Combinazione		& fnum1cnum2pclass1ptype2plun1									\\
\hline
\cellcolor{lightgray}Precondizione		& L'utente ha aperto Eclipse e ha installato il plugin.		\\
\hline
\cellcolor{lightgray}Comando			& L'utente apre la configurazione del plugin, seleziona sorgente ``Tempo'' e inserisce il valore ``-12''. L'utente seleziona il progetto ``Vuoto Versioning 2''  e clicca sull'opzione ``Calcola metrica per il Basic Code Change model''.	\\
\hline
\cellcolor{lightgray}Risultato atteso	& Viene restituito un errore.\\
\hline
\end{tabular}
\end{table}


\begin{table}[ht]
\begin{tabular}{|p{3cm}|p{9cm}|}
\hline
\cellcolor{lightgray}Codice				& TC 9.4								\\
\hline
\cellcolor{lightgray}Combinazione		& fnum1cnum2pclass1ptype2plun2									\\
\hline
\cellcolor{lightgray}Precondizione		& L'utente ha aperto Eclipse e ha installato il plugin.		\\
\hline
\cellcolor{lightgray}Comando			& L'utente apre la configurazione del plugin, seleziona sorgente ``Tempo'' e inserisce il valore ``1000000000''. L'utente seleziona il progetto ``Vuoto Versioning 3''  e clicca sull'opzione ``Calcola metrica per il Basic Code Change model''.	\\
\hline
\cellcolor{lightgray}Risultato atteso	& Viene restituito un errore.\\
\hline
\end{tabular}
\end{table}


\begin{table}[ht]
\begin{tabular}{|p{3cm}|p{9cm}|}
\hline
\cellcolor{lightgray}Codice				& TC 9.5								\\
\hline
\cellcolor{lightgray}Combinazione		& fnum1cnum2pclass1ptype2plun3									\\
\hline
\cellcolor{lightgray}Precondizione		& L'utente ha aperto Eclipse e ha installato il plugin.		\\
\hline
\cellcolor{lightgray}Comando			& L'utente apre la configurazione del plugin, seleziona sorgente ``Tempo'' e inserisce il valore ``18''. L'utente seleziona il progetto ``Vuoto Versioning 4''  e clicca sull'opzione ``Calcola metrica per il Basic Code Change model''.	\\
\hline
\cellcolor{lightgray}Risultato atteso	& Viene salvato nel database il valore 0 relativo a questa metrica.\\
\hline
\end{tabular}
\end{table}


\begin{table}[ht]
\begin{tabular}{|p{3cm}|p{9cm}|}
\hline
\cellcolor{lightgray}Codice				& TC 9.6								\\
\hline
\cellcolor{lightgray}Combinazione		& fnum1cnum2pclass2									\\
\hline
\cellcolor{lightgray}Precondizione		& L'utente ha aperto Eclipse e ha installato il plugin.		\\
\hline
\cellcolor{lightgray}Comando			& L'utente apre la configurazione del plugin, seleziona un'altra sorgente. L'utente seleziona il progetto ``Vuoto Versioning 4''  e clicca sull'opzione ``Calcola metrica per il Basic Code Change model''.	\\
\hline
\cellcolor{lightgray}Risultato atteso	& Viene restituito un errore.\\
\hline
\end{tabular}
\end{table}

\begin{table}[ht]
\begin{tabular}{|p{3cm}|p{9cm}|}
\hline
\cellcolor{lightgray}Codice				& TC 9.7								\\
\hline
\cellcolor{lightgray}Combinazione		& fnum2ftype1fver1cnum2pclass1ptype1									\\
\hline
\cellcolor{lightgray}Precondizione		& L'utente ha aperto Eclipse e ha installato il plugin.		\\
\hline
\cellcolor{lightgray}Comando			& L'utente apre la configurazione del plugin, seleziona sorgente ``Tempo'' e inserisce il valore ``abcdef''. L'utente seleziona il progetto ``Progetto 7''  e clicca sull'opzione ``Calcola metrica per il Basic Code Change model''.	\\
\hline
\cellcolor{lightgray}Risultato atteso	& Viene restituito un errore.\\
\hline
\end{tabular}
\end{table}

\clearpage

\begin{table}[ht]
\begin{tabular}{|p{3cm}|p{9cm}|}
\hline
\cellcolor{lightgray}Codice				& TC 9.8								\\
\hline
\cellcolor{lightgray}Combinazione		& fnum2ftype1fver1cnum2pclass1ptype2plun1									\\
\hline
\cellcolor{lightgray}Precondizione		& L'utente ha aperto Eclipse e ha installato il plugin.		\\
\hline
\cellcolor{lightgray}Comando			& L'utente apre la configurazione del plugin, seleziona sorgente ``Tempo'' e inserisce il valore ``-12''. L'utente seleziona il progetto ``Progetto 8''  e clicca sull'opzione ``Calcola metrica per il Basic Code Change model''.	\\
\hline
\cellcolor{lightgray}Risultato atteso	& Viene restituito un errore.\\
\hline
\end{tabular}
\end{table}

\begin{table}[ht]
\begin{tabular}{|p{3cm}|p{9cm}|}
\hline
\cellcolor{lightgray}Codice				& TC 9.9								\\
\hline
\cellcolor{lightgray}Combinazione		& fnum2ftype1fver1cnum2pclass1ptype2plun2									\\
\hline
\cellcolor{lightgray}Precondizione		& L'utente ha aperto Eclipse e ha installato il plugin.		\\
\hline
\cellcolor{lightgray}Comando			& L'utente apre la configurazione del plugin, seleziona sorgente ``Tempo'' e inserisce il valore ``1000000000''. L'utente seleziona il progetto ``Progetto 9''  e clicca sull'opzione ``Calcola metrica per il Basic Code Change model''.	\\
\hline
\cellcolor{lightgray}Risultato atteso	& Viene restituito un errore.\\
\hline
\end{tabular}
\end{table}

\begin{table}[ht]
\begin{tabular}{|p{3cm}|p{9cm}|}
\hline
\cellcolor{lightgray}Codice				& TC 9.10								\\
\hline
\cellcolor{lightgray}Combinazione		& fnum2ftype1fver1cnum2pclass1ptype2plun3									\\
\hline
\cellcolor{lightgray}Precondizione		& L'utente ha aperto Eclipse e ha installato il plugin.		\\
\hline
\cellcolor{lightgray}Comando			& L'utente apre la configurazione del plugin, seleziona sorgente ``Tempo'' e inserisce il valore ``18''. L'utente seleziona il progetto ``Progetto 10''  e clicca sull'opzione ``Calcola metrica per il Basic Code Change model''.	\\
\hline
\cellcolor{lightgray}Risultato atteso	& Viene salvato nel database il valore 1 relativo a questa metrica.\\
\hline
\end{tabular}
\end{table}


\begin{table}[ht]
\begin{tabular}{|p{3cm}|p{9cm}|}
\hline
\cellcolor{lightgray}Codice				& TC 9.11								\\
\hline
\cellcolor{lightgray}Combinazione		& fnum2ftype1fver1cnum2pclass2									\\
\hline
\cellcolor{lightgray}Precondizione		& L'utente ha aperto Eclipse e ha installato il plugin.		\\
\hline
\cellcolor{lightgray}Comando			& L'utente apre la configurazione del plugin, seleziona un'altra sorgente. L'utente seleziona il progetto ``Progetto 26''  e clicca sull'opzione ``Calcola metrica per il Basic Code Change model''.	\\
\hline
\cellcolor{lightgray}Risultato atteso	& Viene restituito un errore.\\
\hline
\end{tabular}
\end{table}

\clearpage

\begin{table}[ht]
\begin{tabular}{|p{3cm}|p{9cm}|}
\hline
\cellcolor{lightgray}Codice				& TC 9.12								\\
\hline
\cellcolor{lightgray}Combinazione		& fnum2ftype1fver2cnum1									\\
\hline
\cellcolor{lightgray}Precondizione		& L'utente ha aperto Eclipse e ha installato il plugin.		\\
\hline
\cellcolor{lightgray}Comando			& L'utente seleziona il progetto ``Progetto 1''  e clicca sull'opzione ``Calcola metrica per il Basic Code Change model''.	\\
\hline
\cellcolor{lightgray}Risultato atteso	& Viene restituito un errore.\\
\hline
\end{tabular}
\end{table}


\begin{table}[ht]
\begin{tabular}{|p{3cm}|p{9cm}|}
\hline
\cellcolor{lightgray}Codice				& TC 9.13								\\
\hline
\cellcolor{lightgray}Combinazione		& fnum2ftype1fver2cnum2pclass1ptype1									\\
\hline
\cellcolor{lightgray}Precondizione		& L'utente ha aperto Eclipse e ha installato il plugin.		\\
\hline
\cellcolor{lightgray}Comando			& L'utente apre la configurazione del plugin, seleziona sorgente ``Tempo'' e inserisce il valore ``abcdef''. L'utente seleziona il progetto ``Progetto 2''  e clicca sull'opzione ``Calcola metrica per il Basic Code Change model''.	\\
\hline
\cellcolor{lightgray}Risultato atteso	& Viene restituito un errore.\\
\hline
\end{tabular}
\end{table}


\begin{table}[ht]
\begin{tabular}{|p{3cm}|p{9cm}|}
\hline
\cellcolor{lightgray}Codice				& TC 9.14								\\
\hline
\cellcolor{lightgray}Combinazione		& fnum2ftype1fver2cnum2pclass1ptype2plun1									\\
\hline
\cellcolor{lightgray}Precondizione		& L'utente ha aperto Eclipse e ha installato il plugin.		\\
\hline
\cellcolor{lightgray}Comando			& L'utente apre la configurazione del plugin, seleziona sorgente ``Tempo'' e inserisce il valore ``-12''. L'utente seleziona il progetto ``Progetto 3''  e clicca sull'opzione ``Calcola metrica per il Basic Code Change model''.	\\
\hline
\cellcolor{lightgray}Risultato atteso	& Viene restituito un errore.\\
\hline
\end{tabular}
\end{table}


\begin{table}[ht]
\begin{tabular}{|p{3cm}|p{9cm}|}
\hline
\cellcolor{lightgray}Codice				& TC 9.15								\\
\hline
\cellcolor{lightgray}Combinazione		& fnum2ftype1fver2cnum2pclass1ptype2plun2									\\
\hline
\cellcolor{lightgray}Precondizione		& L'utente ha aperto Eclipse e ha installato il plugin.		\\
\hline
\cellcolor{lightgray}Comando			& L'utente apre la configurazione del plugin, seleziona sorgente ``Tempo'' e inserisce il valore ``1000000000''. L'utente seleziona il progetto ``Progetto 4''  e clicca sull'opzione ``Calcola metrica per il Basic Code Change model''.	\\
\hline
\cellcolor{lightgray}Risultato atteso	& Viene restituito un errore.\\
\hline
\end{tabular}
\end{table}


\begin{table}[ht]
\begin{tabular}{|p{3cm}|p{9cm}|}
\hline
\cellcolor{lightgray}Codice				& TC 9.16								\\
\hline
\cellcolor{lightgray}Combinazione		& fnum2ftype1fver2cnum2pclass1ptype2plun3									\\
\hline
\cellcolor{lightgray}Precondizione		& L'utente ha aperto Eclipse e ha installato il plugin.		\\
\hline
\cellcolor{lightgray}Comando			& L'utente apre la configurazione del plugin, seleziona sorgente ``Tempo'' e inserisce il valore ``18''. L'utente seleziona il progetto ``Progetto 5''  e clicca sull'opzione ``Calcola metrica per il Basic Code Change model''.	\\
\hline
\cellcolor{lightgray}Risultato atteso	& Viene salvato nel database il valore 0 relativo a questa metrica.\\
\hline
\end{tabular}
\end{table}


\begin{table}[ht]
\begin{tabular}{|p{3cm}|p{9cm}|}
\hline
\cellcolor{lightgray}Codice				& TC 9.17								\\
\hline
\cellcolor{lightgray}Combinazione		& fnum2ftype1fver2cnum2pclass2									\\
\hline
\cellcolor{lightgray}Precondizione		& L'utente ha aperto Eclipse e ha installato il plugin.		\\
\hline
\cellcolor{lightgray}Comando			& L'utente apre la configurazione del plugin, seleziona un'altra sorgente. L'utente seleziona il progetto ``Progetto 3''  e clicca sull'opzione ``Calcola metrica per il Basic Code Change model''.	\\
\hline
\cellcolor{lightgray}Risultato atteso	& Viene restituito un errore.\\
\hline
\end{tabular}
\end{table}

\begin{table}[ht]
\begin{tabular}{|p{3cm}|p{9cm}|}
\hline
\cellcolor{lightgray}Codice				& TC 9.18								\\
\hline
\cellcolor{lightgray}Combinazione		& fnum2ftype2fver1cnum1									\\
\hline
\cellcolor{lightgray}Precondizione		& L'utente ha aperto Eclipse e ha installato il plugin.		\\
\hline
\cellcolor{lightgray}Comando			& L'utente seleziona il progetto ``Progetto 16''  e clicca sull'opzione ``Calcola metrica per il Basic Code Change model''.	\\
\hline
\cellcolor{lightgray}Risultato atteso	& Viene restituito un errore.\\
\hline
\end{tabular}
\end{table}

\clearpage

\begin{table}[ht]
\begin{tabular}{|p{3cm}|p{9cm}|}
\hline
\cellcolor{lightgray}Codice				& TC 9.19								\\
\hline
\cellcolor{lightgray}Combinazione		& fnum2ftype2fver1cnum2pclass1ptype1									\\
\hline
\cellcolor{lightgray}Precondizione		& L'utente ha aperto Eclipse e ha installato il plugin.		\\
\hline
\cellcolor{lightgray}Comando			& L'utente apre la configurazione del plugin, seleziona sorgente ``Tempo'' e inserisce il valore ``abcdef''. L'utente seleziona il progetto ``Progetto 17''  e clicca sull'opzione ``Calcola metrica per il Basic Code Change model''.	\\
\hline
\cellcolor{lightgray}Risultato atteso	& Viene restituito un errore.\\
\hline
\end{tabular}
\end{table}

\begin{table}[ht]
\begin{tabular}{|p{3cm}|p{9cm}|}
\hline
\cellcolor{lightgray}Codice				& TC 9.20								\\
\hline
\cellcolor{lightgray}Combinazione		& fnum2ftype2fver1cnum2pclass1ptype2plun1									\\
\hline
\cellcolor{lightgray}Precondizione		& L'utente ha aperto Eclipse e ha installato il plugin.		\\
\hline
\cellcolor{lightgray}Comando			& L'utente apre la configurazione del plugin, seleziona sorgente ``Tempo'' e inserisce il valore ``-12''. L'utente seleziona il progetto ``Progetto 18''  e clicca sull'opzione ``Calcola metrica per il Basic Code Change model''.	\\
\hline
\cellcolor{lightgray}Risultato atteso	& Viene restituito un errore.\\
\hline
\end{tabular}
\end{table}

\begin{table}[ht]
\begin{tabular}{|p{3cm}|p{9cm}|}
\hline
\cellcolor{lightgray}Codice				& TC 9.21								\\
\hline
\cellcolor{lightgray}Combinazione		& fnum2ftype2fver1cnum2pclass1ptype2plun2									\\
\hline
\cellcolor{lightgray}Precondizione		& L'utente ha aperto Eclipse e ha installato il plugin.		\\
\hline
\cellcolor{lightgray}Comando			& L'utente apre la configurazione del plugin, seleziona sorgente ``Tempo'' e inserisce il valore ``1000000000''. L'utente seleziona il progetto ``Progetto 19''  e clicca sull'opzione ``Calcola metrica per il Basic Code Change model''.	\\
\hline
\cellcolor{lightgray}Risultato atteso	& Viene restituito un errore.\\
\hline
\end{tabular}
\end{table}

\begin{table}[ht]
\begin{tabular}{|p{3cm}|p{9cm}|}
\hline
\cellcolor{lightgray}Codice				& TC 9.22								\\
\hline
\cellcolor{lightgray}Combinazione		& fnum2ftype2fver1cnum2pclass1ptype2plun3									\\
\hline
\cellcolor{lightgray}Precondizione		& L'utente ha aperto Eclipse e ha installato il plugin.		\\
\hline
\cellcolor{lightgray}Comando			& L'utente apre la configurazione del plugin, seleziona sorgente ``Tempo'' e inserisce il valore ``18''. L'utente seleziona il progetto ``Progetto 20''  e clicca sull'opzione ``Calcola metrica per il Basic Code Change model''.	\\
\hline
\cellcolor{lightgray}Risultato atteso	& Viene salvato nel database il valore 1 relativo a questa metrica.\\
\hline
\end{tabular}
\end{table}


\clearpage

\begin{table}[ht]
\begin{tabular}{|p{3cm}|p{9cm}|}
\hline
\cellcolor{lightgray}Codice				& TC 9.23								\\
\hline
\cellcolor{lightgray}Combinazione		& fnum2ftype2fver1cnum2pclass2									\\
\hline
\cellcolor{lightgray}Precondizione		& L'utente ha aperto Eclipse e ha installato il plugin.		\\
\hline
\cellcolor{lightgray}Comando			& L'utente apre la configurazione del plugin, seleziona un'altra sorgente. L'utente seleziona il progetto ``Progetto 20''  e clicca sull'opzione ``Calcola metrica per il Basic Code Change model''.	\\
\hline
\cellcolor{lightgray}Risultato atteso	& Viene restituito un errore.\\
\hline
\end{tabular}
\end{table}

\begin{table}[ht]
\begin{tabular}{|p{3cm}|p{9cm}|}
\hline
\cellcolor{lightgray}Codice				& TC 9.24								\\
\hline
\cellcolor{lightgray}Combinazione		& fnum2ftype2fver2cnum1									\\
\hline
\cellcolor{lightgray}Precondizione		& L'utente ha aperto Eclipse e ha installato il plugin.		\\
\hline
\cellcolor{lightgray}Comando			& L'utente seleziona il progetto ``Progetto 11''  e clicca sull'opzione ``Calcola metrica per il Basic Code Change model''.	\\
\hline
\cellcolor{lightgray}Risultato atteso	& Viene restituito un errore.\\
\hline
\end{tabular}
\end{table}

\begin{table}[ht]
\begin{tabular}{|p{3cm}|p{9cm}|}
\hline
\cellcolor{lightgray}Codice				& TC 9.25								\\
\hline
\cellcolor{lightgray}Combinazione		& fnum2ftype2fver2cnum2pclass1ptype1									\\
\hline
\cellcolor{lightgray}Precondizione		& L'utente ha aperto Eclipse e ha installato il plugin.		\\
\hline
\cellcolor{lightgray}Comando			& L'utente apre la configurazione del plugin, seleziona sorgente ``Tempo'' e inserisce il valore ``abcdef''. L'utente seleziona il progetto ``Progetto 12''  e clicca sull'opzione ``Calcola metrica per il Basic Code Change model''.	\\
\hline
\cellcolor{lightgray}Risultato atteso	& Viene restituito un errore.\\
\hline
\end{tabular}
\end{table}

\begin{table}[ht]
\begin{tabular}{|p{3cm}|p{9cm}|}
\hline
\cellcolor{lightgray}Codice				& TC 9.26								\\
\hline
\cellcolor{lightgray}Combinazione		& fnum2ftype2fver2cnum2pclass1ptype2plun1									\\
\hline
\cellcolor{lightgray}Precondizione		& L'utente ha aperto Eclipse e ha installato il plugin.		\\
\hline
\cellcolor{lightgray}Comando			& L'utente apre la configurazione del plugin, seleziona sorgente ``Tempo'' e inserisce il valore ``-12''. L'utente seleziona il progetto ``Progetto 13''  e clicca sull'opzione ``Calcola metrica per il Basic Code Change model''.	\\
\hline
\cellcolor{lightgray}Risultato atteso	& Viene restituito un errore.\\
\hline
\end{tabular}
\end{table}

\begin{table}[ht]
\begin{tabular}{|p{3cm}|p{9cm}|}
\hline
\cellcolor{lightgray}Codice				& TC 9.27								\\
\hline
\cellcolor{lightgray}Combinazione		& fnum2ftype2fver2cnum2pclass1ptype2plun2									\\
\hline
\cellcolor{lightgray}Precondizione		& L'utente ha aperto Eclipse e ha installato il plugin.		\\
\hline
\cellcolor{lightgray}Comando			& L'utente apre la configurazione del plugin, seleziona sorgente ``Tempo'' e inserisce il valore ``1000000000''. L'utente seleziona il progetto ``Progetto 14''  e clicca sull'opzione ``Calcola metrica per il Basic Code Change model''.	\\
\hline
\cellcolor{lightgray}Risultato atteso	& Viene restituito un errore.\\
\hline
\end{tabular}
\end{table}

\begin{table}[ht]
\begin{tabular}{|p{3cm}|p{9cm}|}
\hline
\cellcolor{lightgray}Codice				& TC 9.28								\\
\hline
\cellcolor{lightgray}Combinazione		& fnum2ftype2fver2cnum2pclass1ptype2plun3									\\
\hline
\cellcolor{lightgray}Precondizione		& L'utente ha aperto Eclipse e ha installato il plugin.		\\
\hline
\cellcolor{lightgray}Comando			& L'utente apre la configurazione del plugin, seleziona sorgente ``Tempo'' e inserisce il valore ``18''. L'utente seleziona il progetto ``Progetto 15''  e clicca sull'opzione ``Calcola metrica per il Basic Code Change model''.	\\
\hline
\cellcolor{lightgray}Risultato atteso	& Viene salvato nel database il valore 0 relativo a questa metrica.\\
\hline
\end{tabular}
\end{table}


\begin{table}[ht]
\begin{tabular}{|p{3cm}|p{9cm}|}
\hline
\cellcolor{lightgray}Codice				& TC 9.29								\\
\hline
\cellcolor{lightgray}Combinazione		& fnum2ftype2fver2cnum2pclass2									\\
\hline
\cellcolor{lightgray}Precondizione		& L'utente ha aperto Eclipse e ha installato il plugin.		\\
\hline
\cellcolor{lightgray}Comando			& L'utente apre la configurazione del plugin, seleziona un'altra sorgente. L'utente seleziona il progetto ``Progetto 15''  e clicca sull'opzione ``Calcola metrica per il Basic Code Change model''.	\\
\hline
\cellcolor{lightgray}Risultato atteso	& Viene restituito un errore.\\
\hline
\end{tabular}
\end{table}

\clearpage

\section{Metrica per il calcolo del Extended Code Change model}

\begin{table}[ht]
\begin{tabular}{|p{3cm}|p{9cm}|}
\hline
\cellcolor{lightgray}Codice				& TC 10.1								\\
\hline
\cellcolor{lightgray}Combinazione		& fnum1cnum1									\\
\hline
\cellcolor{lightgray}Precondizione		& L'utente ha aperto Eclipse e ha installato il plugin.		\\
\hline
\cellcolor{lightgray}Comando			& L'utente seleziona il progetto ``Vuoto''  e clicca sull'opzione ``Calcola metrica Extended Code Change model''.	\\
\hline
\cellcolor{lightgray}Risultato atteso	& Viene salvato nel database il valore relativo a questa metrica.\\
\hline
\end{tabular}
\end{table}


\begin{table}[ht]
\begin{tabular}{|p{3cm}|p{9cm}|}
\hline
\cellcolor{lightgray}Codice				& TC 10.2								\\
\hline
\cellcolor{lightgray}Combinazione		& fnum1cnum2pclass1ptype1									\\
\hline
\cellcolor{lightgray}Precondizione		& L'utente ha aperto Eclipse e ha installato il plugin.		\\
\hline
\cellcolor{lightgray}Comando			& L'utente apre la configurazione del plugin, seleziona sorgente ``Tempo'' e inserisce il valore ``abcdef''. L'utente seleziona il progetto ``Vuoto Versioning 1''  e clicca sull'opzione ``Calcola metrica Extended Code Change model''.	\\
\hline
\cellcolor{lightgray}Risultato atteso	& Viene restituito un errore.\\
\hline
\end{tabular}
\end{table}


\begin{table}[ht]
\begin{tabular}{|p{3cm}|p{9cm}|}
\hline
\cellcolor{lightgray}Codice				& TC 10.3								\\
\hline
\cellcolor{lightgray}Combinazione		& fnum1cnum2pclass1ptype2plun1									\\
\hline
\cellcolor{lightgray}Precondizione		& L'utente ha aperto Eclipse e ha installato il plugin.		\\
\hline
\cellcolor{lightgray}Comando			& L'utente apre la configurazione del plugin, seleziona sorgente ``Tempo'' e inserisce il valore ``-12''. L'utente seleziona il progetto ``Vuoto Versioning 2''  e clicca sull'opzione ``Calcola metrica Extended Code Change model''.	\\
\hline
\cellcolor{lightgray}Risultato atteso	& Viene restituito un errore.\\
\hline
\end{tabular}
\end{table}


\begin{table}[ht]
\begin{tabular}{|p{3cm}|p{9cm}|}
\hline
\cellcolor{lightgray}Codice				& TC 10.4								\\
\hline
\cellcolor{lightgray}Combinazione		& fnum1cnum2pclass1ptype2plun2									\\
\hline
\cellcolor{lightgray}Precondizione		& L'utente ha aperto Eclipse e ha installato il plugin.		\\
\hline
\cellcolor{lightgray}Comando			& L'utente apre la configurazione del plugin, seleziona sorgente ``Tempo'' e inserisce il valore ``1000000000''. L'utente seleziona il progetto ``Vuoto Versioning 3''  e clicca sull'opzione ``Calcola metrica Extended Code Change model''.	\\
\hline
\cellcolor{lightgray}Risultato atteso	& Viene restituito un errore.\\
\hline
\end{tabular}
\end{table}


\begin{table}[ht]
\begin{tabular}{|p{3cm}|p{9cm}|}
\hline
\cellcolor{lightgray}Codice				& TC 10.5								\\
\hline
\cellcolor{lightgray}Combinazione		& fnum1cnum2pclass1ptype2plun4									\\
\hline
\cellcolor{lightgray}Precondizione		& L'utente ha aperto Eclipse e ha installato il plugin.		\\
\hline
\cellcolor{lightgray}Comando			& L'utente apre la configurazione del plugin, seleziona sorgente ``Tempo'' e inserisce il valore ``17''. L'utente seleziona il progetto ``Vuoto Versioning 4''  e clicca sull'opzione ``Calcola metrica Extended Code Change model''.	\\
\hline
\cellcolor{lightgray}Risultato atteso	& Viene salvato nel database il valore relativo a questa metrica.\\
\hline
\end{tabular}
\end{table}


\begin{table}[ht]
\begin{tabular}{|p{3cm}|p{9cm}|}
\hline
\cellcolor{lightgray}Codice				& TC 10.6								\\
\hline
\cellcolor{lightgray}Combinazione		& fnum1cnum2pclass2ptype1									\\
\hline
\cellcolor{lightgray}Precondizione		& L'utente ha aperto Eclipse e ha installato il plugin.		\\
\hline
\cellcolor{lightgray}Comando			& L'utente apre la configurazione del plugin, seleziona sorgente ``Cambiamenti'' e inserisce il valore ``abcdef''. L'utente seleziona il progetto ``Vuoto Versioning 5''  e clicca sull'opzione ``Calcola metrica Extended Code Change model''.	\\
\hline
\cellcolor{lightgray}Risultato atteso	& Viene restituito un errore.\\
\hline
\end{tabular}
\end{table}

\begin{table}[ht]
\begin{tabular}{|p{3cm}|p{9cm}|}
\hline
\cellcolor{lightgray}Codice				& TC 10.7								\\
\hline
\cellcolor{lightgray}Combinazione		& fnum1cnum2pclass2ptype2plun1									\\
\hline
\cellcolor{lightgray}Precondizione		& L'utente ha aperto Eclipse e ha installato il plugin.		\\
\hline
\cellcolor{lightgray}Comando			& L'utente apre la configurazione del plugin, seleziona sorgente ``Cambiamenti'' e inserisce il valore ``-12''. L'utente seleziona il progetto ``Vuoto Versioning 6''  e clicca sull'opzione ``Calcola metrica Extended Code Change model''.	\\
\hline
\cellcolor{lightgray}Risultato atteso	& Viene restituito un errore.\\
\hline
\end{tabular}
\end{table}

\begin{table}[ht]
\begin{tabular}{|p{3cm}|p{9cm}|}
\hline
\cellcolor{lightgray}Codice				& TC 10.8								\\
\hline
\cellcolor{lightgray}Combinazione		& fnum1cnum2pclass2ptype2plun3									\\
\hline
\cellcolor{lightgray}Precondizione		& L'utente ha aperto Eclipse e ha installato il plugin.		\\
\hline
\cellcolor{lightgray}Comando			& L'utente apre la configurazione del plugin, seleziona sorgente ``Cambiamenti'' e inserisce il valore ``1000000000''. L'utente seleziona il progetto ``Vuoto Versioning 1''  e clicca sull'opzione ``Calcola metrica Extended Code Change model''.	\\
\hline
\cellcolor{lightgray}Risultato atteso	& Viene restituito un errore.\\
\hline
\end{tabular}
\end{table}

\begin{table}[ht]
\begin{tabular}{|p{3cm}|p{9cm}|}
\hline
\cellcolor{lightgray}Codice				& TC 10.9								\\
\hline
\cellcolor{lightgray}Combinazione		& fnum1cnum2pclass2ptype2plun4									\\
\hline
\cellcolor{lightgray}Precondizione		& L'utente ha aperto Eclipse e ha installato il plugin.		\\
\hline
\cellcolor{lightgray}Comando			& L'utente apre la configurazione del plugin, seleziona sorgente ``Cambiamenti'' e inserisce il valore ``17''. L'utente seleziona il progetto ``Vuoto Versioning 2''  e clicca sull'opzione ``Calcola metrica Extended Code Change model''.	\\
\hline
\cellcolor{lightgray}Risultato atteso	& Viene salvato nel database il valore relativo a questa metrica.\\
\hline
\end{tabular}
\end{table}

\begin{table}[ht]
\begin{tabular}{|p{3cm}|p{9cm}|}
\hline
\cellcolor{lightgray}Codice				& TC 10.10								\\
\hline
\cellcolor{lightgray}Combinazione		& fnum1cnum2pclass3									\\
\hline
\cellcolor{lightgray}Precondizione		& L'utente ha aperto Eclipse e ha installato il plugin.		\\
\hline
\cellcolor{lightgray}Comando			& L'utente apre la configurazione del plugin, seleziona sorgente ``Burst''. L'utente seleziona il progetto ``Vuoto Versioning 3''  e clicca sull'opzione ``Calcola metrica Extended Code Change model''.	\\
\hline
\cellcolor{lightgray}Risultato atteso	& Viene salvato nel database il valore relativo a questa metrica.\\
\hline
\end{tabular}
\end{table}

\begin{table}[ht]
\begin{tabular}{|p{3cm}|p{9cm}|}
\hline
\cellcolor{lightgray}Codice				& TC 10.11								\\
\hline
\cellcolor{lightgray}Combinazione		& fnum1cnum2pclass4									\\
\hline
\cellcolor{lightgray}Precondizione		& L'utente ha aperto Eclipse e ha installato il plugin.		\\
\hline
\cellcolor{lightgray}Comando			& L'utente apre la configurazione del plugin, seleziona un'altra sorgente. L'utente seleziona il progetto ``Vuoto Versioning 4''  e clicca sull'opzione ``Calcola metrica Extended Code Change model''.	\\
\hline
\cellcolor{lightgray}Risultato atteso	& Viene restituito un errore.\\
\hline
\end{tabular}
\end{table}

\begin{table}[ht]
\begin{tabular}{|p{3cm}|p{9cm}|}
\hline
\cellcolor{lightgray}Codice				& TC 10.12								\\
\hline
\cellcolor{lightgray}Combinazione		& fnum2ftype1fver1cnum2pclass1ptype1									\\
\hline
\cellcolor{lightgray}Precondizione		& L'utente ha aperto Eclipse e ha installato il plugin.		\\
\hline
\cellcolor{lightgray}Comando			& L'utente apre la configurazione del plugin, seleziona sorgente ``Tempo'' e inserisce il valore ``abcdef''. L'utente seleziona il progetto ``Progetto 7''  e clicca sull'opzione ``Calcola metrica Extended Code Change model''.	\\
\hline
\cellcolor{lightgray}Risultato atteso	& Viene restituito un errore.\\
\hline
\end{tabular}
\end{table}

\clearpage

\begin{table}[ht]
\begin{tabular}{|p{3cm}|p{9cm}|}
\hline
\cellcolor{lightgray}Codice				& TC 10.13								\\
\hline
\cellcolor{lightgray}Combinazione		& fnum2ftype1fver1cnum2pclass1ptype2plun1									\\
\hline
\cellcolor{lightgray}Precondizione		& L'utente ha aperto Eclipse e ha installato il plugin.		\\
\hline
\cellcolor{lightgray}Comando			& L'utente apre la configurazione del plugin, seleziona sorgente ``Tempo'' e inserisce il valore ``-12''. L'utente seleziona il progetto ``Progetto 8''  e clicca sull'opzione ``Calcola metrica Extended Code Change model''.	\\
\hline
\cellcolor{lightgray}Risultato atteso	& Viene restituito un errore.\\
\hline
\end{tabular}
\end{table}

\begin{table}[ht]
\begin{tabular}{|p{3cm}|p{9cm}|}
\hline
\cellcolor{lightgray}Codice				& TC 10.14								\\
\hline
\cellcolor{lightgray}Combinazione		& fnum2ftype1fver1cnum2pclass1ptype2plun2									\\
\hline
\cellcolor{lightgray}Precondizione		& L'utente ha aperto Eclipse e ha installato il plugin.		\\
\hline
\cellcolor{lightgray}Comando			& L'utente apre la configurazione del plugin, seleziona sorgente ``Tempo'' e inserisce il valore ``1000000000''. L'utente seleziona il progetto ``Progetto 9''  e clicca sull'opzione ``Calcola metrica Extended Code Change model''.	\\
\hline
\cellcolor{lightgray}Risultato atteso	& Viene restituito un errore.\\
\hline
\end{tabular}
\end{table}

\begin{table}[ht]
\begin{tabular}{|p{3cm}|p{9cm}|}
\hline
\cellcolor{lightgray}Codice				& TC 10.15								\\
\hline
\cellcolor{lightgray}Combinazione		& fnum2ftype1fver1cnum2pclass1ptype2plun4									\\
\hline
\cellcolor{lightgray}Precondizione		& L'utente ha aperto Eclipse e ha installato il plugin.		\\
\hline
\cellcolor{lightgray}Comando			& L'utente apre la configurazione del plugin, seleziona sorgente ``Tempo'' e inserisce il valore ``17''. L'utente seleziona il progetto ``Progetto 10''  e clicca sull'opzione ``Calcola metrica Extended Code Change model''.	\\
\hline
\cellcolor{lightgray}Risultato atteso	& Viene salvato nel database il valore relativo a questa metrica.\\
\hline
\end{tabular}
\end{table}

\begin{table}[ht]
\begin{tabular}{|p{3cm}|p{9cm}|}
\hline
\cellcolor{lightgray}Codice				& TC 10.16								\\
\hline
\cellcolor{lightgray}Combinazione		& fnum2ftype1fver1cnum2pclass2ptype1									\\
\hline
\cellcolor{lightgray}Precondizione		& L'utente ha aperto Eclipse e ha installato il plugin.		\\
\hline
\cellcolor{lightgray}Comando			& L'utente apre la configurazione del plugin, seleziona sorgente ``Cambiamenti'' e inserisce il valore ``abcdef''. L'utente seleziona il progetto ``Progetto 21''  e clicca sull'opzione ``Calcola metrica Extended Code Change model''.	\\
\hline
\cellcolor{lightgray}Risultato atteso	& Viene restituito un errore.\\
\hline
\end{tabular}
\end{table}

\begin{table}[ht]
\begin{tabular}{|p{3cm}|p{9cm}|}
\hline
\cellcolor{lightgray}Codice				& TC 10.17								\\
\hline
\cellcolor{lightgray}Combinazione		& fnum2ftype1fver1cnum2pclass2ptype2plun1									\\
\hline
\cellcolor{lightgray}Precondizione		& L'utente ha aperto Eclipse e ha installato il plugin.		\\
\hline
\cellcolor{lightgray}Comando			& L'utente apre la configurazione del plugin, seleziona sorgente ``Cambiamenti'' e inserisce il valore ``-12''. L'utente seleziona il progetto ``Progetto 22''  e clicca sull'opzione ``Calcola metrica Extended Code Change model''.	\\
\hline
\cellcolor{lightgray}Risultato atteso	& Viene restituito un errore.\\
\hline
\end{tabular}
\end{table}

\begin{table}[ht]
\begin{tabular}{|p{3cm}|p{9cm}|}
\hline
\cellcolor{lightgray}Codice				& TC 10.18								\\
\hline
\cellcolor{lightgray}Combinazione		& fnum2ftype1fver1cnum2pclass2ptype2plun2									\\
\hline
\cellcolor{lightgray}Precondizione		& L'utente ha aperto Eclipse e ha installato il plugin.		\\
\hline
\cellcolor{lightgray}Comando			& L'utente apre la configurazione del plugin, seleziona sorgente ``Cambiamenti'' e inserisce il valore ``1000000000''. L'utente seleziona il progetto ``Progetto 23''  e clicca sull'opzione ``Calcola metrica Extended Code Change model''.	\\
\hline
\cellcolor{lightgray}Risultato atteso	& Viene restituito un errore.\\
\hline
\end{tabular}
\end{table}

\begin{table}[ht]
\begin{tabular}{|p{3cm}|p{9cm}|}
\hline
\cellcolor{lightgray}Codice				& TC 10.19								\\
\hline
\cellcolor{lightgray}Combinazione		& fnum2ftype1fver1cnum2pclass2ptype2plun4									\\
\hline
\cellcolor{lightgray}Precondizione		& L'utente ha aperto Eclipse e ha installato il plugin.		\\
\hline
\cellcolor{lightgray}Comando			& L'utente apre la configurazione del plugin, seleziona sorgente ``Cambiamenti'' e inserisce il valore ``17''. L'utente seleziona il progetto ``Progetto 24''  e clicca sull'opzione ``Calcola metrica Extended Code Change model''.	\\
\hline
\cellcolor{lightgray}Risultato atteso	& Viene salvato nel database il valore relativo a questa metrica.\\
\hline
\end{tabular}
\end{table}

\begin{table}[ht]
\begin{tabular}{|p{3cm}|p{9cm}|}
\hline
\cellcolor{lightgray}Codice				& TC 10.20								\\
\hline
\cellcolor{lightgray}Combinazione		& fnum2ftype1fver1cnum2pclass3									\\
\hline
\cellcolor{lightgray}Precondizione		& L'utente ha aperto Eclipse e ha installato il plugin.		\\
\hline
\cellcolor{lightgray}Comando			& L'utente apre la configurazione del plugin, seleziona sorgente ``Burst''. L'utente seleziona il progetto ``Progetto 25''  e clicca sull'opzione ``Calcola metrica Extended Code Change model''.	\\
\hline
\cellcolor{lightgray}Risultato atteso	& Viene salvato nel database il valore relativo a questa metrica.\\
\hline
\end{tabular}
\end{table}

\begin{table}[ht]
\begin{tabular}{|p{3cm}|p{9cm}|}
\hline
\cellcolor{lightgray}Codice				& TC 10.21								\\
\hline
\cellcolor{lightgray}Combinazione		& fnum2ftype1fver1cnum2pclass4									\\
\hline
\cellcolor{lightgray}Precondizione		& L'utente ha aperto Eclipse e ha installato il plugin.		\\
\hline
\cellcolor{lightgray}Comando			& L'utente apre la configurazione del plugin, seleziona un'altra sorgente. L'utente seleziona il progetto ``Progetto 26''  e clicca sull'opzione ``Calcola metrica Extended Code Change model''.	\\
\hline
\cellcolor{lightgray}Risultato atteso	& Viene restituito un errore.\\
\hline
\end{tabular}
\end{table}

\clearpage

\begin{table}[ht]
\begin{tabular}{|p{3cm}|p{9cm}|}
\hline
\cellcolor{lightgray}Codice				& TC 10.22								\\
\hline
\cellcolor{lightgray}Combinazione		& fnum2ftype1fver2cnum1									\\
\hline
\cellcolor{lightgray}Precondizione		& L'utente ha aperto Eclipse e ha installato il plugin.		\\
\hline
\cellcolor{lightgray}Comando			& L'utente seleziona il progetto ``Progetto 1''  e clicca sull'opzione ``Calcola metrica Extended Code Change model''.	\\
\hline
\cellcolor{lightgray}Risultato atteso	& Viene salvato nel database il valore relativo a questa metrica.\\
\hline
\end{tabular}
\end{table}


\begin{table}[ht]
\begin{tabular}{|p{3cm}|p{9cm}|}
\hline
\cellcolor{lightgray}Codice				& TC 10.23								\\
\hline
\cellcolor{lightgray}Combinazione		& fnum2ftype1fver2cnum2pclass1ptype1									\\
\hline
\cellcolor{lightgray}Precondizione		& L'utente ha aperto Eclipse e ha installato il plugin.		\\
\hline
\cellcolor{lightgray}Comando			& L'utente apre la configurazione del plugin, seleziona sorgente ``Tempo'' e inserisce il valore ``abcdef''. L'utente seleziona il progetto ``Progetto 2''  e clicca sull'opzione ``Calcola metrica Extended Code Change model''.	\\
\hline
\cellcolor{lightgray}Risultato atteso	& Viene restituito un errore.\\
\hline
\end{tabular}
\end{table}


\begin{table}[ht]
\begin{tabular}{|p{3cm}|p{9cm}|}
\hline
\cellcolor{lightgray}Codice				& TC 10.24								\\
\hline
\cellcolor{lightgray}Combinazione		& fnum2ftype1fver2cnum2pclass1ptype2plun1									\\
\hline
\cellcolor{lightgray}Precondizione		& L'utente ha aperto Eclipse e ha installato il plugin.		\\
\hline
\cellcolor{lightgray}Comando			& L'utente apre la configurazione del plugin, seleziona sorgente ``Tempo'' e inserisce il valore ``-12''. L'utente seleziona il progetto ``Progetto 3''  e clicca sull'opzione ``Calcola metrica Extended Code Change model''.	\\
\hline
\cellcolor{lightgray}Risultato atteso	& Viene restituito un errore.\\
\hline
\end{tabular}
\end{table}


\begin{table}[ht]
\begin{tabular}{|p{3cm}|p{9cm}|}
\hline
\cellcolor{lightgray}Codice				& TC 10.25								\\
\hline
\cellcolor{lightgray}Combinazione		& fnum2ftype1fver2cnum2pclass1ptype2plun2									\\
\hline
\cellcolor{lightgray}Precondizione		& L'utente ha aperto Eclipse e ha installato il plugin.		\\
\hline
\cellcolor{lightgray}Comando			& L'utente apre la configurazione del plugin, seleziona sorgente ``Tempo'' e inserisce il valore ``1000000000''. L'utente seleziona il progetto ``Progetto 4''  e clicca sull'opzione ``Calcola metrica Extended Code Change model''.	\\
\hline
\cellcolor{lightgray}Risultato atteso	& Viene restituito un errore.\\
\hline
\end{tabular}
\end{table}


\begin{table}[ht]
\begin{tabular}{|p{3cm}|p{9cm}|}
\hline
\cellcolor{lightgray}Codice				& TC 10.26								\\
\hline
\cellcolor{lightgray}Combinazione		& fnum2ftype1fver2cnum2pclass1ptype2plun4									\\
\hline
\cellcolor{lightgray}Precondizione		& L'utente ha aperto Eclipse e ha installato il plugin.		\\
\hline
\cellcolor{lightgray}Comando			& L'utente apre la configurazione del plugin, seleziona sorgente ``Tempo'' e inserisce il valore ``17''. L'utente seleziona il progetto ``Progetto 5''  e clicca sull'opzione ``Calcola metrica Extended Code Change model''.	\\
\hline
\cellcolor{lightgray}Risultato atteso	& Viene salvato nel database il valore relativo a questa metrica.\\
\hline
\end{tabular}
\end{table}


\begin{table}[ht]
\begin{tabular}{|p{3cm}|p{9cm}|}
\hline
\cellcolor{lightgray}Codice				& TC 10.27								\\
\hline
\cellcolor{lightgray}Combinazione		& fnum2ftype1fver2cnum2pclass2ptype1									\\
\hline
\cellcolor{lightgray}Precondizione		& L'utente ha aperto Eclipse e ha installato il plugin.		\\
\hline
\cellcolor{lightgray}Comando			& L'utente apre la configurazione del plugin, seleziona sorgente ``Cambiamenti'' e inserisce il valore ``abcdef''.  L'utente seleziona il progetto ``Progetto 27''  e clicca sull'opzione ``Calcola metrica Extended Code Change model''.	\\
\hline
\cellcolor{lightgray}Risultato atteso	& Viene restituito un errore.\\
\hline
\end{tabular}
\end{table}

\begin{table}[ht]
\begin{tabular}{|p{3cm}|p{9cm}|}
\hline
\cellcolor{lightgray}Codice				& TC 10.28								\\
\hline
\cellcolor{lightgray}Combinazione		& fnum2ftype1fver2cnum2pclass2ptype2plun1									\\
\hline
\cellcolor{lightgray}Precondizione		& L'utente ha aperto Eclipse e ha installato il plugin.		\\
\hline
\cellcolor{lightgray}Comando			& L'utente apre la configurazione del plugin, seleziona sorgente ``Cambiamenti'' e inserisce il valore ``-12''. L'utente seleziona il progetto ``Progetto 28''  e clicca sull'opzione ``Calcola metrica Extended Code Change model''.	\\
\hline
\cellcolor{lightgray}Risultato atteso	& Viene restituito un errore.\\
\hline
\end{tabular}
\end{table}

\begin{table}[ht]
\begin{tabular}{|p{3cm}|p{9cm}|}
\hline
\cellcolor{lightgray}Codice				& TC 10.29								\\
\hline
\cellcolor{lightgray}Combinazione		& fnum2ftype1fver2cnum2pclass2ptype2plun2									\\
\hline
\cellcolor{lightgray}Precondizione		& L'utente ha aperto Eclipse e ha installato il plugin.		\\
\hline
\cellcolor{lightgray}Comando			& L'utente apre la configurazione del plugin, seleziona sorgente ``Cambiamenti'' e inserisce il valore ``1000000000''. L'utente seleziona il progetto ``Progetto 2''  e clicca sull'opzione ``Calcola metrica Extended Code Change model''.	\\
\hline
\cellcolor{lightgray}Risultato atteso	& Viene restituito un errore.\\
\hline
\end{tabular}
\end{table}

\begin{table}[ht]
\begin{tabular}{|p{3cm}|p{9cm}|}
\hline
\cellcolor{lightgray}Codice				& TC 10.30								\\
\hline
\cellcolor{lightgray}Combinazione		& fnum2ftype1fver2cnum2pclass2ptype2plun4									\\
\hline
\cellcolor{lightgray}Precondizione		& L'utente ha aperto Eclipse e ha installato il plugin.		\\
\hline
\cellcolor{lightgray}Comando			& L'utente apre la configurazione del plugin, seleziona sorgente ``Cambiamenti'' e inserisce il valore ``17''. L'utente seleziona il progetto ``Progetto 3''  e clicca sull'opzione ``Calcola metrica Extended Code Change model''.	\\
\hline
\cellcolor{lightgray}Risultato atteso	& Viene salvato nel database il valore relativo a questa metrica.\\
\hline
\end{tabular}
\end{table}

\begin{table}[ht]
\begin{tabular}{|p{3cm}|p{9cm}|}
\hline
\cellcolor{lightgray}Codice				& TC 10.31								\\
\hline
\cellcolor{lightgray}Combinazione		& fnum2ftype1fver2cnum2pclass3									\\
\hline
\cellcolor{lightgray}Precondizione		& L'utente ha aperto Eclipse e ha installato il plugin.		\\
\hline
\cellcolor{lightgray}Comando			& L'utente apre la configurazione del plugin, seleziona sorgente ``Burst''. L'utente seleziona il progetto ``Progetto 4''  e clicca sull'opzione ``Calcola metrica Extended Code Change model''.	\\
\hline
\cellcolor{lightgray}Risultato atteso	& Viene salvato nel database il valore relativo a questa metrica.\\
\hline
\end{tabular}
\end{table}

\begin{table}[ht]
\begin{tabular}{|p{3cm}|p{9cm}|}
\hline
\cellcolor{lightgray}Codice				& TC 10.32								\\
\hline
\cellcolor{lightgray}Combinazione		& fnum2ftype1fver2cnum2pclass4									\\
\hline
\cellcolor{lightgray}Precondizione		& L'utente ha aperto Eclipse e ha installato il plugin.		\\
\hline
\cellcolor{lightgray}Comando			& L'utente apre la configurazione del plugin, seleziona un'altra sorgente. L'utente seleziona il progetto ``Progetto 5''  e clicca sull'opzione ``Calcola metrica Extended Code Change model''.	\\
\hline
\cellcolor{lightgray}Risultato atteso	& Viene restituito un errore.\\
\hline
\end{tabular}
\end{table}

\begin{table}[ht]
\begin{tabular}{|p{3cm}|p{9cm}|}
\hline
\cellcolor{lightgray}Codice				& TC 10.33								\\
\hline
\cellcolor{lightgray}Combinazione		& fnum2ftype2fver1cnum1									\\
\hline
\cellcolor{lightgray}Precondizione		& L'utente ha aperto Eclipse e ha installato il plugin.		\\
\hline
\cellcolor{lightgray}Comando			& L'utente seleziona il progetto ``Progetto 16''  e clicca sull'opzione ``Calcola metrica Extended Code Change model''.	\\
\hline
\cellcolor{lightgray}Risultato atteso	& Viene salvato nel database il valore relativo a questa metrica.\\
\hline
\end{tabular}
\end{table}

\clearpage

\begin{table}[ht]
\begin{tabular}{|p{3cm}|p{9cm}|}
\hline
\cellcolor{lightgray}Codice				& TC 10.34								\\
\hline
\cellcolor{lightgray}Combinazione		& fnum2ftype2fver1cnum2pclass1ptype1									\\
\hline
\cellcolor{lightgray}Precondizione		& L'utente ha aperto Eclipse e ha installato il plugin.		\\
\hline
\cellcolor{lightgray}Comando			& L'utente apre la configurazione del plugin, seleziona sorgente ``Tempo'' e inserisce il valore ``abcdef''. L'utente seleziona il progetto ``Progetto 17''  e clicca sull'opzione ``Calcola metrica Extended Code Change model''.	\\
\hline
\cellcolor{lightgray}Risultato atteso	& Viene restituito un errore.\\
\hline
\end{tabular}
\end{table}

\begin{table}[ht]
\begin{tabular}{|p{3cm}|p{9cm}|}
\hline
\cellcolor{lightgray}Codice				& TC 10.35								\\
\hline
\cellcolor{lightgray}Combinazione		& fnum2ftype2fver1cnum2pclass1ptype2plun1									\\
\hline
\cellcolor{lightgray}Precondizione		& L'utente ha aperto Eclipse e ha installato il plugin.		\\
\hline
\cellcolor{lightgray}Comando			& L'utente apre la configurazione del plugin, seleziona sorgente ``Tempo'' e inserisce il valore ``-12''. L'utente seleziona il progetto ``Progetto 18''  e clicca sull'opzione ``Calcola metrica Extended Code Change model''.	\\
\hline
\cellcolor{lightgray}Risultato atteso	& Viene restituito un errore.\\
\hline
\end{tabular}
\end{table}

\begin{table}[ht]
\begin{tabular}{|p{3cm}|p{9cm}|}
\hline
\cellcolor{lightgray}Codice				& TC 10.36								\\
\hline
\cellcolor{lightgray}Combinazione		& fnum2ftype2fver1cnum2pclass1ptype2plun2									\\
\hline
\cellcolor{lightgray}Precondizione		& L'utente ha aperto Eclipse e ha installato il plugin.		\\
\hline
\cellcolor{lightgray}Comando			& L'utente apre la configurazione del plugin, seleziona sorgente ``Tempo'' e inserisce il valore ``1000000000''. L'utente seleziona il progetto ``Progetto 19''  e clicca sull'opzione ``Calcola metrica Extended Code Change model''.	\\
\hline
\cellcolor{lightgray}Risultato atteso	& Viene restituito un errore.\\
\hline
\end{tabular}
\end{table}

\begin{table}[ht]
\begin{tabular}{|p{3cm}|p{9cm}|}
\hline
\cellcolor{lightgray}Codice				& TC 10.37								\\
\hline
\cellcolor{lightgray}Combinazione		& fnum2ftype2fver1cnum2pclass1ptype2plun4									\\
\hline
\cellcolor{lightgray}Precondizione		& L'utente ha aperto Eclipse e ha installato il plugin.		\\
\hline
\cellcolor{lightgray}Comando			& L'utente apre la configurazione del plugin, seleziona sorgente ``Tempo'' e inserisce il valore ``17''. L'utente seleziona il progetto ``Progetto 20''  e clicca sull'opzione ``Calcola metrica Extended Code Change model''.	\\
\hline
\cellcolor{lightgray}Risultato atteso	& Viene salvato nel database il valore relativo a questa metrica.\\
\hline
\end{tabular}
\end{table}

\begin{table}[ht]
\begin{tabular}{|p{3cm}|p{9cm}|}
\hline
\cellcolor{lightgray}Codice				& TC 10.38								\\
\hline
\cellcolor{lightgray}Combinazione		& fnum2ftype2fver1cnum2pclass2ptype1									\\
\hline
\cellcolor{lightgray}Precondizione		& L'utente ha aperto Eclipse e ha installato il plugin.		\\
\hline
\cellcolor{lightgray}Comando			& L'utente apre la configurazione del plugin, seleziona sorgente ``Cambiamenti'' e inserisce il valore ``abcdef''. L'utente seleziona il progetto ``Progetto 29''  e clicca sull'opzione ``Calcola metrica Extended Code Change model''.	\\
\hline
\cellcolor{lightgray}Risultato atteso	& Viene restituito un errore.\\
\hline
\end{tabular}
\end{table}

\begin{table}[ht]
\begin{tabular}{|p{3cm}|p{9cm}|}
\hline
\cellcolor{lightgray}Codice				& TC 10.39								\\
\hline
\cellcolor{lightgray}Combinazione		& fnum2ftype2fver1cnum2pclass2ptype2plun1									\\
\hline
\cellcolor{lightgray}Precondizione		& L'utente ha aperto Eclipse e ha installato il plugin.		\\
\hline
\cellcolor{lightgray}Comando			& L'utente apre la configurazione del plugin, seleziona sorgente ``Cambiamenti'' e inserisce il valore ``-12''. L'utente seleziona il progetto ``Progetto 30''  e clicca sull'opzione ``Calcola metrica Extended Code Change model''.	\\
\hline
\cellcolor{lightgray}Risultato atteso	& Viene restituito un errore.\\
\hline
\end{tabular}
\end{table}

\begin{table}[ht]
\begin{tabular}{|p{3cm}|p{9cm}|}
\hline
\cellcolor{lightgray}Codice				& TC 10.40								\\
\hline
\cellcolor{lightgray}Combinazione		& fnum2ftype2fver1cnum2pclass2ptype2plun2									\\
\hline
\cellcolor{lightgray}Precondizione		& L'utente ha aperto Eclipse e ha installato il plugin.		\\
\hline
\cellcolor{lightgray}Comando			& L'utente apre la configurazione del plugin, seleziona sorgente ``Cambiamenti'' e inserisce il valore ``1000000000''. L'utente seleziona il progetto ``Progetto 17''  e clicca sull'opzione ``Calcola metrica Extended Code Change model''.	\\
\hline
\cellcolor{lightgray}Risultato atteso	& Viene restituito un errore.\\
\hline
\end{tabular}
\end{table}

\begin{table}[ht]
\begin{tabular}{|p{3cm}|p{9cm}|}
\hline
\cellcolor{lightgray}Codice				& TC 10.41								\\
\hline
\cellcolor{lightgray}Combinazione		& fnum2ftype2fver1cnum2pclass2ptype2plun4									\\
\hline
\cellcolor{lightgray}Precondizione		& L'utente ha aperto Eclipse e ha installato il plugin.		\\
\hline
\cellcolor{lightgray}Comando			& L'utente apre la configurazione del plugin, seleziona sorgente ``Cambiamenti'' e inserisce il valore ``17''. L'utente seleziona il progetto ``Progetto 18''  e clicca sull'opzione ``Calcola metrica Extended Code Change model''.	\\
\hline
\cellcolor{lightgray}Risultato atteso	& Viene salvato nel database il valore relativo a questa metrica.\\
\hline
\end{tabular}
\end{table}

\begin{table}[ht]
\begin{tabular}{|p{3cm}|p{9cm}|}
\hline
\cellcolor{lightgray}Codice				& TC 10.42								\\
\hline
\cellcolor{lightgray}Combinazione		& fnum2ftype2fver1cnum2pclass3									\\
\hline
\cellcolor{lightgray}Precondizione		& L'utente ha aperto Eclipse e ha installato il plugin.		\\
\hline
\cellcolor{lightgray}Comando			& L'utente apre la configurazione del plugin, seleziona sorgente ``Burst''. L'utente seleziona il progetto ``Progetto 19''  e clicca sull'opzione ``Calcola metrica Extended Code Change model''.	\\
\hline
\cellcolor{lightgray}Risultato atteso	& Viene salvato nel database il valore relativo a questa metrica.\\
\hline
\end{tabular}
\end{table}

\clearpage

\begin{table}[ht]
\begin{tabular}{|p{3cm}|p{9cm}|}
\hline
\cellcolor{lightgray}Codice				& TC 10.43								\\
\hline
\cellcolor{lightgray}Combinazione		& fnum2ftype2fver1cnum2pclass4									\\
\hline
\cellcolor{lightgray}Precondizione		& L'utente ha aperto Eclipse e ha installato il plugin.		\\
\hline
\cellcolor{lightgray}Comando			& L'utente apre la configurazione del plugin, seleziona un'altra sorgente. L'utente seleziona il progetto ``Progetto 20''  e clicca sull'opzione ``Calcola metrica Extended Code Change model''.	\\
\hline
\cellcolor{lightgray}Risultato atteso	& Viene restituito un errore.\\
\hline
\end{tabular}
\end{table}

\begin{table}[ht]
\begin{tabular}{|p{3cm}|p{9cm}|}
\hline
\cellcolor{lightgray}Codice				& TC 10.44								\\
\hline
\cellcolor{lightgray}Combinazione		& fnum2ftype2fver2cnum1									\\
\hline
\cellcolor{lightgray}Precondizione		& L'utente ha aperto Eclipse e ha installato il plugin.		\\
\hline
\cellcolor{lightgray}Comando			& L'utente seleziona il progetto ``Progetto 11''  e clicca sull'opzione ``Calcola metrica Extended Code Change model''.	\\
\hline
\cellcolor{lightgray}Risultato atteso	& Viene salvato nel database il valore relativo a questa metrica.\\
\hline
\end{tabular}
\end{table}

\begin{table}[ht]
\begin{tabular}{|p{3cm}|p{9cm}|}
\hline
\cellcolor{lightgray}Codice				& TC 10.45								\\
\hline
\cellcolor{lightgray}Combinazione		& fnum2ftype2fver2cnum2pclass1ptype1									\\
\hline
\cellcolor{lightgray}Precondizione		& L'utente ha aperto Eclipse e ha installato il plugin.		\\
\hline
\cellcolor{lightgray}Comando			& L'utente apre la configurazione del plugin, seleziona sorgente ``Tempo'' e inserisce il valore ``abcdef''. L'utente seleziona il progetto ``Progetto 12''  e clicca sull'opzione ``Calcola metrica Extended Code Change model''.	\\
\hline
\cellcolor{lightgray}Risultato atteso	& Viene restituito un errore.\\
\hline
\end{tabular}
\end{table}

\begin{table}[ht]
\begin{tabular}{|p{3cm}|p{9cm}|}
\hline
\cellcolor{lightgray}Codice				& TC 10.46								\\
\hline
\cellcolor{lightgray}Combinazione		& fnum2ftype2fver2cnum2pclass1ptype2plun1									\\
\hline
\cellcolor{lightgray}Precondizione		& L'utente ha aperto Eclipse e ha installato il plugin.		\\
\hline
\cellcolor{lightgray}Comando			& L'utente apre la configurazione del plugin, seleziona sorgente ``Tempo'' e inserisce il valore ``-12''. L'utente seleziona il progetto ``Progetto 13''  e clicca sull'opzione ``Calcola metrica Extended Code Change model''.	\\
\hline
\cellcolor{lightgray}Risultato atteso	& Viene restituito un errore.\\
\hline
\end{tabular}
\end{table}

\begin{table}[ht]
\begin{tabular}{|p{3cm}|p{9cm}|}
\hline
\cellcolor{lightgray}Codice				& TC 10.47								\\
\hline
\cellcolor{lightgray}Combinazione		& fnum2ftype2fver2cnum2pclass1ptype2plun2									\\
\hline
\cellcolor{lightgray}Precondizione		& L'utente ha aperto Eclipse e ha installato il plugin.		\\
\hline
\cellcolor{lightgray}Comando			& L'utente apre la configurazione del plugin, seleziona sorgente ``Tempo'' e inserisce il valore ``1000000000''. L'utente seleziona il progetto ``Progetto 14''  e clicca sull'opzione ``Calcola metrica Extended Code Change model''.	\\
\hline
\cellcolor{lightgray}Risultato atteso	& Viene restituito un errore.\\
\hline
\end{tabular}
\end{table}

\begin{table}[ht]
\begin{tabular}{|p{3cm}|p{9cm}|}
\hline
\cellcolor{lightgray}Codice				& TC 10.48								\\
\hline
\cellcolor{lightgray}Combinazione		& fnum2ftype2fver2cnum2pclass1ptype2plun4									\\
\hline
\cellcolor{lightgray}Precondizione		& L'utente ha aperto Eclipse e ha installato il plugin.		\\
\hline
\cellcolor{lightgray}Comando			& L'utente apre la configurazione del plugin, seleziona sorgente ``Tempo'' e inserisce il valore ``17''. L'utente seleziona il progetto ``Progetto 15''  e clicca sull'opzione ``Calcola metrica Extended Code Change model''.	\\
\hline
\cellcolor{lightgray}Risultato atteso	& Viene salvato nel database il valore relativo a questa metrica.\\
\hline
\end{tabular}
\end{table}

\begin{table}[ht]
\begin{tabular}{|p{3cm}|p{9cm}|}
\hline
\cellcolor{lightgray}Codice				& TC 10.49								\\
\hline
\cellcolor{lightgray}Combinazione		& fnum2ftype2fver2cnum2pclass2ptype1									\\
\hline
\cellcolor{lightgray}Precondizione		& L'utente ha aperto Eclipse e ha installato il plugin.		\\
\hline
\cellcolor{lightgray}Comando			& L'utente apre la configurazione del plugin, seleziona sorgente ``Cambiamenti'' e inserisce il valore ``abcdef''. L'utente seleziona il progetto ``Progetto 31''  e clicca sull'opzione ``Calcola metrica Extended Code Change model''.	\\
\hline
\cellcolor{lightgray}Risultato atteso	& Viene restituito un errore.\\
\hline
\end{tabular}
\end{table}

\begin{table}[ht]
\begin{tabular}{|p{3cm}|p{9cm}|}
\hline
\cellcolor{lightgray}Codice				& TC 10.50								\\
\hline
\cellcolor{lightgray}Combinazione		& fnum2ftype2fver2cnum2pclass2ptype2plun1									\\
\hline
\cellcolor{lightgray}Precondizione		& L'utente ha aperto Eclipse e ha installato il plugin.		\\
\hline
\cellcolor{lightgray}Comando			& L'utente apre la configurazione del plugin, seleziona sorgente ``Cambiamenti'' e inserisce il valore ``-12''. L'utente seleziona il progetto ``Progetto 32''  e clicca sull'opzione ``Calcola metrica Extended Code Change model''.	\\
\hline
\cellcolor{lightgray}Risultato atteso	& Viene restituito un errore.\\
\hline
\end{tabular}
\end{table}

\begin{table}[ht]
\begin{tabular}{|p{3cm}|p{9cm}|}
\hline
\cellcolor{lightgray}Codice				& TC 10.51								\\
\hline
\cellcolor{lightgray}Combinazione		& fnum2ftype2fver2cnum2pclass2ptype2plun2									\\
\hline
\cellcolor{lightgray}Precondizione		& L'utente ha aperto Eclipse e ha installato il plugin.		\\
\hline
\cellcolor{lightgray}Comando			& L'utente apre la configurazione del plugin, seleziona sorgente ``Cambiamenti'' e inserisce il valore ``1000000000''. L'utente seleziona il progetto ``Progetto 12''  e clicca sull'opzione ``Calcola metrica Extended Code Change model''.	\\
\hline
\cellcolor{lightgray}Risultato atteso	& Viene restituito un errore.\\
\hline
\end{tabular}
\end{table}

\begin{table}[ht]
\begin{tabular}{|p{3cm}|p{9cm}|}
\hline
\cellcolor{lightgray}Codice				& TC 10.52								\\
\hline
\cellcolor{lightgray}Combinazione		& fnum2ftype2fver2cnum2pclass2ptype2plun4									\\
\hline
\cellcolor{lightgray}Precondizione		& L'utente ha aperto Eclipse e ha installato il plugin.		\\
\hline
\cellcolor{lightgray}Comando			& L'utente apre la configurazione del plugin, seleziona sorgente ``Cambiamenti'' e inserisce il valore ``17''. L'utente seleziona il progetto ``Progetto 13''  e clicca sull'opzione ``Calcola metrica Extended Code Change model''.	\\
\hline
\cellcolor{lightgray}Risultato atteso	& Viene salvato nel database il valore relativo a questa metrica.\\
\hline
\end{tabular}
\end{table}

\begin{table}[ht]
\begin{tabular}{|p{3cm}|p{9cm}|}
\hline
\cellcolor{lightgray}Codice				& TC 10.53								\\
\hline
\cellcolor{lightgray}Combinazione		& fnum2ftype2fver2cnum2pclass3									\\
\hline
\cellcolor{lightgray}Precondizione		& L'utente ha aperto Eclipse e ha installato il plugin.		\\
\hline
\cellcolor{lightgray}Comando			& L'utente apre la configurazione del plugin, seleziona sorgente ``Burst''. L'utente seleziona il progetto ``Progetto 14''  e clicca sull'opzione ``Calcola metrica Extended Code Change model''.	\\
\hline
\cellcolor{lightgray}Risultato atteso	& Viene restituito un errore.\\
\hline
\end{tabular}
\end{table}

\begin{table}[ht]
\begin{tabular}{|p{3cm}|p{9cm}|}
\hline
\cellcolor{lightgray}Codice				& TC 10.54								\\
\hline
\cellcolor{lightgray}Combinazione		& fnum2ftype2fver2cnum2pclass4									\\
\hline
\cellcolor{lightgray}Precondizione		& L'utente ha aperto Eclipse e ha installato il plugin.		\\
\hline
\cellcolor{lightgray}Comando			& L'utente apre la configurazione del plugin, seleziona un'altra sorgente. L'utente seleziona il progetto ``Progetto 15''  e clicca sull'opzione ``Calcola metrica Extended Code Change model''.	\\
\hline
\cellcolor{lightgray}Risultato atteso	& Viene restituito un errore.\\
\hline
\end{tabular}
\end{table}

\clearpage
