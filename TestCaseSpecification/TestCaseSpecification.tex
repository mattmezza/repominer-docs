\chapter{Specifica dei casi di test}

%\section{Tabelle bavota style}
%		\newcolumntype{C}[1]{>{\centering}p{#1}}
%		\begin{table}[ht]
%		\centering
%			\begin{tabular}{|p{3cm}|p{9cm}|}
%				\hline
%				\textbf{Parametro} 			& pInstanceVariable\tabularnewline
%				\hline
%				\textbf{Categoria} 			& \textbf{Scelta}\tabularnewline
%				\hline
%				\multirow{2}{*}{valore vcp} & 1: se valore = null [errore] \tabularnewline
%				\cline{2-2}
%											& 2: se valore != null [property valoreVPIVok]\tabularnewline
%				\hline
%			\end{tabular}
%		\end {table}
%		
%		\begin{table}[ht]
%		\centering
%			\begin{tabular}{|p{4cm}|p{4cm}|p{4cm}|}
%				\hline
%				\textbf{Codice} & \textbf{Combinazione} & \textbf{Esito}\tabularnewline
%				\hline
%				TC 0.01 		& vpiv1 				& errore \tabularnewline
%				\hline
%				TC 0.02 		& vpiv1vcp1				& errore \tabularnewline
%				\hline
%				TC 0.03 		& vpiv2vcp2vcb1 		& errore \tabularnewline
%				\hline
%				TC 0.04 		& vpiv2vcp2vcb2 		& parsing \tabularnewline
%				\hline
%			\end{tabular}
%		\label{Test Case Parsing Field}
%		\end {table}
%		
%\clearpage
		
\section{Tabelle oliveto slides style}
		\begin{table}[ht]
		\centering
			\begin{tabular}{|p{4cm}|p{4cm}|p{4cm}|}
				\hline
				\textbf{Parametri} & \textbf{Categorie} & \textbf{Scelte} \tabularnewline
				\hline
				\multirow{2}{*}{Intero x}		& In-range 				& 1, 2-19, 20	\tabularnewline
				\cline{2-3}
												& Out-of-range			& 0, 21			\tabularnewline
				\hline
				\multirow{3}{*}{Stringa a}		& Minimal length 		& 1 			\tabularnewline
				\cline{2-3}
												& Intermediate length	& 2-19			\tabularnewline
				\cline{2-3}
												& Maximal length		& 20			\tabularnewline
				\hline
				\multirow{2}{*}{Carattere c}	& Beginning				& First			\tabularnewline
				\cline{2-3}
												& Middle				& Middle		\tabularnewline
				\cline{2-3}
												& End					& Last			\tabularnewline
				\cline{2-3}
												& Not occur 			& Not occur		\tabularnewline
				\hline
			\end{tabular}
		\end {table}

\vspace{1cm}

\begin{tabular}{lll}
\multicolumn{3}{l}{ \textbf{Specifica test formale} }						\\
x		&								&									\\
1.		&	0							&	[error]							\\
2.		&	1							&	[property stringok, length1]	\\
3.		&	2-19						&	[property stringok, midlength]	\\
4.		&	20							&	[property stringok, length20]	\\
5.		&	21							&	[error]							\\
		&								&									\\
a		&								&									\\
1.		&	Length 1					&	[if stringok and length1]		\\
2.		&	Length 2-19 				&	[if stringok and midlength]		\\
3.		&	Length 20					&	[if stringok and length20]		\\
		&								&									\\
c		&								&									\\
1.		&	At first position in string	&	[if stringok]					\\
2.		&	At last position in string	&	[if stringok and not length1]	\\
3.		&	In middle of string			&	[if stringok and not length1]	\\
4.		&	Not in string				&	[if stringok]					\\
\end{tabular}

\vspace{1cm}

\begin{tabular}{p{2cm}l}
\multicolumn{2}{l}{ \textbf{Test frames \& test cases} }	\\
x1		&	x = 0											\\
x2a1c1	&	x = 1, a = ‘A’, c = ‘A’							\\
x2a1c4	&	x = 1, a = ‘A’, c = ‘B’							\\
x3a2c1	&	x = 7, a = ‘ABCDEFG’, c = ‘A’					\\
x3a2c2	&	x = 7, a = ‘ABCDEFG’, c = ‘G’					\\
x3a2c3	&	x = 7, a = ‘ABCDEFG’, c = ‘D’					\\
x3a2c4	&	x = 7, a = ‘ABCDEFG’, c = ‘X’					\\
x4a3c1	&	x = 20, a = ‘ABCDEFGHIJKLMNOPQRST’, c = ‘A’		\\
x4a3c2	&	x = 20, a = ‘ABCDEFGHIJKLMNOPQRST’, c = ‘T’		\\
x4a3c3	&	x = 20, a = ‘ABCDEFGHIJKLMNOPQRST’, c = ‘J’		\\
x4a3c4	&	x = 20, a = ‘ABCDEFGHIJKLMNOPQRST’, c = ‘X’		\\
x5		&	x = 21											\\
\end{tabular}

\clearpage
		
\section{Tabelle category partition paper style}

\begin{tabular}{lll}
\multicolumn{3}{l}{ \textbf{\# Test specification for find command} }								\\
\multicolumn{3}{l}{ }																				\\
\textbf{Parameters:}	& 										&									\\
Pattern size:			&										&									\\
						&	empty								&	[property Empty					\\
						&	single character					&	[property NonEmpty]				\\
						&	many character						&	[property NonEmpty]				\\
						&	longer than any line in the file	&	[error]							\\
Quoting:				&										&									\\
						&	pattern is quoted					&	[property Quoted]				\\
						&	pattern is not quoted				&	[if NonEmpty]					\\
						&	pattern is improperly quoted		&	[error]							\\
Embedded blanks:		&										&									\\
						&	no embedded blank					&	[if NonEmpty]					\\
						&	one embedded blank					&	[if NonEmpty and Quoted]		\\
						&	several embedded blanks				&	[if NonEmpty and Quoted]		\\
Embedded quotes:		&										&									\\
						&	no embedded quote					&	[if NonEmpty]					\\
						&	one embedded quote					&	[if NonEmpty]					\\
						&	several embedded quotes				&	[if NonEmpty][single]			\\
File name:				&										&									\\
						&	good file name						&									\\
						&	no file with this name				&	[error]							\\
						&	omitted								&	[error]							\\
\end{tabular}

\vspace{1cm}

\begin{tabular}{ll}
\hline
\multicolumn{2}{l}{ \textbf{Test Frame:} }			\\
\hline
Test Case 28:		&	{Key = 3.1.3.2.1.}			\\
Pattern size:		&	many character				\\
Quoting:			&	pattern is quoted			\\
Embedded blanks:	&	several embedded blanks		\\
Embedded quotes:	&	one embedded quotes			\\
File name:			&	good file name				\\
\multicolumn{2}{l}{ }											\\
\multicolumn{2}{l}{ \textbf{find} command to perform this test: }		\\
\multicolumn{2}{l}{ find "has "" one quote" testfile }			\\
\multicolumn{2}{l}{ }											\\
\multicolumn{2}{l}{ Instructions for checking the test: }		\\
\multicolumn{2}{l}{ The following line should be displayed: }	\\
\multicolumn{2}{l}{ This line has " one quote on it }			\\
\hline
\end{tabular}

\clearpage

\subsection{Calcolo numero di revisioni del sistema}

\begin{tabular}{lp{6cm}l}
\multicolumn{3}{l}{ \textbf{\# Specifica test per il calcolo del numero di revisioni del sistema } }	\\
\multicolumn{3}{l}{ }																\\
\textbf{Parametri:}		& 										&					\\
Nome progetto:			&										&					\\
						&	nome progetto corretto				&					\\
						&	nome progetto errato				&	[error]			\\
						&	omesso								&	[error]			\\

Metrica:				&										&					\\
						&	nome metrica corretto				&					\\
						&	nome metrica errato					&	[error]			\\
						&	omesso								&	[error]			\\
\end{tabular}

\vspace{1cm}

\begin{tabular}{ll}
\hline
\multicolumn{2}{l}{ \textbf{Test Frame:} }							\\
\hline
Test Case 1:		&	{Key = 1.1.}								\\
Nome progetto:		&	nome progetto corretto						\\
Metrica:			&	nome metrica corretto						\\
\multicolumn{2}{l}{ }												\\
\multicolumn{2}{l}{ Comando per eseguire il test: }					\\
\multicolumn{2}{l}{ calculate metric\_name project\_name }			\\
\multicolumn{2}{l}{ }												\\
\multicolumn{2}{l}{ Istruzioni per la verifica del test: }			\\
\multicolumn{2}{l}{ Dovrebbe apparire il seguente output: }			\\
\multicolumn{2}{l}{ metric\_name of project\_name is value }		\\
\hline
\end{tabular}

\clearpage

\subsection{Calcolo Extended Change Model}

\begin{tabular}{lp{6cm}l}
\multicolumn{3}{l}{ \textbf{\# Specifica test per il calcolo dell'Extended Change Model} }		\\
\multicolumn{3}{l}{ }																			\\
\textbf{Parametri:}		& 													&					\\
Periodo	:				&													&					\\
						&	vuoto											&	[error]			\\
						&	un mese											&					\\
						&	più mesi										&					\\
						&	periodo più lungo dell'esistenza del sistema	&	[error]			\\
Numero di cambiamenti:	&													&					\\
						&	nessun cambiamento								&	[error]			\\
						&	un cambiamento									&					\\
						&	più cambiamenti									&					\\
Periodo di burst:		&													&					\\
						&	scelta1											&					\\
						&	scelta2											&					\\
						&	scelta3											&					\\
\end{tabular}

\vspace{1cm}