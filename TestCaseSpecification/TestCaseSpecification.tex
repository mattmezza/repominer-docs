\chapter{Test Case}

\section{Metrica per il calcolo del numero di revisioni del sistema}

\begin{table}[ht]
\begin{tabular}{|p{3cm}|p{9cm}|}
\hline
\cellcolor{lightgray}Codice				& TC 1.1								\\
\hline
\cellcolor{lightgray}Combinazione		& cnum1									\\
\hline
\cellcolor{lightgray}Precondizione		& L'utente ha aperto Eclipse e ha installato il plugin.		\\
\hline
\cellcolor{lightgray}Comando			& L'utente seleziona l'opzione ``Calcola metrica per il numero medio di revisioni del sistema''.	\\
\hline
\cellcolor{lightgray}Risultato atteso	& Viene salvato nel database il valore relativo a questa metrica.\\
\hline
\end{tabular}
\end{table}

\begin{table}[ht]
\begin{tabular}{|p{3cm}|p{9cm}|}
\hline
\cellcolor{lightgray}Codice				& TC 1.2								\\
\hline
\cellcolor{lightgray}Combinazione		& cnum2									\\
\hline
\cellcolor{lightgray}Precondizione		& L'utente ha aperto Eclipse e ha installato il plugin.		\\
\hline
\cellcolor{lightgray}Comando			& L'utente seleziona l'opzione ``Calcola metrica per il numero medio di revisioni del sistema''.	\\
\hline
\cellcolor{lightgray}Risultato atteso	& Viene salvato nel database il valore relativo a questa metrica.\\
\hline
\end{tabular}
\end{table}
\clearpage

\section{Metrica per il calcolo del numero medio di volte in cui i file di un package hanno subito cambiamenti}

\begin{table}[ht]
\begin{tabular}{|p{3cm}|p{9cm}|}
\hline
\cellcolor{lightgray}Codice				& TC 2.1								\\
\hline
\cellcolor{lightgray}Combinazione		& fnum1cnum1								\\
\hline
\cellcolor{lightgray}Precondizione		& L'utente ha aperto Eclipse e ha installato il plugin.		\\
\hline
\cellcolor{lightgray}Comando			& L'utente seleziona l'opzione ``Calcola metrica per il numero medio di volte in cui i file di un package hanno subito cambiamenti''.	\\
\hline
\cellcolor{lightgray}Risultato atteso	& Viene salvato nel database il valore relativo a questa metrica.\\
\hline
\end{tabular}
\end{table}

\begin{table}[ht]
\begin{tabular}{|p{3cm}|p{9cm}|}
\hline
\cellcolor{lightgray}Codice				& TC 2.2								\\
\hline
\cellcolor{lightgray}Combinazione		& fnum1cnum2								\\
\hline
\cellcolor{lightgray}Precondizione		& L'utente ha aperto Eclipse e ha installato il plugin.		\\
\hline
\cellcolor{lightgray}Comando			& L'utente seleziona l'opzione ``Calcola metrica per il numero medio di volte in cui i file di un package hanno subito cambiamenti''.	\\
\hline
\cellcolor{lightgray}Risultato atteso	& Viene salvato nel database il valore relativo a questa metrica.\\
\hline
\end{tabular}
\end{table}

\begin{table}[ht]
\begin{tabular}{|p{3cm}|p{9cm}|}
\hline
\cellcolor{lightgray}Codice				& TC 2.3								\\
\hline
\cellcolor{lightgray}Combinazione		& fnum2ftype1fver1cnum1									\\
\hline
\cellcolor{lightgray}Precondizione		& L'utente ha aperto Eclipse e ha installato il plugin.		\\
\hline
\cellcolor{lightgray}Comando			& L'utente seleziona l'opzione ``Calcola metrica per il numero medio di volte in cui i file di un package hanno subito cambiamenti''.	\\
\hline
\cellcolor{lightgray}Risultato atteso	& Viene salvato nel database il valore relativo a questa metrica.\\
\hline
\end{tabular}
\end{table}

\begin{table}[ht]
\begin{tabular}{|p{3cm}|p{9cm}|}
\hline
\cellcolor{lightgray}Codice				& TC 2.4								\\
\hline
\cellcolor{lightgray}Combinazione		& fnum2ftype1fver2cnum1									\\
\hline
\cellcolor{lightgray}Precondizione		& L'utente ha aperto Eclipse e ha installato il plugin.		\\
\hline
\cellcolor{lightgray}Comando			& L'utente seleziona l'opzione ``Calcola metrica per il numero medio di volte in cui i file di un package hanno subito cambiamenti''.	\\
\hline
\cellcolor{lightgray}Risultato atteso	& Viene salvato nel database il valore relativo a questa metrica.\\
\hline
\end{tabular}
\end{table}

\begin{table}[ht]
\begin{tabular}{|p{3cm}|p{9cm}|}
\hline
\cellcolor{lightgray}Codice				& TC 2.5								\\
\hline
\cellcolor{lightgray}Combinazione		& fnum2ftype2fver1cnum1								\\
\hline
\cellcolor{lightgray}Precondizione		& L'utente ha aperto Eclipse e ha installato il plugin.		\\
\hline
\cellcolor{lightgray}Comando			& L'utente seleziona l'opzione ``Calcola metrica per il numero medio di volte in cui i file di un package hanno subito cambiamenti''.	\\
\hline
\cellcolor{lightgray}Risultato atteso	& Viene salvato nel database il valore relativo a questa metrica.\\
\hline
\end{tabular}
\end{table}

\begin{table}[ht]
\begin{tabular}{|p{3cm}|p{9cm}|}
\hline
\cellcolor{lightgray}Codice				& TC 2.6								\\
\hline
\cellcolor{lightgray}Combinazione		& fnum2ftype2fver2cnum1								\\
\hline
\cellcolor{lightgray}Precondizione		& L'utente ha aperto Eclipse e ha installato il plugin.		\\
\hline
\cellcolor{lightgray}Comando			& L'utente seleziona l'opzione ``Calcola metrica per il numero medio di volte in cui i file di un package hanno subito cambiamenti''.	\\
\hline
\cellcolor{lightgray}Risultato atteso	& Viene salvato nel database il valore relativo a questa metrica.\\
\hline
\end{tabular}
\end{table}

\begin{table}[ht]
\begin{tabular}{|p{3cm}|p{9cm}|}
\hline
\cellcolor{lightgray}Codice				& TC 2.7								\\
\hline
\cellcolor{lightgray}Combinazione		& fnum2ftype1fver1cnum2									\\
\hline
\cellcolor{lightgray}Precondizione		& L'utente ha aperto Eclipse e ha installato il plugin.		\\
\hline
\cellcolor{lightgray}Comando			& L'utente seleziona l'opzione ``Calcola metrica per il numero medio di volte in cui i file di un package hanno subito cambiamenti''.	\\
\hline
\cellcolor{lightgray}Risultato atteso	& Viene salvato nel database il valore relativo a questa metrica.\\
\hline
\end{tabular}
\end{table}

\begin{table}[ht]
\begin{tabular}{|p{3cm}|p{9cm}|}
\hline
\cellcolor{lightgray}Codice				& TC 2.8								\\
\hline
\cellcolor{lightgray}Combinazione		& fnum2ftype1fver2cnum2										\\
\hline
\cellcolor{lightgray}Precondizione		& L'utente ha aperto Eclipse e ha installato il plugin.		\\
\hline
\cellcolor{lightgray}Comando			& L'utente seleziona l'opzione ``Calcola metrica per il numero medio di volte in cui i file di un package hanno subito cambiamenti''.	\\
\hline
\cellcolor{lightgray}Risultato atteso	& Viene salvato nel database il valore relativo a questa metrica.\\
\hline
\end{tabular}
\end{table}

\begin{table}[ht]
\begin{tabular}{|p{3cm}|p{9cm}|}
\hline
\cellcolor{lightgray}Codice				& TC 2.9								\\
\hline
\cellcolor{lightgray}Combinazione		& fnum2ftype2fver1cnum2									\\
\hline
\cellcolor{lightgray}Precondizione		& L'utente ha aperto Eclipse e ha installato il plugin.		\\
\hline
\cellcolor{lightgray}Comando			& L'utente seleziona l'opzione ``Calcola metrica per il numero medio di volte in cui i file di un package hanno subito cambiamenti''.	\\
\hline
\cellcolor{lightgray}Risultato atteso	& Viene salvato nel database il valore relativo a questa metrica.\\
\hline
\end{tabular}
\end{table}

\begin{table}[ht]
\begin{tabular}{|p{3cm}|p{9cm}|}
\hline
\cellcolor{lightgray}Codice				& TC 2.10								\\
\hline
\cellcolor{lightgray}Combinazione		& fnum2ftype2fver2cnum2									\\
\hline
\cellcolor{lightgray}Precondizione		& L'utente ha aperto Eclipse e ha installato il plugin.		\\
\hline
\cellcolor{lightgray}Comando			& L'utente seleziona l'opzione ``Calcola metrica per il numero medio di volte in cui i file di un package hanno subito cambiamenti''.	\\
\hline
\cellcolor{lightgray}Risultato atteso	& Viene salvato nel database il valore relativo a questa metrica.\\
\hline
\end{tabular}
\end{table}
\clearpage

\section{Metrica per il calcolo del numero medio di refactoring di un package}
\begin{table}[ht]
\begin{tabular}{|p{3cm}|p{9cm}|}
\hline
\cellcolor{lightgray}Codice				& TC 3.1								\\
\hline
\cellcolor{lightgray}Combinazione		& fnum1cnum1									\\
\hline
\cellcolor{lightgray}Precondizione		& L'utente ha aperto Eclipse e ha installato il plugin.		\\
\hline
\cellcolor{lightgray}Comando			& L'utente seleziona l'opzione ``Calcola metrica per il numero medio di refactoring di un package''.	\\
\hline
\cellcolor{lightgray}Risultato atteso	& Viene salvato nel database il valore 0 relativo a questa metrica.\\
\hline
\end{tabular}
\end{table}

\begin{table}[ht]
\begin{tabular}{|p{3cm}|p{9cm}|}
\hline
\cellcolor{lightgray}Codice				& TC 3.2								\\
\hline
\cellcolor{lightgray}Combinazione		& fnum1cnum2cmes1 									\\
\hline
\cellcolor{lightgray}Precondizione		& L'utente ha aperto Eclipse e ha installato il plugin.				\\
\hline
\cellcolor{lightgray}Comando			& L'utente seleziona l'opzione ``Calcola metrica per il numero medio di refactoring di un package''.	\\
\hline
\cellcolor{lightgray}Risultato atteso	& Viene salvato nel database il valore relativo a questa metrica	\\
\hline
\end{tabular}
\end{table}

\begin{table}[ht]
\begin{tabular}{|p{3cm}|p{9cm}|}
\hline
\cellcolor{lightgray}Codice				& TC 3.3								\\
\hline
\cellcolor{lightgray}Combinazione		& fnum1cnum2cmes2									\\
\hline
\cellcolor{lightgray}Precondizione		& L'utente ha aperto Eclipse e ha installato il plugin.					\\
\hline
\cellcolor{lightgray}Comando			& L'utente seleziona l'opzione ``Calcola metrica per il numero medio di refactoring di un package''.	\\
\hline
\cellcolor{lightgray}Risultato atteso	& Viene salvato nel database il valore relativo a questa metrica	\\
\hline
\end{tabular}
\end{table}

\begin{table}[ht]
\begin{tabular}{|p{3cm}|p{9cm}|}
\hline
\cellcolor{lightgray}Codice				& TC 3.4								\\
\hline
\cellcolor{lightgray}Combinazione		& fnum1cnum2cmes3									\\
\hline
\cellcolor{lightgray}Precondizione		& L'utente ha aperto Eclipse e ha installato il plugin.			\\
\hline
\cellcolor{lightgray}Comando			& L'utente seleziona l'opzione ``Calcola metrica per il numero medio di refactoring di un package''.	\\
\hline
\cellcolor{lightgray}Risultato atteso	& Viene salvato nel database il valore relativo a questa metrica	\\
\hline
\end{tabular}
\end{table}

\begin{table}[ht]
\begin{tabular}{|p{3cm}|p{9cm}|}
\hline
\cellcolor{lightgray}Codice				& TC 3.5								\\
\hline
\cellcolor{lightgray}Combinazione		& fnum2ftype1fver1cnum1 									\\
\hline
\cellcolor{lightgray}Precondizione		& L'utente ha aperto Eclipse e ha installato il plugin.			\\
\hline
\cellcolor{lightgray}Comando			& L'utente seleziona l'opzione ``Calcola metrica per il numero medio di refactoring di un package''.	\\
\hline
\cellcolor{lightgray}Risultato atteso	& Viene salvato nel database il valore relativo a questa metrica	\\
\hline
\end{tabular}
\end{table}

\begin{table}[ht]
\begin{tabular}{|p{3cm}|p{9cm}|}
\hline
\cellcolor{lightgray}Codice				& TC 3.6								\\
\hline
\cellcolor{lightgray}Combinazione		& fnum2ftype1fver2cnum1									\\
\hline
\cellcolor{lightgray}Precondizione		& L'utente ha aperto Eclipse e ha installato il plugin.				\\
\hline
\cellcolor{lightgray}Comando			& L'utente seleziona l'opzione ``Calcola metrica per il numero medio di refactoring di un package''.	\\
\hline
\cellcolor{lightgray}Risultato atteso	& Viene salvato nel database il valore relativo a questa metrica	\\
\hline
\end{tabular}
\end{table}

\begin{table}[ht]
\begin{tabular}{|p{3cm}|p{9cm}|}
\hline
\cellcolor{lightgray}Codice				& TC 3.7								\\
\hline
\cellcolor{lightgray}Combinazione		& fnum2ftype2fver1cnum1								\\
\hline
\cellcolor{lightgray}Precondizione		& L'utente ha aperto Eclipse e ha installato il plugin.									\\
\hline
\cellcolor{lightgray}Comando			& L'utente seleziona l'opzione ``Calcola metrica per il numero medio di refactoring di un package''.	\\
\hline
\cellcolor{lightgray}Risultato atteso	& Viene salvato nel database il valore relativo a questa metrica	\\
\hline
\end{tabular}
\end{table}

\begin{table}[ht]
\begin{tabular}{|p{3cm}|p{9cm}|}
\hline
\cellcolor{lightgray}Codice				& TC 3.8								\\
\hline
\cellcolor{lightgray}Combinazione		& fnum2ftype2fver2cnum1 									\\
\hline
\cellcolor{lightgray}Precondizione		& L'utente ha aperto Eclipse e ha installato il plugin.				\\
\hline
\cellcolor{lightgray}Comando			& L'utente seleziona l'opzione ``Calcola metrica per il numero medio di refactoring di un package''.	\\
\hline
\cellcolor{lightgray}Risultato atteso	& Viene salvato nel database il valore relativo a questa metrica	\\
\hline
\end{tabular}
\end{table}

\begin{table}[ht]
\begin{tabular}{|p{3cm}|p{9cm}|}
\hline
\cellcolor{lightgray}Codice				& TC 3.9								\\
\hline
\cellcolor{lightgray}Combinazione		& fnum2ftype1fver1cnum2cmes1 									\\
\hline
\cellcolor{lightgray}Precondizione		& L'utente ha aperto Eclipse e ha installato il plugin.								\\
\hline
\cellcolor{lightgray}Comando			& L'utente seleziona l'opzione ``Calcola metrica per il numero medio di refactoring di un package''.	\\
\hline
\cellcolor{lightgray}Risultato atteso	& Viene salvato nel database il valore relativo a questa metrica	\\
\hline
\end{tabular}
\end{table}

\begin{table}[ht]
\begin{tabular}{|p{3cm}|p{9cm}|}
\hline
\cellcolor{lightgray}Codice				& TC 3.10								\\
\hline
\cellcolor{lightgray}Combinazione		& fnum2ftype1fver2cnum2cmes1 									\\
\hline
\cellcolor{lightgray}Precondizione		& L'utente ha aperto Eclipse e ha installato il plugin.									\\
\hline
\cellcolor{lightgray}Comando			& L'utente seleziona l'opzione ``Calcola metrica per il numero medio di refactoring di un package''.	\\
\hline
\cellcolor{lightgray}Risultato atteso	& Viene salvato nel database il valore relativo a questa metrica	\\
\hline
\end{tabular}
\end{table}

\clearpage

\begin{table}[ht]
\begin{tabular}{|p{3cm}|p{9cm}|}
\hline
\cellcolor{lightgray}Codice				& TC 3.11								\\
\hline
\cellcolor{lightgray}Combinazione		& fnum2ftype2fver1cnum2cmes1 									\\
\hline
\cellcolor{lightgray}Precondizione		& L'utente ha aperto Eclipse e ha installato il plugin.		\\
\hline
\cellcolor{lightgray}Comando			& L'utente seleziona l'opzione ``Calcola metrica per il numero medio di refactoring di un package''.	\\
\hline
\cellcolor{lightgray}Risultato atteso	& Viene salvato nel database il valore 0 relativo a questa metrica.\\
\hline
\end{tabular}
\end{table}

\begin{table}[ht]
\begin{tabular}{|p{3cm}|p{9cm}|}
\hline
\cellcolor{lightgray}Codice				& TC 3.12								\\
\hline
\cellcolor{lightgray}Combinazione		& fnum2ftype2fver2cnum2cmes1 									\\
\hline
\cellcolor{lightgray}Precondizione		& L'utente ha aperto Eclipse e ha installato il plugin.				\\
\hline
\cellcolor{lightgray}Comando			& L'utente seleziona l'opzione ``Calcola metrica per il numero medio di refactoring di un package''.	\\
\hline
\cellcolor{lightgray}Risultato atteso	& Viene salvato nel database il valore relativo a questa metrica	\\
\hline
\end{tabular}
\end{table}

\begin{table}[ht]
\begin{tabular}{|p{3cm}|p{9cm}|}
\hline
\cellcolor{lightgray}Codice				& TC 3.13								\\
\hline
\cellcolor{lightgray}Combinazione		& fnum2ftype1fver1cnum2cmes2 									\\
\hline
\cellcolor{lightgray}Precondizione		& L'utente ha aperto Eclipse e ha installato il plugin.					\\
\hline
\cellcolor{lightgray}Comando			& L'utente seleziona l'opzione ``Calcola metrica per il numero medio di refactoring di un package''.	\\
\hline
\cellcolor{lightgray}Risultato atteso	& Viene salvato nel database il valore relativo a questa metrica	\\
\hline
\end{tabular}
\end{table}

\begin{table}[ht]
\begin{tabular}{|p{3cm}|p{9cm}|}
\hline
\cellcolor{lightgray}Codice				& TC 3.14								\\
\hline
\cellcolor{lightgray}Combinazione		& fnum2ftype1fver2cnum2cmes2									\\
\hline
\cellcolor{lightgray}Precondizione		& L'utente ha aperto Eclipse e ha installato il plugin.			\\
\hline
\cellcolor{lightgray}Comando			& L'utente seleziona l'opzione ``Calcola metrica per il numero medio di refactoring di un package''.	\\
\hline
\cellcolor{lightgray}Risultato atteso	& Viene salvato nel database il valore relativo a questa metrica	\\
\hline
\end{tabular}
\end{table}

\begin{table}[ht]
\begin{tabular}{|p{3cm}|p{9cm}|}
\hline
\cellcolor{lightgray}Codice				& TC 3.15								\\
\hline
\cellcolor{lightgray}Combinazione		& fnum2ftype2fver1cnum2cmes2  									\\
\hline
\cellcolor{lightgray}Precondizione		& L'utente ha aperto Eclipse e ha installato il plugin.			\\
\hline
\cellcolor{lightgray}Comando			& L'utente seleziona l'opzione ``Calcola metrica per il numero medio di refactoring di un package''.	\\
\hline
\cellcolor{lightgray}Risultato atteso	& Viene salvato nel database il valore relativo a questa metrica	\\
\hline
\end{tabular}
\end{table}

\begin{table}[ht]
\begin{tabular}{|p{3cm}|p{9cm}|}
\hline
\cellcolor{lightgray}Codice				& TC 3.16								\\
\hline
\cellcolor{lightgray}Combinazione		& fnum2ftype1fver1cnum2cmes2 									\\
\hline
\cellcolor{lightgray}Precondizione		& L'utente ha aperto Eclipse e ha installato il plugin.				\\
\hline
\cellcolor{lightgray}Comando			& L'utente seleziona l'opzione ``Calcola metrica per il numero medio di refactoring di un package''.	\\
\hline
\cellcolor{lightgray}Risultato atteso	& Viene salvato nel database il valore relativo a questa metrica	\\
\hline
\end{tabular}
\end{table}

\begin{table}[ht]
\begin{tabular}{|p{3cm}|p{9cm}|}
\hline
\cellcolor{lightgray}Codice				& TC 3.17								\\
\hline
\cellcolor{lightgray}Combinazione		& fnum2ftype1fver2cnum2cmes3 								\\
\hline
\cellcolor{lightgray}Precondizione		& L'utente ha aperto Eclipse e ha installato il plugin.									\\
\hline
\cellcolor{lightgray}Comando			& L'utente seleziona l'opzione ``Calcola metrica per il numero medio di refactoring di un package''.	\\
\hline
\cellcolor{lightgray}Risultato atteso	& Viene salvato nel database il valore relativo a questa metrica	\\
\hline
\end{tabular}
\end{table}

\begin{table}[ht]
\begin{tabular}{|p{3cm}|p{9cm}|}
\hline
\cellcolor{lightgray}Codice				& TC 3.18								\\
\hline
\cellcolor{lightgray}Combinazione		& fnum2ftype2fver1cnum2cmes3  									\\
\hline
\cellcolor{lightgray}Precondizione		& L'utente ha aperto Eclipse e ha installato il plugin.				\\
\hline
\cellcolor{lightgray}Comando			& L'utente seleziona l'opzione ``Calcola metrica per il numero medio di refactoring di un package''.	\\
\hline
\cellcolor{lightgray}Risultato atteso	& Viene salvato nel database il valore relativo a questa metrica	\\
\hline
\end{tabular}
\end{table}

\begin{table}[ht]
\begin{tabular}{|p{3cm}|p{9cm}|}
\hline
\cellcolor{lightgray}Codice				& TC 3.19								\\
\hline
\cellcolor{lightgray}Combinazione		& fnum2ftype2fver2cnum2cmes3 									\\
\hline
\cellcolor{lightgray}Precondizione		& L'utente ha aperto Eclipse e ha installato il plugin.								\\
\hline
\cellcolor{lightgray}Comando			& L'utente seleziona l'opzione ``Calcola metrica per il numero medio di refactoring di un package''.	\\
\hline
\cellcolor{lightgray}Risultato atteso	& Viene salvato nel database il valore relativo a questa metrica	\\
\hline
\end{tabular}
\end{table}

\begin{table}[ht]
\begin{tabular}{|p{3cm}|p{9cm}|}
\hline
\cellcolor{lightgray}Codice				& TC 3.20								\\
\hline
\cellcolor{lightgray}Combinazione		& fnum2ftype1fver2cnum2cmes1 									\\
\hline
\cellcolor{lightgray}Precondizione		& L'utente ha aperto Eclipse e ha installato il plugin.									\\
\hline
\cellcolor{lightgray}Comando			& L'utente seleziona l'opzione ``Calcola metrica per il numero medio di refactoring di un package''.	\\
\hline
\cellcolor{lightgray}Risultato atteso	& Viene salvato nel database il valore relativo a questa metrica	\\
\hline
\end{tabular}
\end{table}

\clearpage

\section{Metrica per il calcolo del numero medio di bug-fix di un package}

\begin{table}[ht]
\begin{tabular}{|p{3cm}|p{9cm}|}
\hline
\cellcolor{lightgray}Codice				& TC 4.1								\\
\hline
\cellcolor{lightgray}Combinazione		& fnum1cnum1									\\
\hline
\cellcolor{lightgray}Precondizione		& L'utente ha aperto Eclipse e ha installato il plugin.		\\
\hline
\cellcolor{lightgray}Comando			& L'utente seleziona l'opzione ``Calcola metrica per il numero medio di bug-fix di un package''.	\\
\hline
\cellcolor{lightgray}Risultato atteso	& Viene salvato nel database il valore 0 relativo a questa metrica.\\
\hline
\end{tabular}
\end{table}

\begin{table}[ht]
\begin{tabular}{|p{3cm}|p{9cm}|}
\hline
\cellcolor{lightgray}Codice				& TC 4.2								\\
\hline
\cellcolor{lightgray}Combinazione		& fnum1cnum2 									\\
\hline
\cellcolor{lightgray}Precondizione		& L'utente ha aperto Eclipse e ha installato il plugin.				\\
\hline
\cellcolor{lightgray}Comando			& L'utente seleziona l'opzione ``Calcola metrica per il numero medio di bug-fix di un package''.	\\
\hline
\cellcolor{lightgray}Risultato atteso	& Viene salvato nel database il valore relativo a questa metrica	\\
\hline
\end{tabular}
\end{table}

\begin{table}[ht]
\begin{tabular}{|p{3cm}|p{9cm}|}
\hline
\cellcolor{lightgray}Codice				& TC 4.3								\\
\hline
\cellcolor{lightgray}Combinazione		& fnum2ftype1fver1cnum1									\\
\hline
\cellcolor{lightgray}Precondizione		& L'utente ha aperto Eclipse e ha installato il plugin.					\\
\hline
\cellcolor{lightgray}Comando			& L'utente seleziona l'opzione ``Calcola metrica per il numero medio di bug-fix di un package''.	\\
\hline
\cellcolor{lightgray}Risultato atteso	& Viene salvato nel database il valore relativo a questa metrica	\\
\hline
\end{tabular}
\end{table}

\begin{table}[ht]
\begin{tabular}{|p{3cm}|p{9cm}|}
\hline
\cellcolor{lightgray}Codice				& TC 4.4								\\
\hline
\cellcolor{lightgray}Combinazione		& fnum2ftype1fver1cnum2								\\
\hline
\cellcolor{lightgray}Precondizione		& L'utente ha aperto Eclipse e ha installato il plugin.			\\
\hline
\cellcolor{lightgray}Comando			& L'utente seleziona l'opzione ``Calcola metrica per il numero medio di bug-fix di un package''.	\\
\hline
\cellcolor{lightgray}Risultato atteso	& Viene salvato nel database il valore relativo a questa metrica	\\
\hline
\end{tabular}
\end{table}

\begin{table}[ht]
\begin{tabular}{|p{3cm}|p{9cm}|}
\hline
\cellcolor{lightgray}Codice				& TC 4.5								\\
\hline
\cellcolor{lightgray}Combinazione		& fnum2ftype1fver2cnum1 									\\
\hline
\cellcolor{lightgray}Precondizione		& L'utente ha aperto Eclipse e ha installato il plugin.			\\
\hline
\cellcolor{lightgray}Comando			& L'utente seleziona l'opzione ``Calcola metrica per il numero medio di bug-fix di un package''.	\\
\hline
\cellcolor{lightgray}Risultato atteso	& Viene salvato nel database il valore relativo a questa metrica	\\
\hline
\end{tabular}
\end{table}

\begin{table}[ht]
\begin{tabular}{|p{3cm}|p{9cm}|}
\hline
\cellcolor{lightgray}Codice				& TC 4.6								\\
\hline
\cellcolor{lightgray}Combinazione		& fnum2ftype1fver2cnum2								\\
\hline
\cellcolor{lightgray}Precondizione		& L'utente ha aperto Eclipse e ha installato il plugin.				\\
\hline
\cellcolor{lightgray}Comando			& L'utente seleziona l'opzione ``Calcola metrica per il numero medio di bug-fix di un package''.	\\
\hline
\cellcolor{lightgray}Risultato atteso	& Viene salvato nel database il valore relativo a questa metrica	\\
\hline
\end{tabular}
\end{table}

\begin{table}[ht]
\begin{tabular}{|p{3cm}|p{9cm}|}
\hline
\cellcolor{lightgray}Codice				& TC 4.7								\\
\hline
\cellcolor{lightgray}Combinazione		& fnum2ftype2fver1cnum1							\\
\hline
\cellcolor{lightgray}Precondizione		& L'utente ha aperto Eclipse e ha installato il plugin.									\\
\hline
\cellcolor{lightgray}Comando			& L'utente seleziona l'opzione ``Calcola metrica per il numero medio di bug-fix di un package''.	\\
\hline
\cellcolor{lightgray}Risultato atteso	& Viene salvato nel database il valore relativo a questa metrica	\\
\hline
\end{tabular}
\end{table}

\begin{table}[ht]
\begin{tabular}{|p{3cm}|p{9cm}|}
\hline
\cellcolor{lightgray}Codice				& TC 4.8								\\
\hline
\cellcolor{lightgray}Combinazione		& fnum2ftype2fver1cnum2 									\\
\hline
\cellcolor{lightgray}Precondizione		& L'utente ha aperto Eclipse e ha installato il plugin.				\\
\hline
\cellcolor{lightgray}Comando			& L'utente seleziona l'opzione ``Calcola metrica per il numero medio di bug-fix di un package''.	\\
\hline
\cellcolor{lightgray}Risultato atteso	& Viene salvato nel database il valore relativo a questa metrica	\\
\hline
\end{tabular}
\end{table}

\begin{table}[ht]
\begin{tabular}{|p{3cm}|p{9cm}|}
\hline
\cellcolor{lightgray}Codice				& TC 4.9								\\
\hline
\cellcolor{lightgray}Combinazione		& fnum2ftype2fver2cnum1 									\\
\hline
\cellcolor{lightgray}Precondizione		& L'utente ha aperto Eclipse e ha installato il plugin.								\\
\hline
\cellcolor{lightgray}Comando			& L'utente seleziona l'opzione ``Calcola metrica per il numero medio di bug-fix di un package''.	\\
\hline
\cellcolor{lightgray}Risultato atteso	& Viene salvato nel database il valore relativo a questa metrica	\\
\hline
\end{tabular}
\end{table}

\begin{table}[ht]
\begin{tabular}{|p{3cm}|p{9cm}|}
\hline
\cellcolor{lightgray}Codice				& TC 4.10								\\
\hline
\cellcolor{lightgray}Combinazione		& fnum2ftype2fver2cnum2 									\\
\hline
\cellcolor{lightgray}Precondizione		& L'utente ha aperto Eclipse e ha installato il plugin.									\\
\hline
\cellcolor{lightgray}Comando			& L'utente seleziona l'opzione ``Calcola metrica per il numero medio di bug-fix di un package''.	\\
\hline
\cellcolor{lightgray}Risultato atteso	& Viene salvato nel database il valore relativo a questa metrica	\\
\hline
\end{tabular}
\end{table}
\clearpage

\section{Metrica per il calcolo del numero di autori di commit di un package}

\begin{table}[ht]
\begin{tabular}{|p{3cm}|p{9cm}|}
\hline
\cellcolor{lightgray}Codice				& TC 5.1								\\
\hline
\cellcolor{lightgray}Combinazione		& fnum1cnum1									\\
\hline
\cellcolor{lightgray}Precondizione		& L'utente ha aperto Eclipse e ha installato il plugin.		\\
\hline
\cellcolor{lightgray}Comando			& L'utente seleziona l'opzione ``Calcola metrica per il numero di autori di commit di un package''.	\\
\hline
\cellcolor{lightgray}Risultato atteso	& Viene salvato nel database il valore 0 relativo a questa metrica.\\
\hline
\end{tabular}
\end{table}

\begin{table}[ht]
\begin{tabular}{|p{3cm}|p{9cm}|}
\hline
\cellcolor{lightgray}Codice				& TC 5.2								\\
\hline
\cellcolor{lightgray}Combinazione		& fnum1cnum2cnaut1ceaut1 									\\
\hline
\cellcolor{lightgray}Precondizione		& L'utente ha aperto Eclipse e ha installato il plugin.				\\
\hline
\cellcolor{lightgray}Comando			& L'utente seleziona l'opzione ``Calcola metrica per il numero di autori di commit di un package''.	\\
\hline
\cellcolor{lightgray}Risultato atteso	& Viene salvato nel database il valore relativo a questa metrica	\\
\hline
\end{tabular}
\end{table}

\begin{table}[ht]
\begin{tabular}{|p{3cm}|p{9cm}|}
\hline
\cellcolor{lightgray}Codice				& TC 5.3								\\
\hline
\cellcolor{lightgray}Combinazione		& fnum2ftype1cnaut2ceaut1								\\
\hline
\cellcolor{lightgray}Precondizione		& L'utente ha aperto Eclipse e ha installato il plugin.					\\
\hline
\cellcolor{lightgray}Comando			& L'utente seleziona l'opzione ``Calcola metrica per il numero di autori di commit di un package''.	\\
\hline
\cellcolor{lightgray}Risultato atteso	& Viene salvato nel database il valore relativo a questa metrica	\\
\hline
\end{tabular}
\end{table}

\begin{table}[ht]
\begin{tabular}{|p{3cm}|p{9cm}|}
\hline
\cellcolor{lightgray}Codice				& TC 5.4								\\
\hline
\cellcolor{lightgray}Combinazione		& fnum1cnum2cnaut1ceaut2							\\
\hline
\cellcolor{lightgray}Precondizione		& L'utente ha aperto Eclipse e ha installato il plugin.			\\
\hline
\cellcolor{lightgray}Comando			& L'utente seleziona l'opzione ``Calcola metrica per il numero di autori di commit di un package''.	\\
\hline
\cellcolor{lightgray}Risultato atteso	& Viene salvato nel database il valore relativo a questa metrica	\\
\hline
\end{tabular}
\end{table}

\begin{table}[ht]
\begin{tabular}{|p{3cm}|p{9cm}|}
\hline
\cellcolor{lightgray}Codice				& TC 5.5								\\
\hline
\cellcolor{lightgray}Combinazione		& fnum1cnum2cnaut2ceaut2 									\\
\hline
\cellcolor{lightgray}Precondizione		& L'utente ha aperto Eclipse e ha installato il plugin.			\\
\hline
\cellcolor{lightgray}Comando			& L'utente seleziona l'opzione ``Calcola metrica per il numero di autori di commit di un package''.	\\
\hline
\cellcolor{lightgray}Risultato atteso	& Viene salvato nel database il valore relativo a questa metrica	\\
\hline
\end{tabular}
\end{table}

\begin{table}[ht]
\begin{tabular}{|p{3cm}|p{9cm}|}
\hline
\cellcolor{lightgray}Codice				& TC 5.6								\\
\hline
\cellcolor{lightgray}Combinazione		& fnum2ftype1fver1cnum1								\\
\hline
\cellcolor{lightgray}Precondizione		& L'utente ha aperto Eclipse e ha installato il plugin.				\\
\hline
\cellcolor{lightgray}Comando			& L'utente seleziona l'opzione ``Calcola metrica per il numero di autori di commit di un package''.	\\
\hline
\cellcolor{lightgray}Risultato atteso	& Viene salvato nel database il valore relativo a questa metrica	\\
\hline
\end{tabular}
\end{table}

\begin{table}[ht]
\begin{tabular}{|p{3cm}|p{9cm}|}
\hline
\cellcolor{lightgray}Codice				& TC 5.7								\\
\hline
\cellcolor{lightgray}Combinazione		& fnum2ftype1fver1cnum2cnaut1ceaut1							\\
\hline
\cellcolor{lightgray}Precondizione		& L'utente ha aperto Eclipse e ha installato il plugin.									\\
\hline
\cellcolor{lightgray}Comando			& L'utente seleziona l'opzione ``Calcola metrica per il numero di autori di commit di un package''.	\\
\hline
\cellcolor{lightgray}Risultato atteso	& Viene salvato nel database il valore relativo a questa metrica	\\
\hline
\end{tabular}
\end{table}

\begin{table}[ht]
\begin{tabular}{|p{3cm}|p{9cm}|}
\hline
\cellcolor{lightgray}Codice				& TC 5.8								\\
\hline
\cellcolor{lightgray}Combinazione		& fnum2ftype1fver1cnum2cnaut1ceaut2 									\\
\hline
\cellcolor{lightgray}Precondizione		& L'utente ha aperto Eclipse e ha installato il plugin.				\\
\hline
\cellcolor{lightgray}Comando			& L'utente seleziona l'opzione ``Calcola metrica per il numero di autori di commit di un package''.	\\
\hline
\cellcolor{lightgray}Risultato atteso	& Viene salvato nel database il valore relativo a questa metrica	\\
\hline
\end{tabular}
\end{table}

\begin{table}[ht]
\begin{tabular}{|p{3cm}|p{9cm}|}
\hline
\cellcolor{lightgray}Codice				& TC 5.9								\\
\hline
\cellcolor{lightgray}Combinazione		& fnum2ftype1fver1cnum2cnaut2ceaut1 									\\
\hline
\cellcolor{lightgray}Precondizione		& L'utente ha aperto Eclipse e ha installato il plugin.								\\
\hline
\cellcolor{lightgray}Comando			& L'utente seleziona l'opzione ``Calcola metrica per il numero di autori di commit di un package''.	\\
\hline
\cellcolor{lightgray}Risultato atteso	& Viene salvato nel database il valore relativo a questa metrica	\\
\hline
\end{tabular}
\end{table}

\begin{table}[ht]
\begin{tabular}{|p{3cm}|p{9cm}|}
\hline
\cellcolor{lightgray}Codice				& TC 5.10								\\
\hline
\cellcolor{lightgray}Combinazione		& fnum2ftype1fver1cnum2cnaut2ceaut2 									\\
\hline
\cellcolor{lightgray}Precondizione		& L'utente ha aperto Eclipse e ha installato il plugin.									\\
\hline
\cellcolor{lightgray}Comando			& L'utente seleziona l'opzione ``Calcola metrica per il numero di autori di commit di un package''.	\\
\hline
\cellcolor{lightgray}Risultato atteso	& Viene salvato nel database il valore relativo a questa metrica	\\
\hline
\end{tabular}
\end{table}

\begin{table}[ht]
\begin{tabular}{|p{3cm}|p{9cm}|}
\hline
\cellcolor{lightgray}Codice				& TC 5.11								\\
\hline
\cellcolor{lightgray}Combinazione		& fnum2ftype1fver2cnum1									\\
\hline
\cellcolor{lightgray}Precondizione		& L'utente ha aperto Eclipse e ha installato il plugin.		\\
\hline
\cellcolor{lightgray}Comando			& L'utente seleziona l'opzione ``Calcola metrica per il numero di autori di commit di un package''.	\\
\hline
\cellcolor{lightgray}Risultato atteso	& Viene salvato nel database il valore 0 relativo a questa metrica.\\
\hline
\end{tabular}
\end{table}

\begin{table}[ht]
\begin{tabular}{|p{3cm}|p{9cm}|}
\hline
\cellcolor{lightgray}Codice				& TC 5.12								\\
\hline
\cellcolor{lightgray}Combinazione		& fnum2ftype1fver2cnum2cnaut1ceaut1 									\\
\hline
\cellcolor{lightgray}Precondizione		& L'utente ha aperto Eclipse e ha installato il plugin.				\\
\hline
\cellcolor{lightgray}Comando			& L'utente seleziona l'opzione ``Calcola metrica per il numero di autori di commit di un package''.	\\
\hline
\cellcolor{lightgray}Risultato atteso	& Viene salvato nel database il valore relativo a questa metrica	\\
\hline
\end{tabular}
\end{table}

\begin{table}[ht]
\begin{tabular}{|p{3cm}|p{9cm}|}
\hline
\cellcolor{lightgray}Codice				& TC 5.13								\\
\hline
\cellcolor{lightgray}Combinazione		& fnum2ftype1fver2cnum2cnaut1ceaut2									\\
\hline
\cellcolor{lightgray}Precondizione		& L'utente ha aperto Eclipse e ha installato il plugin.					\\
\hline
\cellcolor{lightgray}Comando			& L'utente seleziona l'opzione ``Calcola metrica per il numero di autori di commit di un package''.	\\
\hline
\cellcolor{lightgray}Risultato atteso	& Viene salvato nel database il valore relativo a questa metrica	\\
\hline
\end{tabular}
\end{table}

\begin{table}[ht]
\begin{tabular}{|p{3cm}|p{9cm}|}
\hline
\cellcolor{lightgray}Codice				& TC 5.14								\\
\hline
\cellcolor{lightgray}Combinazione		& fnum2ftype1fver2cnum2cnaut2ceaut1								\\
\hline
\cellcolor{lightgray}Precondizione		& L'utente ha aperto Eclipse e ha installato il plugin.			\\
\hline
\cellcolor{lightgray}Comando			& L'utente seleziona l'opzione ``Calcola metrica per il numero di autori di commit di un package''.	\\
\hline
\cellcolor{lightgray}Risultato atteso	& Viene salvato nel database il valore relativo a questa metrica	\\
\hline
\end{tabular}
\end{table}

\begin{table}[ht]
\begin{tabular}{|p{3cm}|p{9cm}|}
\hline
\cellcolor{lightgray}Codice				& TC 5.15								\\
\hline
\cellcolor{lightgray}Combinazione		& fnum2ftype1fver2cnum2cnaut2ceaut2 									\\
\hline
\cellcolor{lightgray}Precondizione		& L'utente ha aperto Eclipse e ha installato il plugin.			\\
\hline
\cellcolor{lightgray}Comando			& L'utente seleziona l'opzione ``Calcola metrica per il numero di autori di commit di un package''.	\\
\hline
\cellcolor{lightgray}Risultato atteso	& Viene salvato nel database il valore relativo a questa metrica	\\
\hline
\end{tabular}
\end{table}



\begin{table}[ht]
\begin{tabular}{|p{3cm}|p{9cm}|}
\hline
\cellcolor{lightgray}Codice				& TC 5.16								\\
\hline
\cellcolor{lightgray}Combinazione		& fnum2ftype2fver1cnum1								\\
\hline
\cellcolor{lightgray}Precondizione		& L'utente ha aperto Eclipse e ha installato il plugin.				\\
\hline
\cellcolor{lightgray}Comando			& L'utente seleziona l'opzione ``Calcola metrica per il numero di autori di commit di un package''.	\\
\hline
\cellcolor{lightgray}Risultato atteso	& Viene salvato nel database il valore relativo a questa metrica	\\
\hline
\end{tabular}
\end{table}

\clearpage

\begin{table}[ht]
\begin{tabular}{|p{3cm}|p{9cm}|}
\hline
\cellcolor{lightgray}Codice				& TC 5.17								\\
\hline
\cellcolor{lightgray}Combinazione		& fnum2ftype2fver1cnum2cnaut1ceaut1							\\
\hline
\cellcolor{lightgray}Precondizione		& L'utente ha aperto Eclipse e ha installato il plugin.									\\
\hline
\cellcolor{lightgray}Comando			& L'utente seleziona l'opzione ``Calcola metrica per il numero di autori di commit di un package''.	\\
\hline
\cellcolor{lightgray}Risultato atteso	& Viene salvato nel database il valore relativo a questa metrica	\\
\hline
\end{tabular}
\end{table}

\begin{table}[ht]
\begin{tabular}{|p{3cm}|p{9cm}|}
\hline
\cellcolor{lightgray}Codice				& TC 5.18								\\
\hline
\cellcolor{lightgray}Combinazione		& fnum2ftype2fver1cnum2cnaut1ceaut2 									\\
\hline
\cellcolor{lightgray}Precondizione		& L'utente ha aperto Eclipse e ha installato il plugin.				\\
\hline
\cellcolor{lightgray}Comando			& L'utente seleziona l'opzione ``Calcola metrica per il numero di autori di commit di un package''.	\\
\hline
\cellcolor{lightgray}Risultato atteso	& Viene salvato nel database il valore relativo a questa metrica	\\
\hline
\end{tabular}
\end{table}

\begin{table}[ht]
\begin{tabular}{|p{3cm}|p{9cm}|}
\hline
\cellcolor{lightgray}Codice				& TC 5.19								\\
\hline
\cellcolor{lightgray}Combinazione		& fnum2ftype2fver1cnum2cnaut2ceaut1 									\\
\hline
\cellcolor{lightgray}Precondizione		& L'utente ha aperto Eclipse e ha installato il plugin.								\\
\hline
\cellcolor{lightgray}Comando			& L'utente seleziona l'opzione ``Calcola metrica per il numero di autori di commit di un package''.	\\
\hline
\cellcolor{lightgray}Risultato atteso	& Viene salvato nel database il valore relativo a questa metrica	\\
\hline
\end{tabular}
\end{table}

\begin{table}[ht]
\begin{tabular}{|p{3cm}|p{9cm}|}
\hline
\cellcolor{lightgray}Codice				& TC 5.20								\\
\hline
\cellcolor{lightgray}Combinazione		& fnum2ftype2fver1cnum2cnaut2ceaut2 									\\
\hline
\cellcolor{lightgray}Precondizione		& L'utente ha aperto Eclipse e ha installato il plugin.									\\
\hline
\cellcolor{lightgray}Comando			& L'utente seleziona l'opzione ``Calcola metrica per il numero di autori di commit di un package''.	\\
\hline
\cellcolor{lightgray}Risultato atteso	& Viene salvato nel database il valore relativo a questa metrica	\\
\hline
\end{tabular}
\end{table}

\begin{table}[ht]
\begin{tabular}{|p{3cm}|p{9cm}|}
\hline
\cellcolor{lightgray}Codice				& TC 5.21								\\
\hline
\cellcolor{lightgray}Combinazione		& fnum2ftype2fver2cnum1									\\
\hline
\cellcolor{lightgray}Precondizione		& L'utente ha aperto Eclipse e ha installato il plugin.		\\
\hline
\cellcolor{lightgray}Comando			& L'utente seleziona l'opzione ``Calcola metrica per il numero di autori di commit di un package''.	\\
\hline
\cellcolor{lightgray}Risultato atteso	& Viene salvato nel database il valore 0 relativo a questa metrica.\\
\hline
\end{tabular}
\end{table}

\begin{table}[ht]
\begin{tabular}{|p{3cm}|p{9cm}|}
\hline
\cellcolor{lightgray}Codice				& TC 5.22								\\
\hline
\cellcolor{lightgray}Combinazione		& fnum2ftype2fver2cnum2cnaut1ceaut1 									\\
\hline
\cellcolor{lightgray}Precondizione		& L'utente ha aperto Eclipse e ha installato il plugin.				\\
\hline
\cellcolor{lightgray}Comando			& L'utente seleziona l'opzione ``Calcola metrica per il numero di autori di commit di un package''.	\\
\hline
\cellcolor{lightgray}Risultato atteso	& Viene salvato nel database il valore relativo a questa metrica	\\
\hline
\end{tabular}
\end{table}

\begin{table}[ht]
\begin{tabular}{|p{3cm}|p{9cm}|}
\hline
\cellcolor{lightgray}Codice				& TC 5.23								\\
\hline
\cellcolor{lightgray}Combinazione		& fnum2ftype2fver2cnum2cnaut1ceaut2									\\
\hline
\cellcolor{lightgray}Precondizione		& L'utente ha aperto Eclipse e ha installato il plugin.					\\
\hline
\cellcolor{lightgray}Comando			& L'utente seleziona l'opzione ``Calcola metrica per il numero di autori di commit di un package''.	\\
\hline
\cellcolor{lightgray}Risultato atteso	& Viene salvato nel database il valore relativo a questa metrica	\\
\hline
\end{tabular}
\end{table}

\begin{table}[ht]
\begin{tabular}{|p{3cm}|p{9cm}|}
\hline
\cellcolor{lightgray}Codice				& TC 5.24								\\
\hline
\cellcolor{lightgray}Combinazione		& fnum2ftype2fver2cnum2cnaut2ceaut1								\\
\hline
\cellcolor{lightgray}Precondizione		& L'utente ha aperto Eclipse e ha installato il plugin.			\\
\hline
\cellcolor{lightgray}Comando			& L'utente seleziona l'opzione ``Calcola metrica per il numero di autori di commit di un package''.	\\
\hline
\cellcolor{lightgray}Risultato atteso	& Viene salvato nel database il valore relativo a questa metrica	\\
\hline
\end{tabular}
\end{table}

\begin{table}[ht]
\begin{tabular}{|p{3cm}|p{9cm}|}
\hline
\cellcolor{lightgray}Codice				& TC 5.25								\\
\hline
\cellcolor{lightgray}Combinazione		& fnum2ftype2fver2cnum2cnaut2ceaut2 									\\
\hline
\cellcolor{lightgray}Precondizione		& L'utente ha aperto Eclipse e ha installato il plugin.			\\
\hline
\cellcolor{lightgray}Comando			& L'utente seleziona l'opzione ``Calcola metrica per il numero di autori di commit di un package''.	\\
\hline
\cellcolor{lightgray}Risultato atteso	& Viene salvato nel database il valore relativo a questa metrica	\\
\hline
\end{tabular}
\end{table}

\clearpage

\section{Metrica per il calcolo del numero di righe aggiunte e rimosse}

\begin{table}[ht]
\begin{tabular}{|p{3cm}|p{9cm}|}
\hline
\cellcolor{lightgray}Codice				& TC 6.1								\\
\hline
\cellcolor{lightgray}Combinazione		& fnum1cnum1ccan1cins1									\\
\hline
\cellcolor{lightgray}Precondizione		& L'utente ha aperto Eclipse e ha installato il plugin.		\\
\hline
\cellcolor{lightgray}Comando			& L'utente seleziona l'opzione ``Calcola metrica per il numero di righe aggiunte e rimosse''.	\\
\hline
\cellcolor{lightgray}Risultato atteso	& Viene salvato nel database il valore 0 relativo a questa metrica.\\
\hline
\end{tabular}
\end{table}

\begin{table}[ht]
\begin{tabular}{|p{3cm}|p{9cm}|}
\hline
\cellcolor{lightgray}Codice				& TC 6.2								\\
\hline
\cellcolor{lightgray}Combinazione		& fnum1cnum2ccan1cins2 									\\
\hline
\cellcolor{lightgray}Precondizione		& L'utente ha aperto Eclipse e ha installato il plugin.				\\
\hline
\cellcolor{lightgray}Comando			& L'utente seleziona l'opzione ``Calcola metrica per il numero di righe aggiunte e rimosse''.	\\
\hline
\cellcolor{lightgray}Risultato atteso	& Viene salvato nel database il valore relativo a questa metrica	\\
\hline
\end{tabular}
\end{table}

\begin{table}[ht]
\begin{tabular}{|p{3cm}|p{9cm}|}
\hline
\cellcolor{lightgray}Codice				& TC 6.3								\\
\hline
\cellcolor{lightgray}Combinazione		& fnum1cnum2ccan2cins1								\\
\hline
\cellcolor{lightgray}Precondizione		& L'utente ha aperto Eclipse e ha installato il plugin.					\\
\hline
\cellcolor{lightgray}Comando			& L'utente seleziona l'opzione ``Calcola metrica per il numero di righe aggiunte e rimosse''.	\\
\hline
\cellcolor{lightgray}Risultato atteso	& Viene salvato nel database il valore relativo a questa metrica	\\
\hline
\end{tabular}
\end{table}

\begin{table}[ht]
\begin{tabular}{|p{3cm}|p{9cm}|}
\hline
\cellcolor{lightgray}Codice				& TC 6.4								\\
\hline
\cellcolor{lightgray}Combinazione		& fnum1cnum2ccan2cins2							\\
\hline
\cellcolor{lightgray}Precondizione		& L'utente ha aperto Eclipse e ha installato il plugin.			\\
\hline
\cellcolor{lightgray}Comando			& L'utente seleziona l'opzione ``Calcola metrica per il numero di righe aggiunte e rimosse''.	\\
\hline
\cellcolor{lightgray}Risultato atteso	& Viene salvato nel database il valore relativo a questa metrica	\\
\hline
\end{tabular}
\end{table}

\begin{table}[ht]
\begin{tabular}{|p{3cm}|p{9cm}|}
\hline
\cellcolor{lightgray}Codice				& TC 6.5								\\
\hline
\cellcolor{lightgray}Combinazione		& fnum2ftype1fver1cnum1ccan1cins1 									\\
\hline
\cellcolor{lightgray}Precondizione		& L'utente ha aperto Eclipse e ha installato il plugin.			\\
\hline
\cellcolor{lightgray}Comando			& L'utente seleziona l'opzione ``Calcola metrica per il numero di righe aggiunte e rimosse''.	\\
\hline
\cellcolor{lightgray}Risultato atteso	& Viene salvato nel database il valore relativo a questa metrica	\\
\hline
\end{tabular}
\end{table}

\begin{table}[ht]
\begin{tabular}{|p{3cm}|p{9cm}|}
\hline
\cellcolor{lightgray}Codice				& TC 6.6								\\
\hline
\cellcolor{lightgray}Combinazione		& fnum2ftype1fver1cnum2ccan1cins2								\\
\hline
\cellcolor{lightgray}Precondizione		& L'utente ha aperto Eclipse e ha installato il plugin.				\\
\hline
\cellcolor{lightgray}Comando			& L'utente seleziona l'opzione ``Calcola metrica per il numero di righe aggiunte e rimosse''.	\\
\hline
\cellcolor{lightgray}Risultato atteso	& Viene salvato nel database il valore relativo a questa metrica	\\
\hline
\end{tabular}
\end{table}

\begin{table}[ht]
\begin{tabular}{|p{3cm}|p{9cm}|}
\hline
\cellcolor{lightgray}Codice				& TC 6.7								\\
\hline
\cellcolor{lightgray}Combinazione		& fnum2ftype1fver1cnum2ccan2cins1							\\
\hline
\cellcolor{lightgray}Precondizione		& L'utente ha aperto Eclipse e ha installato il plugin.									\\
\hline
\cellcolor{lightgray}Comando			& L'utente seleziona l'opzione ``Calcola metrica per il numero di righe aggiunte e rimosse''.	\\
\hline
\cellcolor{lightgray}Risultato atteso	& Viene salvato nel database il valore relativo a questa metrica	\\
\hline
\end{tabular}
\end{table}

\begin{table}[ht]
\begin{tabular}{|p{3cm}|p{9cm}|}
\hline
\cellcolor{lightgray}Codice				& TC 6.8								\\
\hline
\cellcolor{lightgray}Combinazione		& fnum2ftype1fver1cnum2ccan2cins2 									\\
\hline
\cellcolor{lightgray}Precondizione		& L'utente ha aperto Eclipse e ha installato il plugin.				\\
\hline
\cellcolor{lightgray}Comando			& L'utente seleziona l'opzione ``Calcola metrica per il numero di righe aggiunte e rimosse''.	\\
\hline
\cellcolor{lightgray}Risultato atteso	& Viene salvato nel database il valore relativo a questa metrica	\\
\hline
\end{tabular}
\end{table}

\begin{table}[ht]
\begin{tabular}{|p{3cm}|p{9cm}|}
\hline
\cellcolor{lightgray}Codice				& TC 6.9								\\
\hline
\cellcolor{lightgray}Combinazione		& fnum2ftype1fver2cnum1ccan1cins1 									\\
\hline
\cellcolor{lightgray}Precondizione		& L'utente ha aperto Eclipse e ha installato il plugin.								\\
\hline
\cellcolor{lightgray}Comando			& L'utente seleziona l'opzione ``Calcola metrica per il numero di righe aggiunte e rimosse''.	\\
\hline
\cellcolor{lightgray}Risultato atteso	& Viene salvato nel database il valore relativo a questa metrica	\\
\hline
\end{tabular}
\end{table}

\begin{table}[ht]
\begin{tabular}{|p{3cm}|p{9cm}|}
\hline
\cellcolor{lightgray}Codice				& TC 6.10								\\
\hline
\cellcolor{lightgray}Combinazione		& fnum2ftype1fver2cnum2ccan1cins2 									\\
\hline
\cellcolor{lightgray}Precondizione		& L'utente ha aperto Eclipse e ha installato il plugin.									\\
\hline
\cellcolor{lightgray}Comando			& L'utente seleziona l'opzione ``Calcola metrica per il numero di righe aggiunte e rimosse''.	\\
\hline
\cellcolor{lightgray}Risultato atteso	& Viene salvato nel database il valore relativo a questa metrica	\\
\hline
\end{tabular}
\end{table}

\begin{table}[ht]
\begin{tabular}{|p{3cm}|p{9cm}|}
\hline
\cellcolor{lightgray}Codice				& TC 6.11								\\
\hline
\cellcolor{lightgray}Combinazione		& fnum2ftype1fver2cnum2ccan2cins1									\\
\hline
\cellcolor{lightgray}Precondizione		& L'utente ha aperto Eclipse e ha installato il plugin.		\\
\hline
\cellcolor{lightgray}Comando			& L'utente seleziona l'opzione ``Calcola metrica per il numero di righe aggiunte e rimosse''.	\\
\hline
\cellcolor{lightgray}Risultato atteso	& Viene salvato nel database il valore 0 relativo a questa metrica.\\
\hline
\end{tabular}
\end{table}

\begin{table}[ht]
\begin{tabular}{|p{3cm}|p{9cm}|}
\hline
\cellcolor{lightgray}Codice				& TC 6.12								\\
\hline
\cellcolor{lightgray}Combinazione		& fnum2ftype1fver2cnum2ccan2cins2 									\\
\hline
\cellcolor{lightgray}Precondizione		& L'utente ha aperto Eclipse e ha installato il plugin.				\\
\hline
\cellcolor{lightgray}Comando			& L'utente seleziona l'opzione ``Calcola metrica per il numero di righe aggiunte e rimosse''.	\\
\hline
\cellcolor{lightgray}Risultato atteso	& Viene salvato nel database il valore relativo a questa metrica	\\
\hline
\end{tabular}
\end{table}

\begin{table}[ht]
\begin{tabular}{|p{3cm}|p{9cm}|}
\hline
\cellcolor{lightgray}Codice				& TC 6.13								\\
\hline
\cellcolor{lightgray}Combinazione		& fnum2ftype2fver1cnum1ccan1cins1									\\
\hline
\cellcolor{lightgray}Precondizione		& L'utente ha aperto Eclipse e ha installato il plugin.					\\
\hline
\cellcolor{lightgray}Comando			& L'utente seleziona l'opzione ``Calcola metrica per il numero di righe aggiunte e rimosse''.	\\
\hline
\cellcolor{lightgray}Risultato atteso	& Viene salvato nel database il valore relativo a questa metrica	\\
\hline
\end{tabular}
\end{table}

\begin{table}[ht]
\begin{tabular}{|p{3cm}|p{9cm}|}
\hline
\cellcolor{lightgray}Codice				& TC 6.14								\\
\hline
\cellcolor{lightgray}Combinazione		& fnum2ftype2fver1cnum2ccan1cins2								\\
\hline
\cellcolor{lightgray}Precondizione		& L'utente ha aperto Eclipse e ha installato il plugin.			\\
\hline
\cellcolor{lightgray}Comando			& L'utente seleziona l'opzione ``Calcola metrica per il numero di righe aggiunte e rimosse''.	\\
\hline
\cellcolor{lightgray}Risultato atteso	& Viene salvato nel database il valore relativo a questa metrica	\\
\hline
\end{tabular}
\end{table}

\begin{table}[ht]
\begin{tabular}{|p{3cm}|p{9cm}|}
\hline
\cellcolor{lightgray}Codice				& TC 6.15								\\
\hline
\cellcolor{lightgray}Combinazione		& fnum2ftype2fver1cnum2ccan2cins1 									\\
\hline
\cellcolor{lightgray}Precondizione		& L'utente ha aperto Eclipse e ha installato il plugin.			\\
\hline
\cellcolor{lightgray}Comando			& L'utente seleziona l'opzione ``Calcola metrica per il numero di righe aggiunte e rimosse''.	\\
\hline
\cellcolor{lightgray}Risultato atteso	& Viene salvato nel database il valore relativo a questa metrica	\\
\hline
\end{tabular}
\end{table}



\begin{table}[ht]
\begin{tabular}{|p{3cm}|p{9cm}|}
\hline
\cellcolor{lightgray}Codice				& TC 6.16								\\
\hline
\cellcolor{lightgray}Combinazione		& fnum2ftype2fver1cnum2ccan2cins2								\\
\hline
\cellcolor{lightgray}Precondizione		& L'utente ha aperto Eclipse e ha installato il plugin.				\\
\hline
\cellcolor{lightgray}Comando			& L'utente seleziona l'opzione ``Calcola metrica per il numero di righe aggiunte e rimosse''.	\\
\hline
\cellcolor{lightgray}Risultato atteso	& Viene salvato nel database il valore relativo a questa metrica	\\
\hline
\end{tabular}
\end{table}

\clearpage

\begin{table}[ht]
\begin{tabular}{|p{3cm}|p{9cm}|}
\hline
\cellcolor{lightgray}Codice				& TC 6.17								\\
\hline
\cellcolor{lightgray}Combinazione		& fnum2ftype2fver2cnum1ccan1cins1							\\
\hline
\cellcolor{lightgray}Precondizione		& L'utente ha aperto Eclipse e ha installato il plugin.									\\
\hline
\cellcolor{lightgray}Comando			& L'utente seleziona l'opzione ``Calcola metrica per il numero di righe aggiunte e rimosse''.	\\
\hline
\cellcolor{lightgray}Risultato atteso	& Viene salvato nel database il valore relativo a questa metrica	\\
\hline
\end{tabular}
\end{table}

\begin{table}[ht]
\begin{tabular}{|p{3cm}|p{9cm}|}
\hline
\cellcolor{lightgray}Codice				& TC 6.18								\\
\hline
\cellcolor{lightgray}Combinazione		& fnum2ftype2fver2cnum2ccan1cins2 									\\
\hline
\cellcolor{lightgray}Precondizione		& L'utente ha aperto Eclipse e ha installato il plugin.				\\
\hline
\cellcolor{lightgray}Comando			& L'utente seleziona l'opzione ``Calcola metrica per il numero di righe aggiunte e rimosse''.	\\
\hline
\cellcolor{lightgray}Risultato atteso	& Viene salvato nel database il valore relativo a questa metrica	\\
\hline
\end{tabular}
\end{table}

\begin{table}[ht]
\begin{tabular}{|p{3cm}|p{9cm}|}
\hline
\cellcolor{lightgray}Codice				& TC 6.19								\\
\hline
\cellcolor{lightgray}Combinazione		& fnum2ftype2fver2cnum2ccan2cins1 									\\
\hline
\cellcolor{lightgray}Precondizione		& L'utente ha aperto Eclipse e ha installato il plugin.								\\
\hline
\cellcolor{lightgray}Comando			& L'utente seleziona l'opzione ``Calcola metrica per il numero di righe aggiunte e rimosse''.	\\
\hline
\cellcolor{lightgray}Risultato atteso	& Viene salvato nel database il valore relativo a questa metrica	\\
\hline
\end{tabular}
\end{table}

\begin{table}[ht]
\begin{tabular}{|p{3cm}|p{9cm}|}
\hline
\cellcolor{lightgray}Codice				& TC 6.20								\\
\hline
\cellcolor{lightgray}Combinazione		& fnum2ftype2fver2cnum2ccan2cins2 									\\
\hline
\cellcolor{lightgray}Precondizione		& L'utente ha aperto Eclipse e ha installato il plugin.									\\
\hline
\cellcolor{lightgray}Comando			& L'utente seleziona l'opzione ``Calcola metrica per il numero di righe aggiunte e rimosse''.	\\
\hline
\cellcolor{lightgray}Risultato atteso	& Viene salvato nel database il valore relativo a questa metrica	\\
\hline
\end{tabular}
\end{table}

\clearpage

