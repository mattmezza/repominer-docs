\chapter{Test Case}

\section{Metrica per il calcolo del numero di revisioni del sistema}
\subsection{Esempio Test Case 1}
\begin{table}[ht]
\begin{tabular}{|l|l|}
\hline
Test Case 1.1:				&	{Key = cnum1.}						\\
\hline
Insieme dei cambiamenti:	&	List$<$Changes$>$					\\
\hline
\multicolumn{2}{|l|}{ Comando per eseguire il test: }					\\
\multicolumn{2}{|p{13cm}|}{ L'utente seleziona l'opzione ``Calcola metrica numero di revisioni del sistema'' }			\\
\hline
\multicolumn{2}{|l|}{ Risultato atteso: }			\\
\multicolumn{2}{|p{13cm}|}{ ``Nessun cambiamento nel sistema'' }			\\
\hline
\end{tabular}
\end{table}
\subsection{Esempio Test Case 2}
\begin{table}[ht]
\begin{tabular}{|p{4cm}|p{9cm}|}
\hline
\rowcolor{lightgray}Codice	&	TC 1.2								\\
\hline
Combinazione				&	cnum2								\\
\hline
Parametro					&	Insieme dei cambiamenti				\\
\hline
Categoria					&	Numero								\\
\hline
Scelta						&	List$<$Changes$>$					\\
\hline
Comando						&										\\
\hline
Risultato atteso 			&	``Nessun cambiamento nel sistema''	\\
\hline
Risultato attuale 			&	ERROR								\\
\hline
\end{tabular}
\end{table}

\clearpage
