\chapter{Stato di evoluzione}
\section{Stato del progetto e risultati}
È stata rilasciata la prima versione di buona parte della documentazione del progetto. Una parte di questa documentazione probabilmente richiederà modifiche successive e sarà necessario rilasciare nuove versioni. È stata inoltre rilasciata la prima versione del codice sorgente completamente funzionante.

\vspace{0.5cm}
{\setlength{\parindent}{0cm}
La stesura del \textbf{Documento di analisi} ha compreso le seguenti attività:
\begin{itemize}
 \item Definizione dei requisiti funzionali
 \item Definizione dell'use case model
 \item Creazione del class diagram
 \item Impact analysis sul sistema corrente
\end{itemize}
}

\vspace{0.5cm}

{\setlength{\parindent}{0cm}
La stesura dell'\textbf{ODD} ha compreso le seguenti attività:
\begin{itemize}
 \item Stesura delle naming conventions
 \item Definizione dei trade-off di design
 \item Documentazione del codice e creazione della javadoc
\end{itemize}
Il documento non è completo e necessita di revisione.
}

\vspace{0.5cm}

{{\setlength{\parindent}{0cm}
La stesura del \textbf{Test Plan} ha compreso le seguenti attività:
\begin{itemize}
 \item Definizione della metodologia da utilizzare per la definizione dei casi di test
 \item Definizione della strategia di testing
 \item Progettazione dei casi di test
\end{itemize}
La pianificazione ha riguardato il sistema completo. Questo ha comportato ritardi nello schedule, ma permetterà di velocizzare le fasi successive.
}

\vspace{0.5cm}

{\setlength{\parindent}{0cm}
La stesura del \textbf{TCS} ha compreso le seguenti attività:
\begin{itemize}
 \item Definizione dei progetti di test
 \item Descrizione dettagliata dei casi di test progettati nel TP
\end{itemize}
}

\vspace{0.5cm}

{\setlength{\parindent}{0cm}
L'implementazione del \textbf{codice sorgente} ha riguardato le seguenti metriche:
\begin{itemize}
 \item Numero di revisioni del sistema
 \item Numero medio di volte in cui i file di un package hanno subito cambiamenti
 \item Numero medio di volte in cui i file di un package hanno subito operazioni di refactoring
 \item Numero medio di volte in cui i file di un package hanno subito operazioni di bug fixing
 \item Numero di autori di commit effettuati all’interno di un package
 \item Numero di linee aggiunte o rimosse (somma, media e massimo)
 \item Dimensione media dei file modificati
\end{itemize}
È stata, inoltre, creata la struttura del plugin, che permetterà di aggiungere facilmente nuove metriche.
}

\vspace{0.5cm}

{\setlength{\parindent}{0cm}
La stesura dei \textbf{Report di testing} ha compreso le seguenti attività:
\begin{itemize}
 \item Esecuzione dei casi di test specificati nel TCS (quelli riguardanti il primo incremento)
 \item Stesura del Test Log (traccia di tutti i casi di test eseguiti)
 \item Stesura del Test Incident Report (contenente i test che hanno evidenziato failure del sistema)
 \item Stesura del Test Summary Report (contiene un riassunto dell'attività di testing)
\end{itemize}
}

Attualmente si sta procedendo con l'implementazione del secondo incremento.