\chapter{Progresso delle attività di progetto}
\section{Revisione dell'Object Design Document}
\subsection{Sinossi}
L'attività consiste nella revisione dell'ODD e nella scrittura della javadoc relativamente al codice sorgente del secondo incremento.

\begin{table}[ht]
 \begin{tabular}{|p{6cm}|p{2cm}|p{2cm}|p{2cm}|}
    \hline
    \rowcolor{Gray}\textbf{Descrizione}			& \textbf{Autori}		& \textbf{Data inizio}			& \textbf{Data fine}	\\
    \hline
    Revisione ODD					& GG, MM			& 21/06/2014				& 23/06/2014		\\
    \hline
    Documentazione del codice				& GG, MM			& 24/06/2014				& 1/07/2014		\\
    \hline
 \end{tabular}
\end{table}

\subsection{Obiettivi e risultati}
Gli obiettivi pianificati per la stesura dell'ODD sono specificati nel SPMP. Il documento necessita di una revisione finale, mentre la documentazione del codice sorgente è completa.

\subsection{Allocazione del lavoro}
La tabella mostra come ogni task è stato ripartito tra i membri del team, e quindi in che percentuale ognuno ha partecipato a una determinata attività.
\begin{table}[ht]
 \begin{tabular}{|p{7.5cm}|p{1cm}|p{1cm}|p{1cm}|p{1cm}|}
  \hline
  \rowcolor{Gray}\textbf{Task}			& \textbf{CB}		& \textbf{GG}		& \textbf{MM}		& \textbf{SS}		\\
  \hline
  Revisione ODD					& 0\%			& 50\%			& 50\%			& 0\%			\\
  \hline
  Documentazione del codice			& 0\%			& 50\%			& 50\%			& 0\%			\\
  \hline
 \end{tabular}
\end{table}

\section{Implementazione del codice sorgente}
\subsection{Sinossi}
L'attività consiste nell'implementare quanto specificato nel Documento di analisi, seguendo la progettazione effettuata in fase di object design.

\begin{table}[ht]
 \begin{tabular}{|p{6cm}|p{2cm}|p{2cm}|p{2cm}|}
    \hline
    \rowcolor{Gray}\textbf{Descrizione}			& \textbf{Autori}		& \textbf{Data inizio}			& \textbf{Data fine}	\\
    \hline
    Implementazione del BCC				& GG, MM			& 24/06/2014				& 29/06/2014		\\
    \hline
    Aggiunta delle impostazioni del plugin		& GG, MM			& 24/06/2014				& 23/06/2014		\\
    \hline
 \end{tabular}
\end{table}

\subsection{Obiettivi e risultati}
Gli obiettivi pianificati per l'implementazione del primo incremento sono specificati nel SPMP. La seconda versione è stata rilasciata nei tempi previsti secondo il nuovo schedule, modificato dopo il primo report di management.

\subsection{Allocazione del lavoro}
La tabella mostra come ogni task è stato ripartito tra i membri del team, e quindi in che percentuale ognuno ha partecipato a una determinata attività.
\begin{table}[ht]
 \centering
 \begin{tabular}{|p{7.5cm}|p{1cm}|p{1cm}|p{1cm}|p{1cm}|}
  \hline
  \rowcolor{Gray}\textbf{Task}			& \textbf{CB}		& \textbf{GG}		& \textbf{MM}		& \textbf{SS}		\\
  \hline
  Implementazione del BCC			& 0\%			& 70\%			& 30\%			& 0\%			\\
  \hline
  Aggiunta delle impostazioni del plugin	& 0\%			& 20\%			& 80\%			& 0\%			\\
  \hline
 \end{tabular}
\end{table}



\section{Report di testing}
\subsection{Sinossi}
L'attività consiste nel testare il sistema eseguendo i casi di test specificati nel TCS e nell'annotare i risultati dei test in due documenti: il Test Log, che tiene traccia di tutti i casi di test eseguiti e il Test Incident Report, nel quale sono descritti dettagliatamente i casi di test che hanno evidenziato failure del sistema.

\begin{table}[ht]
 \begin{tabular}{|p{6cm}|p{2cm}|p{2cm}|p{2cm}|}
    \hline
    \rowcolor{Gray}\textbf{Descrizione}			& \textbf{Autori}		& \textbf{Data inizio}			& \textbf{Data fine}	\\
    \hline
    Esecuzione dei casi di test				& CB, SS			& 30/06/2014				& 1/07/2014		\\
    \hline
    Stesura del Test Log				& CB, SS			& 30/06/2014				& 1/07/2014		\\
    \hline
    Stesura del Test Incident Report			& CB, SS			& 30/06/2014				& 1/07/2014		\\
    \hline
    Stesura del Test Summary Report			& CB, SS			& 30/06/2014				& 1/07/2014		\\
    \hline
 \end{tabular}
\end{table}

\subsection{Obiettivi e risultati}
Gli obiettivi pianificati per la reportistica di testing sono specificati nel SPMP. L'attività ha richiesto meno tempo del previsto (2 giorni al posto di 5). L'attività è comunque  terminata in ritardo a causa del ritardo nell'implementazione. Questo spinge a modificare lo schedule, riducendo il tempo previsto per il testing dei prossimi due incrementi. 

\subsection{Allocazione del lavoro}
La tabella mostra come ogni task è stato ripartito tra i membri del team, e quindi in che percentuale ognuno ha partecipato a una determinata attività.
\begin{table}[ht]
 \centering
 \begin{tabular}{|p{7.5cm}|p{1cm}|p{1cm}|p{1cm}|p{1cm}|}
  \hline
  \rowcolor{Gray}\textbf{Task}			& \textbf{CB}		& \textbf{GG}		& \textbf{MM}		& \textbf{SS}		\\
  \hline
  Esecuzione dei casi di test			& 50\%			& 0\%			& 0\%			& 50\%			\\
  \hline
  Stesura del Test Log				& 100\%			& 0\%			& 0\%			& 0\%			\\
  \hline
  Stesura del Test Incident Report		& 0\%			& 0\%			& 0\%			& 100\%			\\
  \hline
  Esecuzione dei casi di test			& 50\%			& 0\%			& 0\%			& 50\%			\\
  \hline
 \end{tabular}
\end{table}
