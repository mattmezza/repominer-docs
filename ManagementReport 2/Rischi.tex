\chapter{Gestione dei rischi}
\section{Evoluzione dei rischi}
In questa sezione si analizza l'evoluzione dei rischi e come sono cambiate le probabilità che un determinato rischi si verifichi. 
\\ \\
\textbf{Legenda} \\ \\
\textit{Probabilità}:
\begin{itemize}
\item \textit{Bassa}:la probabilità che il rischio si verifichi è compresa nell'intervallo 0\%-30\%.
\item \textit{Media}:la probabilità che il rischio si verifichi è compresa nell'intervallo 30\%-60\%.
\item \textit{Alta}:la probabilità che il rischio si verifichi è compresa nell'intervallo 60\%-100\%.
\end{itemize}
\textit{Impatto}:
\begin{itemize}
\item \textit{Insignificante}: il verificarsi del rischio non compromette la buona riuscita del progetto.
\item \textit{Tollerabile}: classifica rischi di semplice gestione in quanto, le problematiche ad essi collegate, sono facilmente risolvibili.
\item \textit{Serio}: rischi di questo tipo possono rallentare notevolmente il progetto mettendone a rischio la buona riuscita; è necessario risolverli nel più breve tempo possibile.
\item \textit{Catastrofico}: rischi di difficile soluzione; se non affrontati per tempo portano di sicuro al fallimento.
\end{itemize}

\newcolumntype{C}[1]{>{\centering}p{#1}}
\begin{table}[ht]
\centering
\begin{tabular}{|C{4cm}|C{5cm}|C{3cm}|}
\hline 
\rowcolor{Gray}\textbf{Rischio} 					& \textbf{Variazione della probabilità} & \textbf{Impatto} \tabularnewline
\hline 
Skill del team insufficienti 						& Invariata: Bassa & Tollerabile \tabularnewline
\hline
Poca conoscenza del dominio applicativo 				& Invariata: Bassa & Serio \tabularnewline
\hline
Perdita di un elemento del team 					& Invariata: Bassa & Tollerabile \tabularnewline
\hline
Ritardo consegna task 							& Invariata: Media & Tollerabile \tabularnewline
\hline
Implementazione non completa 						& Variata: Nulla & Serio \tabularnewline
\hline
Attività prolungata oltre la scadenza prevista in fase di schedule 	& Variata: Bassa   & Serio \tabularnewline
\hline
Incontro settimanale annullato 						& Invariata: Bassa & Tollerabile \tabularnewline
\hline
\end{tabular}
\caption{Identificazione dei rischi}
\end{table}

I rischi identificati in fase di pianificazione sono sostanzialmente rimasti invariati. Il rischio di non completare l'implementazione si è azzerato, dato che, nel peggiore dei casi, è possibile consegnare il sistema fino alla versione 2.0, ovvero quella da poco rilasciata. Si è, inoltre, abbassata la probabilità che le prossime attività previste abbiano ritardi, dato che il team ha acquisito confidenza con le tecnologie usate e con il sistema che sta implementando. Salvo problemi di natura esterna, non dovrebbero esserci ulteriori ritardi nello schedule.

\section{Rischi affrontati e strategie applicate}
Durante questa fase non sono stati affrontati rischi previsti né imprevisti.