\chapter{Test Incidents}

In questa sezione verranno indicati tutti i test incident. Si provvederà a fornire le seguenti informazioni per ogni caso di test:
\begin{itemize}
 \item Identificativo del caso di test (tracciabile nel Test Case Specification e nel Test Plan)
 \item Risultato prodotto e risultato atteso
 \item Input che ha portato alla failure
 \item Note aggiuntive
\end{itemize}

Per visualizzare i risultati è sufficiente eseguire la query SQL descritta nella figura \ref{figmetpack}.
\begin{figure}
 \begin{lstlisting}
  SELECT M.name, M.description, M1.value
  FROM repominer.metrics AS M, repominer.package_metrics AS M1 
  WHERE M.id = M1.metric
  \end{lstlisting}
  \caption{\footnotesize Query utilizzata per estrarre i valori relativi alle metriche di package}
  \label{figmetpack}
\end{figure}

\newpage
\section{Metrica per il calcolo del numero medio di volte in cui i file di un package hanno subito cambiamenti}
\begin{table}[ht]
	\centering
	\begin{tabular}{| l | l | l |}
		\hline
		\rowcolor{lightgray}\textbf{Caso di test}	&	\textbf{Risultato atteso}	&	\textbf{Risultato ottenuto}	\\
		\hline
		TC 2.2						&	1				&	0			   	\\
		\hline
	\end{tabular}
\end{table}

\subsection*{TC 2.2}
Passi da eseguire per arrivare alla failure:
\begin{itemize}
 \item Avviare Eclipse
 \item Pulire il database
 \item Caricare il file ``Vuoto Versioning1.sql'' nella cartella ``sql'' del progetto
 \item Cliccare sul tasto ``Calculate history metrics''
 \item Eseguire la query che mostra i risultati per le metriche di package
\end{itemize}
Verrà mostrata la seguente tupla:
\begin{table}[ht]
 \centering
 \footnotesize
 \begin{tabular}{|l|p{8cm}|l|}
  \hline
  \textbf{name}		& \textbf{description}		& \textbf{value}		\\
  \hline
  mean\_NCHANGE		& Mean number of time in which files of a package have been modified	& 	0 \\
  \hline
 \end{tabular} 
\end{table}
\vspace{0.5cm}
\newline
Il progetto di riferimento ha le seguenti caratteristiche:
\begin{itemize}
 \item Non ci sono file locali
 \item È presente un solo cambiamento (cancellazione di un file)
\end{itemize}



\paragraph{Note} Il sistema non considera i cambiamenti su file che hanno subito cambiamenti ma che non sono più presenti in locale (perché cancellati).


\newpage

\section{Metrica per il calcolo del numero medio di bug-fix di un package}
\begin{table}[ht]
	\centering
	\begin{tabular}{| l | l | l |}
		\hline
		\rowcolor{lightgray}\textbf{Caso di test}	&	\textbf{Risultato atteso}	&	\textbf{Risultato ottenuto}	\\
		\hline
		TC 4.16						&	1				&	0			   	\\
		\hline
	\end{tabular}
\end{table}

\subsection*{TC 4.16}
Passi da eseguire per arrivare alla failure:
\begin{itemize}
 \item Avviare Eclipse
 \item Pulire il database
 \item Caricare il file ``Bugfix4.sql'' nella cartella ``sql'' del progetto
 \item Cliccare sul tasto ``Calculate history metrics''
 \item Eseguire la query che mostra i risultati per le metriche di package
\end{itemize}
Verrà mostrata la seguente tupla:
\begin{table}[ht]
 \centering
 \footnotesize
 \begin{tabular}{|l|p{8cm}|l|}
  \hline
  \textbf{name}		& \textbf{description}		& \textbf{value}		\\
  \hline
  mean\_NFIX		& Mean number of time in which files of a package have been fixed for a bug	& 	0 \\
  \hline
 \end{tabular} 
\end{table}
\vspace{0.5cm}
\newline
Il progetto di riferimento ha le seguenti caratteristiche:
\begin{itemize}
 \item È presente un file locale
 \item È presente un solo cambiamento
 \item Il cambiamento in questione è un bugfix (messaggio: ``Bugfix'')
\end{itemize}

\paragraph{Note} Il sistema non identifica il cambiamento presente come bug-fix.
\newpage



\section{Metrica per il calcolo del numero di autori di commit di un package}
\begin{table}[ht]
	\centering
	\begin{tabular}{| l | l | l |}
		\hline
		\rowcolor{lightgray}\textbf{Caso di test}	&	\textbf{Risultato atteso}	&	\textbf{Risultato ottenuto}	\\
		\hline
		TC 5.9						&	2				&	1			   	\\
		\hline
		TC 5.19						&	2				&	1			   	\\
		\hline
	\end{tabular}
\end{table}

\subsection*{TC 5.9}
Passi da eseguire per arrivare alla failure:
\begin{itemize}
 \item Avviare Eclipse
 \item Pulire il database
 \item Caricare il file ``Progetto9.sql'' nella cartella ``sql'' del progetto
 \item Cliccare sul tasto ``Calculate history metrics''
 \item Eseguire la query che mostra i risultati per le metriche di package
\end{itemize}
Verrà mostrata la seguente tupla:
\begin{table}[ht]
 \centering
 \footnotesize
 \begin{tabular}{|l|p{8cm}|l|}
  \hline
  \textbf{name}		& \textbf{description}		& \textbf{value}		\\
  \hline
  NAUTH		& Number of authors of changes made on a package	& 	1 \\
  \hline
 \end{tabular} 
\end{table}
\vspace{0.5cm}
\newline
Il progetto di riferimento ha le seguenti caratteristiche:
\begin{itemize}
 \item Sono presenti due file locali
 \item Sono presenti due cambiamenti
 \item Gli autori dei cambiamenti sono diversi e non hanno indirizzo email
\end{itemize}

\paragraph{Note} Il sistema non considera il caso in cui un autore non ha indirizzo email.


\subsection*{TC 5.19}
Passi da eseguire per arrivare alla failure:
\begin{itemize}
 \item Avviare Eclipse
 \item Pulire il database
 \item Caricare il file ``Progetto19.sql'' nella cartella ``sql'' del progetto
 \item Cliccare sul tasto ``Calculate history metrics''
 \item Eseguire la query che mostra i risultati per le metriche di package
\end{itemize}
Verrà mostrata la seguente tupla:
\begin{table}[ht]
 \centering
 \footnotesize
 \begin{tabular}{|l|p{8cm}|l|}
  \hline
  \textbf{name}		& \textbf{description}		& \textbf{value}		\\
  \hline
  NAUTH		& Number of authors of changes made on a package	& 	1 \\
  \hline
 \end{tabular} 
\end{table}
\vspace{0.5cm}
\newline
Il progetto di riferimento ha le seguenti caratteristiche:
\begin{itemize}
 \item Sono presenti due file locali
 \item Sono presenti due cambiamenti
 \item Gli autori dei cambiamenti sono diversi e non hanno indirizzo email
\end{itemize}

\paragraph{Note} Il sistema non considera il caso in cui un autore non ha indirizzo email.
