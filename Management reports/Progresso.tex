\chapter{Progresso delle attività di progetto}
\section{Stesura del Documento di analisi}
\subsection{Sinossi}
L'attività consiste nella stesura del documento di analisi, nel quale si dà una descrizione dettagliata del problema e si definiscono i requisiti. Il documento contiene anche modelli che descrivono le richieste di cambiamento e una sezione sull'impact analysis.
\begin{table}[ht]
 \begin{tabular}{|p{6cm}|p{2cm}|p{2cm}|p{2cm}|}
    \hline
    \rowcolor{Gray}\textbf{Descrizione}			& \textbf{Autori}		& \textbf{Data inizio}			& \textbf{Data fine}	\\
    \hline
    Esplorazione del dominio applicativo		& Tutti				& 15/05/2014				& 17/05/2014		\\
    \hline
    Descrizione del sistema corrente			& Tutti				& 15/05/2014				& 17/05/2014		\\
    \hline
    Requisiti funzionali				& Tutti				& 15/05/2014				& 17/05/2014		\\
    \hline
    Requisiti funzionali BCC				& Tutti				& 15/05/2014				& 17/05/2014		\\
    \hline
    Requisiti funzionali ECC				& Tutti				& 15/05/2014				& 17/05/2014		\\
    \hline
    Use case model					& Tutti				& 17/05/2014				& 19/05/2014		\\
    \hline
    Object model					& Tutti				& 17/05/2014				& 19/05/2014		\\
    \hline
    Impact analysis					& Tutti				& 17/05/2014				& 19/05/2014		\\
    \hline
 \end{tabular}
\end{table}

\subsection{Obiettivi e risultati}
Gli obiettivi pianificati per la stesura del Documento di analisi sono specificati nel SPMP. Il documento è stato rilasciato nei tempi previsti ed è in uno stato stabile. Future modifiche potrebbero riguardare la sezione sull'impact analysis, che dipende molto dalle scelte di implementazione.

\subsection{Allocazione del lavoro}
La tabella mostra come ogni task è stato ripartito tra i membri del team, e quindi in che percentuale ognuno ha partecipato a una determinata attività.
\begin{table}[ht]
 \begin{tabular}{|p{7.5cm}|p{1cm}|p{1cm}|p{1cm}|p{1cm}|}
  \hline
  \rowcolor{Gray}\textbf{Task}			& \textbf{CB}		& \textbf{GG}		& \textbf{MM}		& \textbf{SS}		\\
  \hline
  Esplorazione del dominio applicativo		& 25\%			& 25\%			& 25\%			& 25\%			\\
  \hline
  Descrizione del sistema corrente		& 10\%			& 40\%			& 10\%			& 40\%			\\
  \hline
  Requisiti funzionali				& 40\%			& 40\%			& 10\%			& 10\%			\\
  \hline
  Modelli					& 40\%			& 10\%			& 40\%			& 10\%			\\
  \hline
  Impact analysis				& 10\%			& 10\%			& 40\%			& 40\%			\\
  \hline
 \end{tabular}
\end{table}





\section{Stesura dell'Object Design Document}
\subsection{Sinossi}
L'attività consiste nella definizione delle naming conventions (e, più in generale, delle coding conventions) da adottare per la scrittura del codice sorgente e nella definizione di interfacce e pacchetti. Nell'attività sono incluse anche le scelte di design (trade-off, design pattern, riuso) e la stesura del documento, nonché la scrittura della javadoc come documentazione delle classi Java che saranno implementate.

\begin{table}[ht]
 \begin{tabular}{|p{6cm}|p{2cm}|p{2cm}|p{2cm}|}
    \hline
    \rowcolor{Gray}\textbf{Descrizione}			& \textbf{Autori}		& \textbf{Data inizio}			& \textbf{Data fine}	\\
    \hline
    Stesura delle naming conventions			& GG, MM			& 20/05/2014				& 24/05/2014		\\
    \hline
    Scelte di design					& GG, MM			& 24/05/2014				& 26/05/2014		\\
    \hline
    Documentazione del codice				& GG, MM			& 26/05/2014				& 13/06/2014		\\
    \hline
 \end{tabular}
\end{table}

\subsection{Obiettivi e risultati}
Gli obiettivi pianificati per la stesura dell'ODD sono specificati nel SPMP. La parte del documento che si prevedeva di rilasciare è stata rilasciata nei tempi previsti. La scrittura della javadoc procede regolarmente (sono state commentate correttamente tutte le classi). In definitiva il documento, seppur non ancora stabile, è in uno stato accettabile.

\subsection{Allocazione del lavoro}
La tabella mostra come ogni task è stato ripartito tra i membri del team, e quindi in che percentuale ognuno ha partecipato a una determinata attività.
\begin{table}[ht]
 \begin{tabular}{|p{7.5cm}|p{1cm}|p{1cm}|p{1cm}|p{1cm}|}
  \hline
  \rowcolor{Gray}\textbf{Task}			& \textbf{CB}		& \textbf{GG}		& \textbf{MM}		& \textbf{SS}		\\
  \hline
  Stesura delle naming conventions		& 0\%			& 50\%			& 50\%			& 0\%			\\
  \hline
  Scelte di design				& 0\%			& 50\%			& 50\%			& 0\%			\\
  \hline
  Documentazione del codice			& 0\%			& 50\%			& 50\%			& 0\%			\\
  \hline
 \end{tabular}
\end{table}



\section{Stesura del Test Plan}
\subsection{Sinossi}
L'attività consiste nella pianificazione del testing e nella scelta delle strategie da adottare per verificare che il software prodotto sia conforme alla specifica. Fa parte dell'attività anche la progettazione dei casi di test.

\begin{table}[ht]
 \begin{tabular}{|p{6cm}|p{2cm}|p{2cm}|p{2cm}|}
    \hline
    \rowcolor{Gray}\textbf{Descrizione}			& \textbf{Autori}		& \textbf{Data inizio}			& \textbf{Data fine}	\\
    \hline
    Definizione dell'approccio				& CB, SS			& 20/05/2014				& 21/05/2014		\\
    \hline
    Progettazione dei casi di test			& CB, SS			& 21/05/2014				& 29/05/2014		\\
    \hline
 \end{tabular}
\end{table}

\subsection{Obiettivi e risultati}
Gli obiettivi pianificati per la stesura del Test Plan sono specificati nel SPMP. Si è rilasciata la prima versione del documento dopo la scadenza prevista (27/05/2014). Il ritardo si è verificato a causa di un prolungamento inaspettato dell'attività di progettazione dei casi di test, che non si è rilevato importante data la presenza di un prolungamento della fase di implementazione.

\subsection{Allocazione del lavoro}
La tabella mostra come ogni task è stato ripartito tra i membri del team, e quindi in che percentuale ognuno ha partecipato a una determinata attività.
\begin{table}[ht]
 \begin{tabular}{|p{7.5cm}|p{1cm}|p{1cm}|p{1cm}|p{1cm}|}
  \hline
  \rowcolor{Gray}\textbf{Task}			& \textbf{CB}		& \textbf{GG}		& \textbf{MM}		& \textbf{SS}		\\
  \hline
  Definizione dell'approccio			& 50\%			& 0\%			& 0\%			& 50\%			\\
  \hline
  Progettazione dei casi di test		& 50\%			& 0\%			& 0\%			& 50\%			\\
  \hline
 \end{tabular}
\end{table}



\section{Stesura del Test Case Specification}
\subsection{Sinossi}
L'attività consiste nella definizione dei casi di test da eseguire per verificare che il software prodotto sia conforme alla specifica, progettati nel TP.

\begin{table}[ht]
 \begin{tabular}{|p{6cm}|p{2cm}|p{2cm}|p{2cm}|}
    \hline
    \rowcolor{Gray}\textbf{Descrizione}			& \textbf{Autori}		& \textbf{Data inizio}			& \textbf{Data fine}	\\
    \hline
    Definizione dei progetti di test			& CB, SS			& 30/05/2014				& 4/06/2014		\\
    \hline
    Definizione dei casi di test			& CB, SS			& 30/05/2014				& 4/06/2014		\\
    \hline
 \end{tabular}
\end{table}

\subsection{Obiettivi e risultati}
Gli obiettivi pianificati per la stesura del Test Case Specification sono specificati nel SPMP. Si è rilasciata la prima versione del documento dopo la scadenza prevista (1/06/2014). Si è dovuto sostanzialmente a ritardo verificatosi nella stesura del TP. 

\subsection{Allocazione del lavoro}
La tabella mostra come ogni task è stato ripartito tra i membri del team, e quindi in che percentuale ognuno ha partecipato a una determinata attività.
\begin{table}[ht]
 \begin{tabular}{|p{7.5cm}|p{1cm}|p{1cm}|p{1cm}|p{1cm}|}
  \hline
  \rowcolor{Gray}\textbf{Task}			& \textbf{CB}		& \textbf{GG}		& \textbf{MM}		& \textbf{SS}		\\
  \hline
  Definizione dei progetti di test		& 10\%			& 0\%			& 0\%			& 90\%			\\
  \hline
  Definizione dei casi di testing		& 90\%			& 0\%			& 0\%			& 10\%			\\
  \hline
 \end{tabular}
\end{table}


\section{Implementazione del codice sorgente}
\subsection{Sinossi}
L'attività consiste nell'implementare quanto specificato nel Documento di analisi, seguendo la progettazione effettuata in fase di object design.

\begin{table}[ht]
 \begin{tabular}{|p{6cm}|p{2cm}|p{2cm}|p{2cm}|}
    \hline
    \rowcolor{Gray}\textbf{Descrizione}			& \textbf{Autori}		& \textbf{Data inizio}			& \textbf{Data fine}	\\
    \hline
    Creazione della struttura di base			& GG, MM			& 26/05/2014				& 6/06/2014		\\
    \hline
    Implementazione delle metriche di progetto		& GG, MM			& 6/05/2014				& 11/06/2014		\\
    \hline
    Implementazione delle metriche di pacchetto		& GG, MM			& 6/05/2014				& 11/06/2014		\\
    \hline
 \end{tabular}
\end{table}

\subsection{Obiettivi e risultati}
Gli obiettivi pianificati per l'implementazione del primo incremento sono specificati nel SPMP. Si è rilasciata la prima versione del documento con un grave ritardo rispetto alla data di rilascio prevista (4/06/2014). Il problema che si è verificato ha riguardato soprattutto il cambiamento di strategia di implementazione. Inizialmente si pensava di poter integrare i cambiamenti direttamente nel progetto esistente, ma dato che questo non funzionava correttamente sarebbe stato impossibile testare le modifiche apportate senza risolvere prima i problemi presenti o aspettare che l'autore li risolvesse. Per evitare di dipendere dal sistema originale, quindi, si è deciso di implementare il sistema da zero, assumendo che il database che il sistema originale riempie sia consistente. In questo modo si può testare il sistema riempiendo il database (simulando l'azione del sistema esistente) e azionando il plugin. Per questo è stato necessario costruire una struttura di base del plugin, per sviluppare la quale sono serviti dei giorni aggiuntivi. Si prevede, comunque, di recuperare con il secondo e terzo incremento, facilitati dalla struttura di base definita.

\subsection{Allocazione del lavoro}
La tabella mostra come ogni task è stato ripartito tra i membri del team, e quindi in che percentuale ognuno ha partecipato a una determinata attività.
\begin{table}[ht]
 \centering
 \begin{tabular}{|p{7.5cm}|p{1cm}|p{1cm}|p{1cm}|p{1cm}|}
  \hline
  \rowcolor{Gray}\textbf{Task}			& \textbf{CB}		& \textbf{GG}		& \textbf{MM}		& \textbf{SS}		\\
  \hline
  Creazione della struttura di base		& 0\%			& 50\%			& 50\%			& 0\%			\\
  \hline
  Implementazione delle metriche di progetto	& 0\%			& 50\%			& 50\%			& 0\%			\\
  \hline
  Implementazione delle metriche di pacchetto	& 0\%			& 50\%			& 50\%			& 0\%			\\
  \hline
 \end{tabular}
\end{table}



\section{Report di testing}
\subsection{Sinossi}
L'attività consiste nel testare il sistema eseguendo i casi di test specificati nel TCS e nell'annotare i risultati dei test in due documenti: il Test Log, che tiene traccia di tutti i casi di test eseguiti e il Test Incident Report, nel quale sono descritti dettagliatamente i casi di test che hanno evidenziato failure del sistema.

\begin{table}[ht]
 \begin{tabular}{|p{6cm}|p{2cm}|p{2cm}|p{2cm}|}
    \hline
    \rowcolor{Gray}\textbf{Descrizione}			& \textbf{Autori}		& \textbf{Data inizio}			& \textbf{Data fine}	\\
    \hline
    Esecuzione dei casi di test				& CB, SS			& 11/06/2014				& 13/06/2014		\\
    \hline
    Stesura del Test Log				& CB, SS			& 11/06/2014				& 13/06/2014		\\
    \hline
    Stesura del Test Incident Report			& CB, SS			& 11/06/2014				& 13/06/2014		\\
    \hline
 \end{tabular}
\end{table}

\subsection{Obiettivi e risultati}
Gli obiettivi pianificati per la reportistica di testing sono specificati nel SPMP. L'attività ha richiesto meno tempo del previsto (2 giorni al posto di 5). L'attività è comunque  terminata in ritardo a causa del ritardo nell'implementazione. Questo spinge a modificare lo schedule, riducendo il tempo previsto per il testing dei prossimi due incrementi. 

\subsection{Allocazione del lavoro}
La tabella mostra come ogni task è stato ripartito tra i membri del team, e quindi in che percentuale ognuno ha partecipato a una determinata attività.
\begin{table}[ht]
 \centering
 \begin{tabular}{|p{7.5cm}|p{1cm}|p{1cm}|p{1cm}|p{1cm}|}
  \hline
  \rowcolor{Gray}\textbf{Task}			& \textbf{CB}		& \textbf{GG}		& \textbf{MM}		& \textbf{SS}		\\
  \hline
  Esecuzione dei casi di test			& 50\%			& 0\%			& 0\%			& 50\%			\\
  \hline
  Stesura del Test Log				& 100\%			& 0\%			& 0\%			& 0\%			\\
  \hline
  Stesura del Test Incident Report		& 0\%			& 0\%			& 0\%			& 100\%			\\
  \hline
 \end{tabular}
\end{table}
